% Copyright (c) 2008-2009 solvethis
% Copyright (c) 2010-2016,2018 Casper Ti. Vector
% Public domain.
%
% 使用前请先仔细阅读 pkuthss 和 biblatex-caspervector 的文档,
% 特别是其中的 FAQ 部分和用红色强调的部分。
% 两者可在终端/命令提示符中用
%   texdoc pkuthss
%   texdoc biblatex-caspervector
% 调出。

% 采用了自定义的(包括大小写不同于原文件的)字体文件名,
% 并改动 ctex.cfg 等配置文件的用户请自行加入 nofonts 选项;
% 其它用户不用加入 nofonts 选项,加入之后反而会产生错误。
\documentclass[UTF8]{pkuthss}
% 如果的确须要使脚注按页编号的话,可以去掉后面 footmisc 包的注释。
% 注意:在启用此设定的情况下,可能要多编译一次以产生正确的脚注编号。
%\usepackage[perpage]{footmisc}
% 使用 biblatex 排版参考文献,并规定其格式(详见 biblatex 的文档)。
\usepackage[backend = biber]{biblatex}

% 对于 linespread 值的计算过程有兴趣的同学可以参考 pkuthss.cls。
\renewcommand*{\bibfont}{\zihao{5}\linespread{1.27}\selectfont}
% 按学校要求设定参考文献列表的段间距。
\setlength{\bibitemsep}{3bp}

\ctexset{
	contentsname = {Contents},
	listfigurename = {List of Figures},
	listtablename = {List of Tables},
	figurename = {Figure},
	tablename = {Table},
	indexname = {Index},
	appendixname = {Appendix},
	part/name = {\partname\space},
	part/number = {\thepart},
	chapter/name = {\chaptername\space},
	chapter/number = {\thechapter},
}

% 设定文档的基本信息。
\pkuthssinfo{
	cthesisname = {博士生毕业论文(初稿)}, 
	ethesisname = {Ph.D. Thesis},
	ctitle = {临界马氏分支过程的脊柱分解和极限定理},
	etitle = {Spine decompositions and limit theorems for some critical Markovian branching processes (draft)},
	cauthor = {孙振尧},
	eauthor = {Zhenyao Sun},
	studentid = {1501110037},
	date = {二〇一八年十二月},
	school = {数学科学学院},
	cmajor = {概率论}, emajor = {Probability},
	direction = {分支粒子系统},
	cmentor = {任艳霞教授}, ementor = {Prof. Yan-Xia Ren},
	ckeywords = {},
	ekeywords = {Branching process, Galton-Watson tree, superprocess, spine decomposition, 2-spine decomposition, asymptotic behavior}
}
% 载入参考文献数据库(注意不要省略“.bib”)。
\addbibresource{thesis.bib}

\begin{document}
	% 以下为正文之前的部分,默认不进行章节编号。
	\frontmatter
	% 此后到下一 \pagestyle 命令之前不排版页眉或页脚。
	\pagestyle{empty}
	% 自动生成封面。
	\maketitle
	% 版权声明。封面要求单面打印,故须新开右页。
	\cleardoublepage
	\include{chap/copy}

	% 此后到下一 \pagestyle 命令之前正常排版页眉和页脚。
	\cleardoublepage
	\pagestyle{plain}
	% 重置页码计数器,用大写罗马数字排版此部分页码。
	\setcounter{page}{0}
	\pagenumbering{Roman}
	% 中西文摘要。
	% Copyright (c) 2014,2016 Casper Ti. Vector
% Public domain.

\begin{cabstract}
	中文摘要。
\end{cabstract}

\begin{eabstract}
	This thesis focusses on the asymptotic behaviors of some Markovian branching processes in the critical regime.
	Properties and relationships between the multi-spine theory and the limiting theory of some Markovian branching processes are considered.
	In particular, we study systematically the spine decompositions of critical Galton-Watson trees and critical superprocesses, and their Kolmogrov type, Yaglom type and Slack type results. 

	We begin by proposing a two-spine decomposition of the critical Galton-Watson tree and useing that decomposition to give a new probabilistic proof of Yaglom’s theorem.
	
	Next, we establish a spine decomposition theorem and a 2-spine decomposition theorem for some critical superprocesses. These two kinds of decompositions are unified as a decomposition theorem for size-biased Poisson random measures. We use these decompositions to give probabilistic proofs of the asymptotic behavior of the survival probability and Yaglom’s exponential limit law for some critical superprocesses with second moments.

	Then, we proof that the characteristic functions of superprocesses satisfy a mild complex-valued integral equation using the spine decomposition theory developed here. Using this equation we estimate the tail probability of superprocess with stable branching.

	Finally, we consider a critical superprocess $\{X;\mathbf P_\mu\}$ with general spatial motion and spatially dependent stable branching mechanism with lowest stable index $\gamma_0 > 1$. We show that, under some conditions, $\mathbf P_{\mu}(\|X_t\|\neq 0)$ converges to $0$ as $t\to \infty$ and is regularly varying with index $(\gamma_0-1)^{-1}$. Then we proof the Slack type result that for a large class of non-negative testing functions $f$, the distribution of $\{X_t(f);\mathbf P_\mu(\cdot|\|X_t\|\neq 0)\}$, after appropriate rescaling, converges weakly to a positive random variable $\mathbf z^{(\gamma_0-1)}$ with Laplace transform $E[e^{-u\mathbf z^{(\gamma_0-1)}}]=1-(1+u^{-(\gamma_0-1)})^{-1/(\gamma_0-1)}.$
\end{eabstract}

% vim:ts=4:sw=4

	% 自动生成目录。
	\tableofcontents

	% 以下为正文部分,默认要进行章节编号。
	\mainmatter
	% 各章节。
	% Copyright (c) 2014,2016,2018 Casper Ti. Vector
% Public domain.

\chapter{Introduction}
Test text.

% vim:ts=4:sw=4

	% Copyright (c) 2014,2016 Casper Ti. Vector
% Public domain.

\chapter{A 2-spine Decomposition of the Critical Galton-Watson Tree}

\section{Introduction}
\subsection{Model}
\label{sec:model}
Consider a critical Galton-Watson process
$(Z_n)_{n\ge 0}$ 	with $Z_0 = 1$
and offspring distribution $\mu$ on $\mathbb N_0 : = \{0,1,\dots\}$ which has mean $1$ and finite variance $\sigma^2>0$, i.e.,
\begin{equation}\label{eq:mean}
\sum_{k=0}^\infty k \mu(k)	=1
\end{equation}
and
\begin{equation}\label{eq:variance}
0	
<	\sigma^2
:=	\sum_{k=0}^\infty  (k-1)^2 \mu(k)
=	\sum_{k=0}^\infty k(k-1) \mu(k)
<	\infty.
\end{equation}
For simplicity,
we will refer to $(Z_n)_{n\geq 0}$ as a  \emph{$\mu$-Galton-Watson process}.
It is well known that
\begin{thm}[\cite{KestenNeySpitzer1966The-Galton-Watson}] 
\label{thm: Kolmogrov and Yaglom theorem}
	For a $\mu$-Galton-Watson process $(Z_n)_{n\geq 0}$
	satisfying \eqref{eq:mean} and \eqref{eq:variance}, we have
	\begin{enumerate}
		\item \label{thm:kolmogorov}
		$n P (Z_n>0) \xrightarrow[n \to \infty]{} 2/\sigma^2;$
		\item \label{thm:yaglom}
		$\{n^{-1}Z_n; P(\cdot | Z_n>0)\}\xrightarrow[n \to \infty]{d} Y,$
	\end{enumerate}
	where $Y$ is an exponential random variable with mean $\sigma^2/2$.
\end{thm}

Under a third moment assumption, assertions \eqref{thm:kolmogorov} and \eqref{thm:yaglom} of Theorem \ref{thm: Kolmogrov and Yaglom theorem} are due to \cite{Kolmogorov1938Zur-losung} and \cite{Yaglom1947Certain} respectively.
Theorem \ref{thm: Kolmogrov and Yaglom theorem}(2) is usually called Yaglom's theorem.
For probabilistic proofs of the above results, we refer our readers to
\cite{Geiger1999Elementary}, \cite{Geiger2000A-new-proof} and \cite{LyonsPemantlePeres1995Conceptual}.

In \cite{LyonsPemantlePeres1995Conceptual}, Lyons, Pemantle and Peres gave a probabilistic proof of Theorem \ref{thm: Kolmogrov and Yaglom theorem} using the so-called size-biased $\mu$-Galton-Watson tree.
In this note, by \emph{size-biased transform} we mean the following:
Let $X$ be a random variable
and $g(X)$ be a Borel function of $X$ with $P(g(X) \geq 0) = 1$ and $E[g(X)]\in (0,\infty)$.
We say a random variable $W$ is
a $g(X)$-size-biased transform (or simply $g(X)$-transform) of $X$ if
\[
E[f( W )]
= \frac{ E[g(X)f(X)]}{E[g(X)]}
\]
for each positive Borel function $f$.
An $X$-transform of $X$ is sometimes called a size-biased transform of $X$.

We now recall the size-biased $\mu$-Galton-Watson tree introduced in \cite{LyonsPemantlePeres1995Conceptual}.
Let $L$ be a random variable with distribution $\mu$.
Denote by $\dot L$ an \emph{$L$-transform} of $L$.
The celebrated \emph{size-biased $\mu$-Galton-Watson tree} is then constructed as follows:
\begin{itemize}
	\item
	There is an initial particle which is marked.
	\item
	Any marked particle gives independent birth to a random number of children according to $\dot L$. Pick one of those children randomly as the new marked particle while leaving the other children as unmarked particles.
	\item
	Any unmarked particle gives
	birth independently to a random number of unmarked children according to $L$.
	\item
	The evolution goes on.
\end{itemize}

Notice that the marked particles form a descending family line which will be referred to as the \emph{spine}.
Define $\dot Z_n$ as the population of the $n$th generation in the size-biased tree.
It is proved in \cite{LyonsPemantlePeres1995Conceptual} that the process $(\dot Z_n)_{n\ge 0}$ is a martingale transform of the process $(Z_n)_{n\ge 0}$ via the martingale $(Z_n)_{n\ge 0}.$
That is, for any generation number $n$ and any bounded Borel function $g$ on $\mathbb N_0^{n}$,
\begin{equation}
\label{eq:htransformation}
E [ g ( \dot Z_1, \dots, \dot Z_n) ]
= \frac { E[ Z_n g( Z_1, \dots, Z_n)]} {E [ Z_n]}.
\end{equation}

It is natural to consider probabilistic proofs of analogous results of Theorem \ref{thm: Kolmogrov and Yaglom theorem} for more general critical branching processes.
Vatutin and  Dyakonova \cite{VatutinDyakonova2001The-survival} gave a probabilistic proof of Theorem \ref{thm: Kolmogrov and Yaglom theorem}(1) for multitype critical branching processes.
As far as we know, there is no probabilistic proof of Yaglom's theorem for multitype critical branching processes.
It seems that it is difficult to adapt the probabilistic proofs in \cite{Geiger2000A-new-proof} and \cite{LyonsPemantlePeres1995Conceptual} for monotype branching processes to more general models, such as multitype branching processes, branching Hunt processes and superprocesses.

In this note, we propose a $k(k-1)$-type size-biased $\mu$-Galton-Watson tree equipped with a two-spine skeleton, which serves as a change-of-measure of the original $\mu$-Galton-Watson tree;
and with the help of this two-spine technique, we give a new probabilistic proof of Theorem \ref{thm: Kolmogrov and Yaglom theorem}(2), i.e. Yaglom's theorem.
The main motivation for developing this new proof for the classical Yaglom's theorem is that this new method is generic, in the sense that it can be generalized to more complicated critical branching systems.
In fact, in our	follow-up
paper \cite{RenSongSun2017Spine}, we show that, in a similar spirit, a two-spine structure can be constructed for a class of critical superprocesses, and a probabilistic proof of a Yaglom type theorem can be obtained for those processes.



Another aspect of our new proof is that we take advantage of a fact that the exponential distribution can be characterized by a particular $x^2$-type size-biased distributional equation.
An intuitive explanation of our method,
and a comparison with the methods of \cite{Geiger2000A-new-proof} and \cite{LyonsPemantlePeres1995Conceptual}, are
made in the next subsection.
We think this new point of view of convergence to the exponential law provides an alternative insight on the classical Yaglom's theorem.

We now give a formal construction of our $k(k-1)$-type size-biased $\mu$-Galton-Watson tree.
Denote by $\dot L$ an $L$-transform of $L$, and by $\ddot L$ an $L(L-1)$-transform of $L$.
Fix a generation number $n$ and pick a random generation number $K_n$ uniformly among $\{0,\dots,n-1\}$.
The \emph{$k(k-1)$-type size-biased $\mu$-Galton-Watson tree with height $n$} is then defined as a particle system such that:
\begin{itemize}
	\item
	There is an initial particle which is marked.
	\item
	Before or after generation $K_n$, any marked particle gives birth independently to a random number of children according to $\dot L$.
	Pick one of those children randomly as the new marked particle while leaving the other children as unmarked particles.
	\item
	The marked particle at generation $K_n$, however, gives birth, independent of other particles, to a random number of children according to $\ddot L$.
	Pick two different particles randomly among those children as the new marked particles while leaving the other children as unmarked particles.
	\item
	Any unmarked particle gives birth independently to a random number of unmarked children according to $L$.
	\item
	The system stops at generation $n$.
\end{itemize}

If we track all the marked particles, it is clear that they form a \emph{two-spine skeleton} with $K_n$ being the last generation where those two spines are together.
It would be helpful to consider this skeleton as two disjoint spines,
where \emph{the longer spine} is a family line from generation $0$ to $n$ and \emph{the shorter spine} is a family line from generation $K_n+1$ to $n$.

For any $0\le m \le n$, denote by $\ddot Z_m^{(n)}$ the population of the $m$th generation in the $k(k-1)$-type size-biased $\mu$-Galton-Watson tree with height $n$.
The main reason for proposing such a model is that the process $(\ddot Z_m^{(n)})_{0\le m\le n}$ can be viewed as
a $Z_n(Z_n-1)$-transform of the process $(Z_m)_{0\le m\le n}$.
This is made precise in the result below which will be proved in Section \ref{sec:spacesandmeasures}.
\begin{thm}
	\label{thm: change of measure}
	Let $(Z_m)_{m\ge 0}$ be a $\mu$-Galton-Watson process and $(\ddot Z_m^{(n)})_{0\le m\le n}$ be the population of a $k(k-1)$-type size-biased $\mu$-Galton-Watson tree with height $n$.
	Suppose that $\mu$ satisfies \eqref{eq:mean} and \eqref{eq:variance}.
	Then, for any bounded Borel function $g$ on $\mathbb N^{n}_0$,
	\[
	E[ g ( \ddot Z_1^{(n)}, \dots, \ddot Z_n^{(n)})]
	=
	\frac{ E[ Z_n(Z_n-1) g( Z_1, \dots, Z_n)]} {E [ Z_n ( Z_n - 1)]}.		
	\]
\end{thm}

The idea of considering a branching particle system with more than one spine is not new.
A particle system with $k$ spines  was constructed in \cite{HarrisRoberts2017The-many-to-few} and used in the  many-to-few formula for branching Markov processes and branching random walks.
Inspired by \cite{HarrisRoberts2017The-many-to-few}, we use a two-spine model to characterize the $k(k-1)$-type size-biased branching process.


\subsection{Methods.}
\label{sec: Methods}
Suppose that $X$ is a non-negative random variable
with $E[X] \in (0,\infty)$. Then
its distribution conditioned on $\{ X > 0\}$ can be characterized by its conditional expectation $E[X|X>0]$ and its size-biased transform $\dot X$.
In fact, for each $\lambda \geq 0$,
\begin{equation}
\label{eq: conditional and size-biased transform}
\begin{split}
&E[1-e^{-\lambda X}|X>0]
= \frac{E[1-e^{-\lambda X}]}{P(X>0)}
\\&\quad = \frac{1}{P(X>0)}\int_0^\lambda E[Xe^{-s X}]ds = E[X|X>0]\int_0^\lambda E[e^{-s \dot X}]ds.
\end{split}
\end{equation}
As a consequence,
Theorem \ref{thm: Kolmogrov and Yaglom theorem}	is equivalent to
\begin{equation}
\label{eq: convergence of conditional expectation}
E\big[\frac{Z_n}{n}| Z_n > 0\big]
\xrightarrow[n\to \infty]{} \frac{\sigma^2}{2}
\end{equation}
and
\begin{equation}
\label{eq: convergence after size-biased}
E[e^{-s \frac{\dot Z_n}{n}}]
\xrightarrow[n\to \infty]{} E[e^{-s \dot Y}].
\end{equation}
where $\dot Y$ is a $Y$-transform
of the exponential random variable $Y$.
Indeed, since $E[Z_n] = 1$, \eqref{eq: convergence of conditional expectation} is equivalent to Theorem \ref{thm: Kolmogrov and Yaglom theorem}(\ref{thm:kolmogorov});
and assuming \eqref{eq: convergence of conditional expectation},
according to \eqref{eq: conditional and size-biased transform}, we can see that \eqref{eq: convergence after size-biased} is equivalent to Theorem \ref{thm: Kolmogrov and Yaglom theorem}(\ref{thm:yaglom}).
In Section \ref{sec: proofs}, for completeness, we will simplify
the argument of \cite{Geiger1999Elementary} and \cite{VatutinDyakonova2001The-survival},
and give a proof of Theorem \ref{thm: Kolmogrov and Yaglom theorem}(\ref{thm:kolmogorov}).

Our method of proving
\eqref{eq: convergence after size-biased}
takes advantage of a fact that the exponential distribution is characterized by an $x^2$-type size-biased distributional equation.
This is made precise in the next lemma, which will be proved in Section \ref{sec: proofs}:
\begin{lem} \label{lem: our equation}
	Let $Y$ be a strictly positive random variable with finite second moment.
	Then $Y$ is exponentially distributed if and only if
	\begin{equation}
	\label{eq: x2 type size-biased equation for exponential distribution}
	\ddot Y \overset{d}
	= \dot Y + U \cdot \dot Y',
	\end{equation}
	where $\dot Y$ and $\dot Y'$ are both $Y$-transforms of  $Y$,
	$\ddot Y$ is a $Y^2$-transform of $Y$,
	$U$ is a uniform random variable on $[0,1]$, and $\dot Y$, $\dot Y'$, $\ddot Y$
	and $U$ are independent.
\end{lem}	
	With this lemma and Theorem \ref{thm: change of measure}, we can give an intuitive explanation of the exponential convergence in Yaglom's Theorem.
	From the construction of the $k(k-1)$-type size-biased $\mu$-Galton-Watson tree $(\ddot Z^{(n)}_m)_{0\le m\le n}$, we see that the population $\ddot Z^{(n)}_n$ in the $n$th generation can be separated into two parts:
	descendants from the longer spine and descendants from the shorter spine.
	Due to their construction, the first part, the descendants from the longer spine at generation $n$, is distributed approximately like $\dot Z_n$, while the second part, the descendants from the shorter spine at generation $n$,
	is distributed approximately like $\dot Z_{ \lfloor U\cdot n \rfloor}$.
	Those two parts are approximately independent of each other.
	So, after a renormalization, we have roughly that
\begin{equation}
\label{eq: Our insight}
	\frac{\ddot Z_n^{(n)}}{n}
	 \overset{d} \approx \frac{\dot Z_n}{n} + U \cdot \frac{   \dot Z'_{  \lfloor U n \rfloor  }   }    {   Un   },
\end{equation}
where the process $(\dot Z'_m)$ is an independent copy of $(\dot Z_m)$.
Suppose that $\dot Z_n/n$ converges weakly to a random variable $\dot Y$, and $\ddot Z_n/n$ converges weakly to a random variable $\ddot Y$.
Then, according to \cite[Lemma 4.3]{LyonsPemantlePeres1995Conceptual}, $\ddot Y$ is a size-biased transform of $\dot Y$.
Therefore, letting $n\to\infty$ in \eqref{eq: Our insight},
$\dot Y$ should satisfy \eqref{eq: x2 type size-biased equation for exponential distribution}, which, by Lemma \ref{lem: our equation}, suggests that \eqref{eq: convergence after size-biased} is true.

It is interesting to compare this method of proving exponential convergence with the methods
used in \cite{Geiger2000A-new-proof} and \cite{LyonsPemantlePeres1995Conceptual}.
In \cite{LyonsPemantlePeres1995Conceptual}, Lyons, Pemantle and Peres characterize the exponential distribution by a different
but well-known $x$-type size-biased distributional equation:
A nonnegative random variable $Y$ with positive finite mean is exponentially distributed if and only if it satisfies that
\begin{equation}
\label{eq: Lyons' distributional equation}
Y 		\overset{d}= U \cdot \dot Y
\end{equation}
	where $\dot Y$ is a $Y$-transform of $Y$,  and $U$ is a uniform random variable on $[0,1]$, which is independent of $\dot Y$.
	With the help of the size-biased tree,
	they then show that $\lceil U \cdot \dot Z_n \rceil$ is distributed approximately like $Z_n$ conditioned on $\{Z_n > 0\}$.
	So, after a renormalization, they have roughly that
\begin{equation}
\label{eq: Lyons' insight}
\Big\{\frac{Z_n}{n} ; P(  \cdot| Z_n > 0) \Big\}
\overset{d}{\approx} U \cdot \frac{ \dot Z_n}{n}.
\end{equation}
Suppose that $\{Z_n/n; P(\cdot | Z_n > 0)\}$ converge weakly to a random variable $Y$, and $\dot Z_n /n$ converge weakly to a random variable $\dot Y$.
Then, according to \cite[Lemma 4.3]{LyonsPemantlePeres1995Conceptual}, $\dot Y$ is the size-biased transform of $Y$.
Therefore, letting $n\to \infty$ in \eqref{eq: Lyons' insight},
$Y$ should satisfy \eqref{eq: Lyons' distributional equation}, which suggests that $Y$ is exponentially distributed.

In \cite{Geiger2000A-new-proof}, Geiger characterizes the exponential distribution by another distributional equation:
If $Y^{(1)}$ and $Y^{(2)}$ are independent copies of
a random variable $Y$ with positive finite variance,
and $U$ is an independent uniform random variable on $[0,1]$, then $Y$ is exponentially distributed if and only if
\begin{equation}
\label{eq: Geiger's equation}
Y	\overset{d} = U (Y^{(1)} + Y^{(2)}).
\end{equation}
Geiger then shows that for $(Z_n)$, conditioned on non-extinction at generation $n$,
the distribution of the generation of the most recent common ancestor (MRCA) of the particles at generation $n$ is asymptotically uniform among $\{0,1,\dots,n\}$ (a result due to \cite{Zubkov1975Limit}, see also \cite{Geiger1999Elementary}), and there are asymptotically two children of
the MRCA, each with at least 1 descendant in generation $n$.
After a renormalization, roughly speaking, Geiger has that
\begin{equation}
\label{eq: Geiger's insight}
	\Big\{\frac{Z_n}{n} ; P(  \cdot| Z_n > 0) \Big\}
	\overset{d}{\approx} U \cdot \frac{ Z^{(1)}_{  \lfloor Un \rfloor  }}{Un} + U \cdot \frac{ Z^{(2)}_{ \lfloor Un \rfloor }}{Un} ,
\end{equation}
where for each $m$, $Z_m^{(1)}$ and $Z_m^{(2)}$ are independent copies of $\{Z_m; P(\cdot | Z_m > 0)\}$.
Therefore, if $\{Z_n/n; P(\cdot| Z_n > 0)\}$ converges weakly to a random variable $Y$, then $Y$ should satisfy \eqref{eq: Geiger's equation}, which suggests that $Y$ is exponentially distributed.

	From this comparison, we see that all the methods mentioned above share one similarity: They all establish the exponential convergence via some particular distributional equation.
	However, since the equations \eqref{eq: x2 type size-biased equation for exponential distribution}, \eqref{eq: Lyons' distributional equation} and \eqref{eq: Geiger's equation} are different, the actual way of proving the convergence varies.
	In \cite{LyonsPemantlePeres1995Conceptual}, an elegant tightness argument is made along with \eqref{eq: Lyons' insight}.
	However, it seems that this tightness argument is not suitable for \eqref{eq: Geiger's insight}, due to a property that the conditional convergence for some subsequence $Z_{n_k}/n_k$ implies the convergence of $U \cdot \dot Z_{n_k}/n_k$,
	 but does not imply the convergence of $Z^{(i)}_{ \lfloor Un_k \rfloor }/Un_k, i=1, 2$.
	Instead, a contraction type argument in the $L^2$-Wasserstein metric is used in \cite{Geiger2000A-new-proof}.

For similar reasons,
in this note,
to actually prove the exponential convergence using \eqref{eq: Our insight} and \eqref{eq: x2 type size-biased equation for exponential distribution}, some efforts also must be made.
We observe that the distributional equation \eqref{eq: Our insight} admits
a so-called size-biased add-on structure, which is related to L\`evy's
theory of infinitely divisible distributions: Suppose that $X$ is a nonnegative random variable with
$ a := E [X]\in (0,\infty)$; then
$X$ is infinitely divisible if and only if there exists a nonnegative random variable $A$ independent of $X$ such that $\dot X 	\overset{d} = X + A$.
In fact,
the Laplace exponent of $X$ can be expressed as
\[
-\ln E[ e^{-\lambda X}]
=  a \alpha(\{0\}) \lambda+ a \int_{(0,\infty)} \frac{1 - e^{-\lambda y}}{y} \alpha(dy),
\]
where $\alpha$ is the distribution of $A$.
Moreover, if $A$ is strictly positive, then
\[
-\ln E[ e^{-\lambda X}]
=  a  \int_0^\lambda E [e^{-s A}] ds.
\]
From this point of view, after considering the Laplace transforms of
\eqref{eq: Our insight} and \eqref{eq: x2 type size-biased equation for exponential distribution}, we can establish the convergence of $E[e^{-\lambda \dot Z_n/n}]$ to $E[e^{-\lambda \dot {Y}}]$, which will eventually lead us to Yaglom's theorem.
This is made precise in Section \ref{sec: proofs}.
A similar type of argument is also  used in our
follow-up
paper \cite{RenSongSun2017Spine} for critical superprocesses.

\section{Trees and their decompositions}
\label{sec:preliminary}
\subsection{Spaces and measures}
\label{sec:spacesandmeasures}
In this subsection, we give a proof of Theorem \ref{thm: change of measure}.
Consider \emph{particles} as elements in the space
\[
\mathcal U
:=
\{\emptyset\}\cup\bigcup_{k=1}^\infty \mathbb N^k,
\]
where $\mathbb N:=\{1,2,\dots\}$.
Therefore elements in $\mathcal U$ are of the form 213, which we read as the individual being the 3rd child of the 1st child of the 2nd child of the initial ancestor $\emptyset$.
For two particles $u=u_1\dots u_n, v=v_1\dots v_m\in\mathcal U$, $uv$ denotes the concatenated particle $uv:=u_1\dots u_nv_1\dots v_m$.
We use the convention $u\emptyset = \emptyset u = u$ and $u_1\dots u_n=\emptyset$ if $n=0$.
For any particle $u:=u_1\dots u_{n-1}u_n$, we define its \emph{generation} as $| u |:=n$ and its \emph{parent particle} as $\overleftarrow{u}:=u_1\dots u_{n-1}$.
For any particle $u \in \mathcal U$ and any subset $\mathbf a \subset \mathcal U$, we define the \emph{number of children of $u$ in $\mathbf a$} as $l_u(\mathbf a) := \#\{\alpha\in \mathbf a:\overleftarrow{\alpha}=u\} $.
We also define the \emph{height} of $\mathbf a$ as $|\mathbf a|:=\sup_{\alpha\in \mathbf a}|\alpha|$ and its \emph{population in the $n$th generation} as $X_n(\mathbf a):=\#\{u\in \mathbf a:|u|=n\}$.
A \emph{tree} $ \mathbf t $ is defined as a subset of $\mathcal U$ such that there exists an $\mathbb N_0$-valued sequence $(l_u)_{u\in \mathcal U}$,
indexed by $\mathcal U$, satisfying
\[
\mathbf t
=\{u_1\dots u_m\in \mathcal U: m\ge 0, u_j\leq l_{u_1\dots u_{j-1}}, \forall  j=1,\dots,m\}.
\]
A \emph{spine} $ \mathbf v$ on a  tree $ \mathbf t $ is defined as a sequence of particles $\{v^{(k)}:k=0,1,\dots,| \mathbf t |\}\subset \mathbf t $ such that $v^{(0)}=\emptyset$ and $\overleftarrow{v^{(k)}}=v^{(k-1)}$ for any $k=1,\dots, | \mathbf t |$.
In the case that $| \mathbf t |=\infty$, we simply write $k=0,1,\dots$ as $k=0,1,\dots, | \mathbf t |$.

Fix a generation number $n\in \mathbb N$. Define the following spaces:
\begin{itemize}
	\item
	\emph{The space of trees with height no more than $n$},
	\[
	\mathbb T_{\leq n}
	:=\{ \mathbf t : \mathbf t \text{ is a tree with }| \mathbf t | \leq n\}.
	\]
	\item
	\emph {The space of $n$-height trees with one distinguishable spine},
	\[
	\dot{\mathbb T}_n
	:=\{( \mathbf t , \mathbf v): \mathbf t  \text{ is a tree with } | \mathbf t |=n,  \mathbf v \text{ is a spine on }  \mathbf t \}.
	\]
	\item
	\emph{The space of $n$-height trees with two different distinguishable spines},
	\[
	\ddot{\mathbb T}_n
	:=\{( \mathbf t , \mathbf v, \mathbf v'):( \mathbf t , \mathbf v)\in\dot{\mathbb T}_n,( \mathbf t , \mathbf v')\in\dot{\mathbb T}_n, \mathbf v\neq \mathbf v'\}.
	\]
\end{itemize}

Let $(L_u)_{u\in\mathcal U}$ be a collection of independent random variables with law $\mu$, indexed by $\mathcal U$.
Denote by $T$ the random tree defined by
\[
T
:=\{u_1\dots u_m\in \mathcal U: 0\le m\le n, u_j\leq L_{u_1\dots u_{j-1}},\forall j=1,\dots,m\}.
\]
We refer to $T$ as a \emph{$\mu$-Galton-Watson tree with height no more than n} since its population $(X_m(T))_{0\le m\le n}$ is a $\mu$-Galton-Watson process stopped at generation $n$.
Define the \emph{$\mu$-Galton-Watson measure $\mathbf G_n$} on $\mathbb T_{\leq n}$ as the law of the random tree $T$.
That is, for any $ \mathbf t \in\mathbb T_{\leq n}$,
\[
\mathbf G_n( \mathbf t )
:=P(T= \mathbf t )
=P(L_u=l_u( \mathbf t )\text{ for any } u\in \mathbf t  \text{ with }|u|<n)
=\prod_{u\in  \mathbf t :|u|<n}\mu(l_u( \mathbf t )).
\]

Recall that $\dot L$ is an $L$-transform of $L$.
Define $\dot C$ as a random number which, conditioned on $\dot L$, is uniformly distributed on $\{1,\dots,\dot L\}$.
Independent of $(L_u)_{u\in\mathcal U}$, let $(\dot L_u,\dot C_u)_{u\in \mathcal U}$ be a collection of independent copies of $(\dot L,\dot C)$, indexed by $\mathcal U$.
We then use $(L_u)_{u\in\mathcal U}$ and $(\dot L_u,\dot C_u)_{u\in\mathcal U}$ as the building blocks to construct the size-biased $\mu$-Galton-Watson tree $\dot T$ and its distinguishable spine $\dot V$ following the steps described in Section \ref{sec:model}.
We use $L_u$ as the number of children of particle $u$ if $u$ is unmarked and use $\dot L_u$ if $u$ is marked.
In the latter case, we always set the
$\dot C_u$-th child of $u$, i.e. particle $u \dot C_u$,
as the new marked particle.
For convenience, we stop the system at generation $n$. To be precise, the random spine $\dot V$ is defined by
\[
\dot V
:=\{v_1\dots v_m\in \mathcal U:0\le m\le n, v_j=\dot C_{v_1\dots v_{j-1}},\forall j=1,\dots,m\},
\]
and the random tree $\dot T$ is defined by
\[
\dot T
:=\{u_1\dots u_m\in\mathcal U: 0\le m\le n,u_j\leq \tilde L_{u_1\dots u_{j-1}},\forall j=1,\dots,m\},
\]
where, for any $u\in\mathcal U$, $\tilde L_u:=L_u\mathbf 1_{u\not\in \dot V}+\dot L_u\mathbf 1_{u\in \dot V}$.

We now consider the distribution of the $\dot{\mathbb T}_n$-valued random element $(\dot T,\dot V)$.
For any $( \mathbf t , \mathbf v)\in\dot{\mathbb T}_n$, the event $\{(\dot T,\dot V)=( \mathbf t , \mathbf v)\}$ occurs if and only if:
\begin{itemize}
	\item
	$L_u=l_u( \mathbf t )$ for each $u\in  \mathbf t \setminus \mathbf v$ with $| u |<n$ and
	\item
	$(\dot L_{v_1\dots v_m},\dot C_{v_1\dots v_m})=(l_{v_1\dots v_m}( \mathbf t ),v_{m+1})$ for each $v_1\dots v_{m+1}\in \mathbf v$ with $0\le m\le n-1$.
\end{itemize}
Therefore, the distribution of $(\dot T,\dot V)$ can be determined by
\begin{equation}
\label{eq:treespinemeasure}
P((\dot T,\dot V)=( \mathbf t , \mathbf v))
=\prod_{u\in  \mathbf t \setminus \mathbf v:|u|<n}\mu(l_u( \mathbf t ))
\cdot \prod_{u\in  \mathbf v:| u| <n}l_u( \mathbf t )\mu(l_u( \mathbf t ))\frac{1}{l_u( \mathbf t )}
= \mathbf G_n( \mathbf t ).
\end{equation}

The \emph{size-biased $\mu$-Galton-Watson measure $\dot {\mathbf G}_n$} on $\mathbb T_{\leq n}$ is then defined as the law of the $\mathbb T_{\leq n}$-valued random element $\dot T$. That is, for any $ \mathbf t \in\mathbb T_{\leq n}$,
\begin{equation}
\label{eq:sizebiasedGWmeasure}
\begin{split}
\dot {\mathbf G}_n( \mathbf t )
&:= P(\dot T= \mathbf t )
= \sum_{ \mathbf v:( \mathbf t , \mathbf v)\in \dot{\mathbb T}_n} P((\dot T,\dot V)=( \mathbf t , \mathbf v))
\\&= \#\{ \mathbf v:( \mathbf t , \mathbf v)\in \dot{\mathbb T}_n\} \cdot \mathbf G_n( \mathbf t )
= X_n( \mathbf t ) \cdot \mathbf G_n( \mathbf t ).
\end{split}
\end{equation}

Equations \eqref{eq:treespinemeasure}, \eqref{eq:sizebiasedGWmeasure} and their consequence \eqref{eq:htransformation} were first obtained in \cite{LyonsPemantlePeres1995Conceptual}.
We use these equations to help us to understand how the $k(k-1)$-type size-biased $\mu$-Galton-Watson tree can be represented.

Recall that $K_n$ is a random generation number uniformly distributed on $\{0,\dots,n-1\}$,
and $\ddot L$ is an $L(L-1)$-transform of $L$.
Define $(\ddot C,\ddot C')$ as a random vector which, conditioned on $\ddot L$, is uniformly distributed on $\{(i,j)\in\mathbb N^2:1\leq i\neq j\leq \ddot L\}$.
Suppose that $(L_u)_{u\in\mathcal U}, (\dot L_u,\dot C_u)_{u\in \mathcal U}$, $(\ddot L,\ddot C,\ddot C')$ and $K_n$ are independent of each other.
We now use these elements to build the $k(k-1)$-type size-biased $\mu$-Galton-Watson tree $\ddot T$ and its two different distinguishable spines $\ddot V$ and $\ddot V'$ following the steps described in Section \ref{sec:model}.
Write $C_u:=\dot C_u\mathbf 1_{|u|\neq K_n}+\ddot C\mathbf 1_{|u|=K_n}$ and $C'_u:=\dot C_u\mathbf 1_{|u|\neq K_n}+\ddot C'\mathbf 1_{|u|=K_n}$.
We define the random spines $\ddot V$ and $\ddot V'$ as
\[ \begin{split}
\ddot V
&:= \{v_1\dots v_m\in \mathcal U:0\le m\le n, v_j= C_{v_1\dots v_{j-1}},\forall j=1,\dots,m\},
\\ \ddot V'
&:= \{v_1\dots v_m\in \mathcal U:0\le m \le n, v_j= C'_{v_1\dots v_{j-1}},\forall j=1,\dots,m\},
\end{split}\]
and the random tree $\ddot T$ as
\[
\ddot T
:=
\{u_1\dots u_m\in\mathcal U: 0\le m\le n,u_j\leq L''_{u_1\dots u_{j-1}},\forall j=1,\dots,m\},
\]
where, for any $u\in\mathcal U$, $L''_u:=L_u \mathbf 1_{u\not\in \ddot V\cup\ddot V'}+\dot L_u \mathbf 1_{u\in \ddot V\cup\ddot V',|u|\neq K_n}+\ddot L\mathbf 1_{u\in \ddot V\cup\ddot V',|u|=K_n}$.

We now consider the distribution of $(\ddot T,\ddot V,\ddot V')$.
For any $( \mathbf t , \mathbf v, \mathbf v')\in\ddot {\mathbb T}_n$, the event $\{(\ddot T,\ddot V,\ddot V')=( \mathbf t , \mathbf v, \mathbf v')\}$ occurs if and only if:
\begin{itemize}
	\item
	$K_n=k_n:=| \mathbf v\cap \mathbf v'|$,
	\item
	$L_u=l_u( \mathbf t )$ for each $u\in  \mathbf t \setminus( \mathbf v\cup \mathbf v')$ with $| u| <n$,
	\item
	$(\dot L_{v_1\dots v_m},\dot C_{v_1\dots v_m})=(l_{v_1\dots v_m}( \mathbf t ),v_{m+1})$ for each $v_1\dots v_mv_{m+1}\in \mathbf v\cup \mathbf v'$ with $k_n\neq m<n$ and
	\item
	$(\ddot L,\ddot C,\ddot C')=(l_{v_1\dots v_{k_n}}( \mathbf t ),v_{k_n+1},v'_{k_n+1})$ for $v_1\dots v_{k_n}v_{k_n+1}\in \mathbf v$ and $v_1\dots v_{k_n}v'_{k_n+1}\in \mathbf v'$.
\end{itemize}
Using this analysis, we get that
\[\begin{split}
P\big((\ddot T,\ddot V,\ddot V')=( \mathbf t , \mathbf v, \mathbf v')\big)
&=\frac{1}{n} \cdot \prod_{u\in  \mathbf t \setminus( \mathbf v\cup  \mathbf v'):|u|<n} \mu(l_u( \mathbf t )) \cdot \prod_{u\in  \mathbf v\cup  \mathbf v':k_n\neq|u|<n}l_u( \mathbf t ) \mu(l_u( \mathbf t ))\frac{1}{l_u( \mathbf t )}
\\&\qquad \cdot \prod_{u\in  \mathbf v \cup  \mathbf v':|u|=k_n}\frac{l_u( \mathbf t )(l_u( \mathbf t )-1) \mu(l_u( \mathbf t ))}{\sigma^2}\frac{1}{l_u( \mathbf t )(l_u( \mathbf t )-1)}\\
&= \frac{1}{n\sigma^2} \mathbf G_n( \mathbf t ).
\end{split}\]

The \emph{$k(k-1)$-type size-biased $\mu$-Galton-Watson measure $\ddot{\mathbf G}_n$} on $\mathbb T_{\leq n}$ is then defined as the law of the random element $\ddot T$. That is, for any $ \mathbf t \in\mathbb T_{\leq n}$,
\begin{equation}
\label{eq:k(k-1)typesizebiasedGWmeasure}
\begin{split}
\ddot{\mathbf G}_n( \mathbf t )
&:= P(\ddot T= \mathbf t )
= \sum_{( \mathbf v, \mathbf v'):( \mathbf t , \mathbf v, \mathbf v')\in \ddot {\mathbb T}_n} P\big((\ddot T,\ddot V,\ddot V')=( \mathbf t , \mathbf v, \mathbf v')\big)
\\&= \#\{( \mathbf v, \mathbf v'):( \mathbf t , \mathbf v, \mathbf v')\in \ddot {\mathbb T}_n\} \cdot \frac{\mathbf G_n( \mathbf t )}{n\sigma^2}
= \frac{X_n( \mathbf t )(X_n( \mathbf t )-1)}{n\sigma^2} \cdot{\mathbf G}_n( \mathbf t ).
\end{split}
\end{equation}

We note in passing that, because of the way they are constructed, the measures $(\ddot{\mathbf G}_n)_{n\ge 1}$ are not consistent, that is, the measure $\ddot{\mathbf G}_n$ is not the restriction of $\ddot{\mathbf G}_{n+1}$.
This implies that the change of measure in Theorem \ref{thm: change of measure} is not a martingale change of measure.
\medskip
\begin{proof}[Proof of Theorem \ref{thm: change of measure}]
	Note that
	\[
	\{(X_m( \mathbf t ))_{0\le m\le n}; {\mathbf G}_n\}  \overset{d}{=} (Z_m)_{0\le m\le n}
	\quad    \mbox{and} \quad \{(X_m( \mathbf t ))_{0\le m\le n};\ddot{\mathbf G}_n\}  \overset{d}{=} (\ddot Z_m)_{0\le m\le n}.
	\]
	According to \eqref{eq:k(k-1)typesizebiasedGWmeasure}, for any bounded Borel function $g$ on $\mathbb N_0^n$, we can verify that
	\begin{equation} \label{eq:proofofchangeofmeasure}
	\begin{split}
	&E [ g ( \ddot Z_1^{(n)}, \dots, \ddot Z_n^{(n)})]
	= \ddot{\mathbf G}_n [g ( X_1(  \mathbf t ), \dots, X_n(  \mathbf t ))]
	\\ &\quad = {\mathbf G}_n \big[ \frac { X_n( \mathbf t ) ( X_n( \mathbf t ) - 1)} {n \sigma^2} g (X_1( \mathbf t ), \dots, X_n( \mathbf t ))\big]
	\\&\quad = \frac { 1} { n \sigma^2} E[ Z_n ( Z_n - 1) g( Z_1, \dots, Z_n)].
	\end{split}
	\end{equation}
	Taking $g\equiv 1$ in equation \eqref{eq:proofofchangeofmeasure}, we get that
	\begin{equation}
	\label{eq: second moment}
	E [Z_n(Z_n-1)]= E [\dot Z_n - 1]= n\sigma^2.
	\end{equation}
\end{proof}

\subsection{Spine decompositions.}
\label{sec:spinesdecomposition}

Using the	notation
introduced in the previous	section, we are now ready to
give a precise meaning to \eqref{eq: Our insight}:
\begin{prop}\label{prop: size-biased add-on of size-biased tree }
	Let $(\dot Z_m)_{0 \leq m \leq n}$ be the population of a size-biased $\mu$-Galton watson tree and $(\ddot Z^{(n)}_m)_{0 \leq m \leq n}$ be the population of a $k(k-1)$-type size-biased $\mu$-Galton-Watson tree with height $n$.
	Suppose that $\mu$ satisfies \eqref{eq:mean} and \eqref{eq:variance}.
	Then
	\[
	E [ e^{- \lambda \ddot Z_n^{(n)}} ]
	 = E [e^{-\lambda \dot Z_n}] E[g(\lambda, \lfloor Un \rfloor )e^{-\lambda \dot Z_{ \lfloor Un \rfloor}}],
	\]
	where $U$ is a uniform random variable on $[0,1]$ independent of $\{\dot Z_m: 0\le m\le n\}$;
	and $g(\lambda, m)$ is a function on $[0,\infty) \times \mathbb N_0$ such that
	$g(\lambda, m) \to 1$, uniformly in $\lambda$ as $m\to \infty$.
\end{prop}

\begin{proof}
	For any particle $u=u_1\dots u_n$, we define
	$	[\emptyset, u]
	:= \{u_1\dots u_j:j=0,\dots, n \}$
	as the \emph{descending family line from $\emptyset$ to $u$}.
	The particles in $\dot T$ can be separated according to their nearest spine ancestor.
	For each $k = 0, \dots, n$, we write
	$\dot A_k
	:= \{u\in\dot T:| [\emptyset, u] \cap \dot V |=k\}.$
	Then
	\begin{equation}
	\label{eq:generationseperation}
	X_n(\dot T)
	=
	\sum_{k=0}^nX_n(\dot A_k).
	\end{equation}
	Notice that the right side of the above equation is a
	sum of independent random variables;
	and from their construction, we see that $X_n(\dot A_k) \overset{d}= Z_{n-k-1}^{(\dot L - 1)}$.
	Here,  $Z^{(\dot L - 1)}_{(-1)}:= 1$ and $(Z^{(\dot L - 1)}_m)_{m\in \mathbb N_0}$ denotes a $\mu$-Galton-Watson process with $Z_0^{(\dot L - 1)}$ distributed according to $\dot L - 1$.
	Taking	Laplace transforms
	on both sides of \eqref{eq:generationseperation} we get
	\begin{equation} \label{eq: laplace transform of one-spine decomposition}
	E [e^{-\lambda \dot Z_n}]
	= \prod_{k = 0}^n E[ e^{-\lambda Z^{(\dot L - 1)}_{n-k-1}} ].
	\end{equation}
	
	Similarly, we consider the $k(k-1)$-type size-biased $\mu$-Galton-Watson tree $(\ddot T,\ddot V,\ddot V')$.
	Write
	\[
	\ddot A^l_k := \{u\in\ddot T: | [\emptyset, u]\cap \ddot V | = k, [\emptyset , u] \cap (\ddot V' \setminus \ddot V ) = \emptyset\}
	\]
	and
	\[
	\ddot A^s_k := \{u\in\ddot T: | [\emptyset, u]\cap \ddot V' | = k, [\emptyset , u] \cap (\ddot V' \setminus \ddot V) \neq \emptyset\}.
	\]
	Then,
	\begin{equation}\label{eq:rawtwospinedecomposition}
	X_n(\ddot T)
	=
	\sum_{k=0}^nX_n(\ddot A^l_k) + \sum_{k=K_n + 1}^n X_n(\ddot A^s_k).
	\end{equation}
	Notice that, conditioning on $K_n = m$ with $m\in\{0,\dots,n-1\}$, the right side of the above equation is a
	sum of independent random variables; and from their construction, we see that
	$X_n(\ddot A^l_k) \overset{d}{=} Z^{(\dot L - 1)}_{n-k-1}$
	for each $k \neq m$;
	$X_n(\ddot A^l_m) \overset{d}{=} Z^{(\ddot L - 2)}_{n-m-1}$;
	and $X_n(\ddot A^s_k) \overset{d}{=} Z^{(\dot L - 1)}_{n-k-1}$ for each $k \geq m+1$.
	Here, $Z^{(\ddot L - 2)}_{(-1)}:= 1$ and $(Z^{(\ddot L - 2)}_k)_{k\in \mathbb N_0}$ is a $\mu$-Galton-Watson process with initial population distributed according to $\ddot L-2$.
	
	Taking Laplace transform on both sides of \eqref{eq:rawtwospinedecomposition}  and using \eqref{eq: laplace transform of one-spine decomposition}, we get
	\[  \begin{split}
	&E [ e^{- \lambda \ddot Z_n^{(n)}} ]
	= \frac{1}{n}\sum_{m=0}^{n-1} \Big( \prod_{k=0,k\neq m}^{n} E[e^{-\lambda Z^{(\dot L - 1)}_{n-k-1}}] \Big) \cdot E [e^{-\lambda Z^{(\ddot L - 2)}_{n-m-1}}] \cdot \Big(\prod_{k= m+1}^n E [e^{-\lambda Z^{(\dot L - 1)}_{n-k-1}}]\Big)
	\\&\quad = E [e^{-\lambda \dot Z_n}]  \frac{1}{n} \sum_{m=0}^{n-1}   \frac{ E [e^{-\lambda Z^{(\ddot L - 2)}_{n-m-1} }] }{ E[e^{-\lambda Z^{(\dot L - 1)}_{n-m-1}  }] } \cdot E[e^{- \lambda \dot Z_{n-m-1}}]
	\\&\quad = E [e^{-\lambda \dot Z_n}]  \frac{1}{n}\sum_{m=0}^{n-1} \frac{ E [e^{-\lambda Z^{(\ddot L - 2)}_{m}}] }{ E[e^{-\lambda Z^{(\dot L - 1)}_{m}}] } \cdot E[e^{- \lambda \dot Z_{m}}]
	= E [e^{-\lambda \dot Z_n}] E[g(\lambda,\lfloor Un \rfloor )e^{-\lambda \dot Z_{ \lfloor Un \rfloor}}],
	\end{split}
	\]
	where
	\[
	P( Z^{(\ddot L - 2)}_m=0 )
	\leq	g(\lambda,m)
	: = \frac{ E [e^{-\lambda Z^{(\ddot L - 2)}_{m}}] }{ E[e^{-\lambda Z^{(\dot L - 1)}_{m}}] }
	\leq P ( Z^{(\dot L - 1)}_m = 0 )^{-1}.
	\]
	Notice that, from the criticality, $P (  Z^{(\ddot L - 2)}_m=0 )$ and $P ( Z^{(\dot L - 1)}_m = 0 )^{-1}$ converge to $1$.
\end{proof}



\section{Proofs}
\label{sec: proofs}
\begin{proof}[Proof of Theorem \ref{thm: Kolmogrov and Yaglom theorem}(\ref{thm:kolmogorov})]
	Denote by $B_n^j$ the event that the Galton-Watson process
	$(Z_n)_{n\geq 0}$
	survives up to generation $n$, and the left-most particle in the $n$-th generation is
	a descendant of the $j$th particle of the first generation.
	Write $q_n = P[Z_n = 0] = f^{(n)}(0)$ and $p_n = 1- q_n$ where $f$ is the probability generating function of the offspring distribution $\mu$.
	Then
	\begin{equation}
	\label{eq: iterate of conditional expectation}
	\begin{split}
	&E[Z_n| Z_n>0]
	= \sum_{k=1}^\infty E[Z_n; Z_1=k| Z_n>0]
	= p_n^{-1} \sum_{k=1}^\infty E[Z_n; Z_1=k;Z_{n}>0]
	\\&\quad = p_n^{-1} \sum_{k=1}^\infty \sum_{j=1}^k E[Z_n; Z_1=k;B_n^j]
	= p_n^{-1} \sum_{k=1}^\infty \sum_{j=1}^k P[Z_1=k;B_n^j] E[Z_n| Z_1=k,B_n^j]
	\\&\quad = p_n^{-1} \sum_{k=1}^\infty \sum_{j=1}^k P[Z_1=k;B_n^j] \Big( E[Z_{n-1}| Z_{n-1}>0] +k-j\Big)
	\\&\quad = E[Z_{n-1}|Z_{n-1}> 0]  + \frac{p_{n-1}}{p_n}\sum_{k=1}^\infty \sum_{j=1}^k \mu(k) q_{n-1}^{j-1}(k-j).
	\end{split}
	\end{equation}
	The criticality implies that $q_n \uparrow 1$ as $n \to \infty$, and that
	\[
	\frac{p_n}{p_{n-1}} = \frac{1- f^{(n)}(0)}{1-f^{(n-1)}(0)} = \frac{1- f(q_{n-1})}{1-q_{n-1}} \xrightarrow[n \to \infty]{} f'(1) = 1.
	\]
	By the monotone convergence theorem,
	\[
	\frac{p_{n-1}}{p_n} \sum_{k=1}^\infty \sum_{j=1}^k \mu(k) q_{n-1}^{j-1} (k-j)
	\xrightarrow[n \to \infty]{} \sum_{k=1}^\infty \sum_{j=1}^k \mu(k) (k-j)
	=  \sum_{k=1}^\infty \mu(k) k(k-1)/2
	= \frac{\sigma^2}{2}.
	\]
	Now combining \eqref{eq: iterate of conditional expectation} with the above, we get
	\[\begin{split}
	\frac{1}{n P(Z_n > 0)}
	&= \frac{1}{n} E[Z_n | Z_n > 0]
	\\&= \frac{1}{n}E[Z_0| Z_0 > 0] + \frac{1}{n} \sum_{m = 1}^n \frac{p_{m-1}}{p_m}\sum_{k=1}^\infty \sum_{j=1}^k \mu(k) q_{m-1}^{j-1}(k-j)
	\\& \xrightarrow[n \to \infty]{} \frac{\sigma^2}{2}.
	\end{split}\]
\end{proof}

In order to compare distributions using their size-biased add-on structures, we need the following lemma:
\begin{lem}\label{lem: compare}
	Let $X_0$ and $X_1$ be two non-negative random variables with the same mean $a = E[X_0] = E[X_1] \in (0,\infty)$.
	Let $F_0$ be defined by $E[e^{-\lambda \dot X_0}] = E[e^{-\lambda X_0}] F_0(\lambda)$,
	where $\dot X_0$ is an $X_0$-transform of $X_0$,
	and $F_1$ be defined by $E[e^{-\lambda \dot X_1}] = E[e^{-\lambda X_1}] F_1(\lambda)$,
	where $\dot X_1$ is an $X_1$-transform of $X_1$.
	Then,
	\[
	\big| E[e^{-\lambda X_0}] - E[e^{-\lambda X_1}] \big|
	\leq a \int_0^\lambda| F_0(s) - F_1(s) |ds, \quad \lambda \geq 0.
	\]
\end{lem}
\begin{proof}
	Since $\dot X_0$ is an $X_0$-transform of $X_0$,
	we have
	\[
	\partial_\lambda ( -\ln E[e^{-\lambda X_0}]) = \frac { E[X_0 e^{-\lambda X_0}]}{ E[e^{-\lambda X_0}] }
	= \frac{a E[e^{-\lambda \dot X_0}]}{E[e^{-\lambda X_0}]}
	= a F_0(\lambda).
	\]
	Similarly, $\partial_\lambda ( -\ln E[e^{-\lambda X_1}]) = a F_1(\lambda)$.
	Therefore,
	since $x - \ln x$ is decreasing on $[0,1]$,
	\[\begin{split}
	\big| E[e^{-\lambda X_0}] - E[e^{-\lambda X_1}] \big|
	&\leq \big| \ln E[e^{-\lambda X_0}] - \ln E[e^{-\lambda X_1}] \big|
	= a\big| \int_0^\lambda F_0(s)ds - \int_0^\lambda F_1(s)ds \big|
	\\& \leq a \int_0^\lambda| F_0(s) - F_1(s) |ds
	\end{split}\]
	as desired.
\end{proof}

We are now ready to prove Lemma \ref{lem: our equation}.
It is elementary to verify that if $Y$ is exponentially distributed, then it satisfies \eqref{eq: x2 type size-biased equation for exponential distribution}.
So we
only need to show
that if $Y$ is a strictly positive random variable with finite second moment, then \eqref{eq: x2 type size-biased equation for exponential distribution} implies that it is exponentially distributed.
The following lemma will be used to prove this.

\begin{lem}\label{lem: zero inequality}
	Suppose that $c>0$ is a constant, and $F$  is a non-negative bounded function on $[0,\infty)$ satisfying that, for any $\lambda\geq 0$,
	\begin{equation}\label{eq: zero inequality}
	F(\lambda)
	\leq
	\frac{1}{c}\int_0^1du
	\int_0^\lambda F(us)ds.
	\end{equation}
	Then $F\equiv 0$.
\end{lem}

\begin{proof}
	By dividing both sides of \eqref{eq: zero inequality} by $\|F\|_\infty$, without loss of any generality, we can assume $F$ is bounded by $1$.
	We prove this lemma by  contradiction.
	Assume that
	\begin{equation}
	\label{eq: contradiction}
	\rho
	:= \inf\{x \geq 0: F(x) \neq 0\}
	< \infty,
	\end{equation}
	with the convention $\inf \emptyset=\infty$.
	Then, for each $\lambda \geq 0$,
	\[
	F(\rho + \lambda)
	= \frac{1}{c} \int_0^1 du \int_0^{\rho +\lambda} F(us) ds
	= \frac{1}{c} \int_0^1 du \int_\rho^{\rho+\lambda} F(us) ds \leq \frac{\lambda}{c}.
	\]
	Using this new upper bound, we have
	\[
	F(\rho + \lambda)
	= \frac{1}{c} \int_0^1 du \int_\rho^{\rho+\lambda} F(us) ds
	\leq \frac{1}{c} \int_0^1 du \int_\rho^{\rho+\lambda} \frac{\lambda}{c}ds
	\leq \frac{\lambda^2}{c^2}.
	\]
	Repeating this process, we have $F(\rho + \lambda) \leq \frac{\lambda^m}{c^m}$ for each $m \in \mathbb N$, which implies that $F = 0$ on $[\rho, \rho + c)$.
	This, however, contradicts \eqref{eq: contradiction}.
\end{proof}

\begin{proof}[Proof of Lemma \ref{lem: our equation}]
	Suppose that $Y$ is a strictly positive random variable with finite second moment, and \eqref{eq: x2 type size-biased equation for exponential distribution} is true.
	Define
	$
	a
	:= E[\dot Y] \in (0,\infty)
	$.
	Consider an exponential random variable $\mathbf e$ with mean $a/2$.
	It is elementary to verify that
	$\mathbf e$ satisfies
	\eqref{eq: x2 type size-biased equation for exponential distribution}, in the sense that
	$
	\ddot {\mathbf e} \overset{d} = \dot {\mathbf e}+U\dot {\mathbf e}',
	$
	where $\dot {\mathbf e}$ and $\dot {\mathbf e}'$ are
	both $\mathbf e$-transforms of $\mathbf e$,
	$\ddot {\mathbf e}$ is an $\mathbf e^2$-transform of $\mathbf e$,
	$U$ is a uniform random variable on $[0,1]$, and $\dot {\mathbf e}$, $\dot {\mathbf e}'$,
	$\ddot {\mathbf e}$ and $U$ are independent.
	Notice that $E[\dot {\mathbf e}] = a$, therefore we can compare the distribution of $\dot Y$
	with that of $\dot {\mathbf e}$
	using Lemma \ref{lem: compare}.
	This gives that
	\[
	\big|E[ e^{-\lambda \dot Y}] - E[ e^{-\lambda \dot {\mathbf e}}] \big|
	\leq  a  \int_0^\lambda \int_0^1 \big| E [e^{-s u \dot Y}] - E [e^{-s u \dot {\mathbf e}}] \big| du ds,
	\quad \lambda \geq 0,
	\]
	which, according to Lemma \ref{lem: zero inequality}, says that
	$\dot Y \overset{d} = \dot {\mathbf e}$.
	Since $Y$ and $\mathbf e$ are strictly positive, according to \eqref{eq: conditional and size-biased transform}, we have
	\[
	E[1-e^{-\lambda Y}]/ E[Y] = E[1-e^{-\lambda \mathbf e}]/ E[\mathbf e], \quad \lambda \geq 0.
	\]
	Letting $\lambda \to \infty$, we get $E[Y] = E[\mathbf e]$. Therefore, $Y \overset{d} = \mathbf e$ as desired.
\end{proof}



\begin{proof}[Proof of Theorem \ref{thm: Kolmogrov and Yaglom theorem}\eqref{thm:yaglom}.]
	Consider an exponential random variable $Y$ with mean $\sigma^2/2$.
	Let $\dot Y$ be a $Y$-transform of $Y$.
	As in Section \ref{sec: Methods},
	we only need to prove that
	$\dot Z_n/n$ converge weakly to $\dot Y$.
	From Proposition \ref{prop: size-biased add-on of size-biased tree }, we know that
	\[
	E [ e^{- \lambda \ddot Z_n^{(n)}} ]
	= E [e^{-\lambda \dot Z_n}] E[g(\lambda, \lfloor Un \rfloor )e^{-\lambda \dot Z_{ \lfloor Un \rfloor}}],
	\]
	where $U$ is a uniform  random variable on $[0,1]$ independent of $\{\dot Z_m: 0\le m\le n\}$;
	and $g(\lambda, m)$ is a function on $[0,\infty) \times \mathbb N_0$ such that
	$g(\lambda, m) \to 1$, uniformly in $\lambda$ as $m\to \infty$.
	After a renormalization, we have that
\[
	E [ e^{- \lambda \frac{\ddot Z_n^{(n)}-1}{n}} ]
	= E [e^{-\lambda \frac{\dot Z_n - 1}{n}}] E\big[g\big(\frac{\lambda}{n}, \lfloor Un \rfloor \big)e^{-\lambda U \frac{\dot Z_{\lfloor Un \rfloor}}{Un} }\big],
	\quad \lambda \geq 0.
\]
	According to Theorem \ref{thm: change of measure}, one can verify that
	$(\ddot Z_n^{(n)} - 1)/n$ is a $(\dot Z_n - 1)/n$ transform of $(\dot Z_n - 1)/n$.
	Therefore, the above equation can be viewed as the size-biased add-on structure for the random variable $(\dot Z_n - 1)/n$.
	It is easy to see that the mean of $\dot Y$ is $\sigma^2$.
	According to \eqref{eq: second moment}, the mean of $(\dot Z_n - 1)/n$ is also $\sigma^2$.
	Then comparing the distribution of $(\dot Z_n - 1)/n$ with that of $\dot Y$, and  using Lemma \ref{lem: compare}, we get that
	\[
	\big| E[e^{-\lambda \frac{\dot Z_n - 1}{n}}] - E[e^{-\lambda \dot Y}]\big|
	\leq \sigma^2 \int_0^\lambda ds \int_0^1 \big| g(\frac{s}{n}, \lfloor un \rfloor ) E[e^{-su \frac { \dot Z_{ \lfloor un \rfloor } } {un} }] - E[e^{- su \dot Y}]\big| du.
	\]
	Taking $n\to \infty$ and using the reverse Fatou's lemma, we arrive at
	\[
	M(\lambda)
	\leq \sigma^2 \int_0^1du \int_0^\lambda M(us)ds,
	\quad \lambda\geq 0,
	\]
	where
	$M(\lambda) := \limsup_{n\to \infty} | E[ e^{- \lambda \frac{\dot Z_n }{n}}] - E[e^{-\lambda \dot Y}]|$.
	Thus by Lemma \ref{lem: zero inequality}, we have $M \equiv 0$, which says that $\dot Z_n/n$ converges weakly to $\dot Y$.
\end{proof}

% vim:ts=4:sw=4

	% Copyright (c) 2014,2016,2018 Casper Ti. Vector
% Public domain.

\chapter{Spine decompositions of critical superprocesses: Yaglom type result}

\section{Introduction}

	It is well known that for a critical Galton-Watson process $\{(Z_n)_{n\in \mathbb N};P\}$, we have
\begin{equation}
\label{eq:Mot-1}
	nP(Z_n>0)
	\xrightarrow[n\to\infty]{} \frac{2}{\sigma^2}
\end{equation}
	and
\begin{equation}
\label{eq:Mot-2}
	\big\{\frac{Z_n}{n}; P(\cdot|Z_n>0)\big\}
	\xrightarrow[n\to\infty]{law} \frac{\sigma^2}{2} \mathbf e,
\end{equation}
	where $\sigma^2$ is the variance of the offspring distribution and $\mathbf e$ is an exponential random variable with mean 1.
	The result \eqref{eq:Mot-1} was first proved by Kolmogorov in \cite{Kolmogorov1938Zur-losung} under a third moment condition, and the result \eqref{eq:Mot-2} is due to Yaglom \cite{Yaglom1947Certain}.
	For further references to these results, see \cite{Harris2002The-theory, KestenNeySpitzer1966The-Galton-Watson}.
	Ever since these pioneering works of Kolmogorov and Yaglom, lots of analogous results were obtained for more general critical branching processes.
	For continuous time critical branching processes, see \cite{AthreyaNey1972Branching}; for discrete time multitype critical branching processes, see \cite{AthreyaNey1972Branching, JoffeSpitzer1967On-multitype}; for continuous time multitype critical branching processes, see \cite{AthreyaNey1974Functionals}; and for critical branching Markov processes, see \cite{AsmussenHering1983Branching}.
	We will call results like \eqref{eq:Mot-1} Kolmogorov type results and results like \eqref{eq:Mot-2} Yaglom type results.
\par
	Similar results have also been obtained for some superprocesses.
	Evans and Perkins \cite{EvansPerkins1990Measure-valued} obtained both Kolmogorov type and Yaglom type results for critical superprocesses when the branching mechanism is $(x,z) \mapsto z^2$ and the spatial motion satisfies some ergodicity conditions.
	Recently, Ren, Song and Zhang \cite{RenSongZhang2015Limit} obtained similar limit results for a class of critical superprocesses with general branching mechanisms and general spatial motions:
	Let $X=\{(X_t)_{t\geq 0};\mathbf P_\mu\}$ be a critical superprocess, in a locally compact separable metric space $E$, starting from a measure $\mu$, with its spatial motion being a Hunt process, and its branching mechanism being
\[
	(x,z) \mapsto -\beta(x)z + \alpha(x)z^2 + \int_{(0,\infty)} (e^{-zy}-1+zy)\pi(x,dy),
\]
	where $\beta\in b\mathscr B_E,\alpha\in bp\mathscr B_E$ and $\pi(x,dy)$ is a kernel from $E$ to $(0,\infty)$ satisfying that $\sup_{x\in E}\int_{(0,\infty)}(y\wedge y^2)\pi(x,dy)<\infty$.
	Under the condition that the transition semigroup of the spatial motion is intrinsically ultracontractive with respect to some reference measure $m$ on $E$, and some other mild assumptions, it was proved in \cite{RenSongZhang2015Limit} that
\begin{equation}
\label{eq:Mot-3}
	t\mathbf P_\mu(X_t\neq \mathbf 0)
	\xrightarrow[t\to\infty]{} c^{-1} \langle \phi,\mu \rangle,
\end{equation}
	and for a large class of functions $f$ on $E$,
\begin{equation}
\label{eq:Mot-4}
	\big\{\frac{X_t(f)}{t};\mathbf P_\mu(\cdot|X_t \neq \mathbf 0)\big\}
\xrightarrow[t\to\infty]{law} c \langle \phi^*, f\rangle_m \mathbf e.
\end{equation}
	Here, the constant $c>0$ is independent of the choice of $\mu$ and $f$;
	$\mathbf e$ is an exponential  random variable with mean 1;
	and $\phi$ (respectively, $\phi^*$) is the principal eigenfunction of (respectively, the dual of) the mean semigroup of $X$.
\par
	The proofs of the limit results in the works mentioned above are all analytic in nature and thus not very transparent.
	More intuitive probabilistic proofs would be very helpful.
	This was first accomplished for critical Galton-Watson processes, see \cite{Geiger1999Elementary, LyonsPemantlePeres1995Conceptual} for probabilistic proofs of \eqref{eq:Mot-1}, and \cite{Geiger2000A-new-proof, LyonsPemantlePeres1995Conceptual, RenSongSun2018A-2-spine} for probabilistic proofs of \eqref{eq:Mot-2}.
	For more general models, Vatutin and Dyakonova \cite{VatutinDyakonova2001The-survival} gave a probabilistic proof of a Kolmogrov type result for multitype critical branching processes.
	Recently, Powell \cite{Powell2015An-invariance} gave probabilistic proofs of both Kolmogorov type and Yaglom type results for a class of critical branching diffusions.
As far as we know, there is no probabilistic proof of Yaglom type result for multitype critical branching processes, and there are no probabilistic proofs of results \eqref{eq:Mot-3} and \eqref{eq:Mot-4} for critical superprocesses yet.
\par
	The method initiated in \cite{LyonsPemantlePeres1995Conceptual} is now known as the spine decomposition.
	Roughly speaking, the spine decomposition theorem says that the size-biased transform of the branching process can be interpreted as an immigration branching process along with an immortal particle.
	This spine approach is generic in the sense that it can be adapted to a variety of general branching processes and is powerful in studying limit behaviors due to its relation with the size-biased transforms.
\par
	In this chapter, by the \emph{size-biased transform} we mean the following:
	Let $X$ be a random variable and $g$ be a Borel function with $\mathbf P(g(X)\geq 0)=1$ and $\mathbf E[g(X)] \in (0,\infty)$.
	We say that a random variable $\dot X$ is the \emph{$g(X)$-size-biased transform (or simply $g(X)$-transform) of $X$} if
\[
	\mathbf E [f (\dot X)]
	= \frac {\mathbf E[ g(X) f(X)]} {\mathbf E[ g(X)]},
\]
	for every positive Borel function $f$.
	More generally, for a non-negative measurable function $g$ on a measure space $(D,\mathscr F_D,\mathbf D)$ with $\mathbf D(g)\in (0,\infty)$, we define the \emph{$g$-transform of the measure $\mathbf D$} as the probability measure $\mathbf D^g$ on $(D,\mathscr F_D)$ such that
\[
	d\mathbf D^g
	:= \frac{g}{\mathbf D(g)} d\mathbf D.
\]
	It is clear that a random variable $\dot X$ is the $g(X)$-transform of a random variable $X$ if and only if the law of $\dot X$ is the $g$-transform of the law of $X$.
For a process $(X_t)_{t\in \Gamma}$ with $\Gamma = \mathbb R^+$ or $\mathbb N$, and a nonnegative random variable $G$, defined in a probability space $(\Omega,\mathscr F,P)$, satisfying $ P[G] \in (0,\infty)$,
	we say that a process $\{(\dot X_t)_{t\in \Gamma}; \dot P\}$ is the \emph{$G$-transform of  $(X_t)_{t\in \Gamma}$} if
\[
	\{(\dot X_t)_{t\in\Gamma}; \dot P\}
	\overset{f.d.d.}{=} \{(X_t)_{t\in \Gamma}; P^G\},
\]
	where $P^G$ is the $G$-transform of the probability measure $P$.
	Here and in the rest of the paper, $\overset{f.d.d.}{=}$ means the two processes have the same finite-dimensional distributions.
	\par
	Using the spine decomposition theorem for the Galton-Watson process $(Z_n)_{n\geq 0}$, Lyons, Pemantle and Peres \cite{LyonsPemantlePeres1995Conceptual} investigated the $Z_n$-transform of the process $(Z_k)_{0\leq k\leq n}$, which is denoted by $(\dot Z_k)_{0\leq k\leq n}$.
	Their key observation in the critical case is that $U\cdot \dot Z_n$ is distributed approximately like $Z_n$ conditioned on $\{Z_n > 0\}$ where $U$ is an independent uniform random variable on $[0,1]$.
	If one denotes by $X$ the weak limit of $\frac{Z_n}{n}$ conditioned on $\{Z_n > 0\}$, and by $\dot X$  the weak limit of $\frac{\dot Z_n}{n}$, then \cite{LyonsPemantlePeres1995Conceptual} proved that $\dot X$ is the $X$-transform of the positive random variable $X$ and
\[
	X
	\overset{law}{=} U \cdot \dot X,
\]
	which implies that $X$ is an exponential random variable.
\par
	Very recently, we developed a 2-spine decomposition technique in \cite{RenSongSun2018A-2-spine} for critical Galton-Watson processes
	and used it to give a new probabilistic proof of Yaglom's result \eqref{eq:Mot-2}.
	One of the facts we used in \cite{RenSongSun2018A-2-spine} is that, if $X$ is a strictly positive random variable with finite second moment,
	then $X$ is an exponential random variable if and only if
\begin{equation}\label{eq:intro-1}
	\ddot X
	\overset{law}{=} \dot X + U \cdot \dot X'
\end{equation}
	where $\dot X$ and $\dot X'$ are independent $X$-transforms of $X$;
	$\ddot X$ is the $X^2$-transform of $X$;
	and $U$ is again an independent uniform random variable on $[0,1]$.
	We then proved in \cite{RenSongSun2018A-2-spine} that the $Z_n(Z_n-1)$-transform of the critical Galton-Watson process $(Z_k)_{0\leq k\leq n}$, which is denoted as $(\ddot Z_k^{(n)})_{0\leq k\leq n}$, can be interpreted as an immigration branching process along a 2-spine skeleton.
	One of those two spines is longer than the other.
	The spirit of our proof in \cite{RenSongSun2018A-2-spine} is to show that the immigration along the longer spine at generation $n$ is distributed approximately like $\dot Z_n$, while the immigration along the shorter spine at generation $n$ is distributed approximately like $\dot Z'_{[U\cdot n]}$.
	Here $\dot Z_n$ and $\dot Z_n'$ are independent $Z_n$-transforms of $Z_n$.
	Roughly speaking, we have
\[
	\ddot Z_n^{(n)}
	\overset{law}{\approx} \dot Z_n + \dot Z'_{[U\cdot n]},
\]
	and therefore, if $X$ is the weak limit of $\frac{Z_n}{n}$ conditioned on $\{Z_n>0\}$, then $X$ is a positive random variable satisfying \eqref{eq:intro-1}.
	Our main motivation for \cite{RenSongSun2018A-2-spine} was to develop a method that can be adapted to give probabilistic proofs of Kolmogorov type and Yaglom type results for general critical branching processes.
\par
	The spine decomposition approach is also very useful in studying limit behaviors of supercritical superprocesses.
	In this case, the spine is the trajectory of an immortal moving particle and the spine decomposition theorem says that, after a martingale change of measure, the transformed superprocess can be decomposed in law as an immigration process along this spine.
	A closely related decomposition, the skeleton decomposition, says that the superprocess itself can be decomposed in law as an immigration process along a branching particle system.
	For the spine method on superprocesses, see \cite{EckhoffKyprianouWinkel2015Spines, EnglanderKyprianou2004Local,  LiuRenSong2009Llog}.
	For the skeleton methods and its applications under a variety of names, see \cite{BerestyckiKyprianouMurillo-Salas2011The-prolific,  BertoinFontbonaMartinez2008On-prolific, DuquesneWinkel2007Growth, EckhoffKyprianouWinkel2015Spines, EnglanderPinsky1999On-the-construction, EvansOConnell1994Weighted,  KyprianouPerezRen2014The-backbone, KyprianouRen2012Backbone,  Milos2012Spatial, RenSongZhang2015Central}.
	However, up to now, there is no known skeleton decomposition for critical superprocesses conditioned on survival up to time $t$ yet.
\par
	In both the spine decomposition and the skeleton decomposition, the Kuznetsov measures (also known as the excursion measures or $\mathbb N$-measures) are often used to describe the infinitesimal immigrations along the spine or the skeleton.
	Such measures are $\sigma$-finite measures
on $\mathcal W$,
	the canonical space of all the finite-measure-valued right-continuous trajectories
	on $[0, \infty)$ and having the null measure as a trap on $(0, \infty)$.
	We refer our readers to \cite{DynkinKuznetsov2004mathbb} and \cite[Section 8.4]{Li2011Measure-valued} for detailed discussions of the definition and the existence of such measures.
\par
	The Kuznetsov measures can also be used to give
	a construction of superprocesses.
	Denote by $\mathbb N_\mu$ the Kuznetsov measure of a superprocess $\{X;\mathbf P_\mu\}$.
	If $\mathcal N$ is a Poisson random measure on $\mathcal W$ with mean measure $\mathbb N_\mu$, then according to \cite[Theorem 8.24]{Li2011Measure-valued}, under some conditions, the process defined by $\widetilde X_0:=\mu$ and
\begin{equation}\label{eq:Mot-5}
	\widetilde X_t(\cdot)
	:= \int_{\mathcal W} w_t(\cdot)\mathcal N(dw),
	\quad t>0,
\end{equation}
	is a realization of $\{X;\mathbf P_{\mu}\}$.
\par
	In this chapter, we first establish a spine decomposition theorem
	(Theorem \ref{prop:sizBiasNMeas}) and a 2-spine decomposition theorem
	(Theorem \ref{prop:2-spine-decomposition}) for critical superprocesses.
	Then we use these tools to give probabilistic proofs of a Kolmogorov type result (Theorem \ref{thm:Kolmogorov-type-of-theorem}) and a Yaglom type result (Theorem \ref{thm:Yaglom-type-theorem}) for a class of critical superprocesses.
	The key to our probabilistic insight is that the spine decomposition theorem for superprocesses and the Poissonian representation \eqref{eq:Mot-5} are closely related, a fact which seems has not been fully exploited before in the literature.
	In fact, we will develop a decomposition theorem for size-biased Poisson random measures which we think should be of independent interest:
\begin{itemize}
\item
	\emph{Let $\{\mathcal N;P\}$ be a Poisson random measure on a measurable space $(S, \mathscr S)$ with mean measure $N$.
	Let $F$ be a non-negative measurable function on $(S, \mathscr S)$ with $N(F)\in (0, \infty)$.
	Then the $N(F)$-transform of $\{\mathcal N;P\}$ can be decomposed in law as a copy of $\{\mathcal N;P\}$ plus an independent random atomic measure $\delta_\vartheta$ with $\vartheta$ distributed according to the $F$-transform of the measure $N$.
This will be made precise in Theorem \ref{prop:sizBaisPoissRandMeas}.}
\end{itemize}
\par
	With this theorem and the Poissonian representation \eqref{eq:Mot-5} of superprocesses, we prove a theorem which will serve as a general framework for the study of a large class of size-biased transforms of superprocesses:
\begin{itemize}
\item
	\emph{Let $\{X;\mathbf P_\mu\}$ be the superprocess defined in \eqref{eq:Mot-5}.
	Let $F$ be a non-negative measurable function on $\mathcal W$ with $\mathbb N_\mu(F)\in(0,\infty)$.
	Then the $\mathcal N(F)$-transform of the superprocess $(X_t)_{t\geq 0}$ can be decomposed in law as a copy of $(X_t)_{t\geq 0}$ plus an independent measure-valued process $(Y_t)_{t\geq 0}$, which is distributed according to the $F$-transform of the Kuznetsov measure $\mathbb N_\mu$.
	This will be made precise in Theorem \ref{prop:sizBiasDecSupProc}.}
\end{itemize}
\par
	Both the spine decomposition and the 2-spine decomposition for superprocesses are special forms of this general theorem.
	Recall that  $\phi$ is the principal eigenfunction of the mean semigroup of our critical superprocess $X$.
	By taking $F(w) := w_T(g)$ where $T > 0$ and $g$ is a nonnegative function on $E$ such that $m(g) > 0$ and $\phi^{-1}g$ is bounded from above, we have the following spine decomposition:
\begin{itemize}
\item
	\emph{Let $(Y_t)_{t\geq 0}$ be a measure-valued process with law $\mathbb N^{w_T(g)}$.
	Then $(Y_t)_{0\leq t\leq T}$ can be interpreted as an immigration process along the trajectory
	of a moving particle (referred as the spine) up to time $T$.
	This will be made precise in Theorem \ref{prop:sizBiasNMeas}.}
\end{itemize}
	By taking $F(w) := w_T(\phi)^2$ where $T>0$, we have the following 2-spine decomposition:
\begin{itemize}
\item
	\emph{Let $(Y_t)_{t\geq 0}$ be a measure-valued process with law $\mathbb N^{w_T(\phi)^2}$.
	Then $(Y_t)_{0\leq t\leq T}$ can be interpreted as an immigration process along the trajectories of two moving particles (referred as the 2-spine skeleton) up to time $T$.
	These two particles will move together at first and then split at a random time.
	This will be made precise in Theorem \ref{prop:2-spine-decomposition}.}
\end{itemize}
\par
	To discuss the spine or the 2-spine decomposition of our superprocesses, only some mild assumptions (Assumptions \ref{asp:1}, \ref{asp:2} and \ref{asp:3}) are needed.
	Assumption \ref{asp:1} guarantees the existence of the Kuznetsov measures;
    Assumption \ref{asp:2} guarantees the existence of the principal eigenfunction $\phi$ of the mean semigroup of the superprocess;
	Assumption \ref{asp:3} is a moment condition which guarantees
	that $X_t(\phi)$ has finite 2nd moment.
\par
	In Section \ref{sec:asymptotic} we will state and prove our Kolmogorov type and Yaglom type results for critical superprocesses.
	In order to develop those limit theorems, some further, stronger, assumptions are needed.
	Assumption \ref{asp:1'} is a condition about the extinction probability of superprocesses.
	Assumption \ref{asp:2'} says that the transition semigroup of the spatial motion is intrinsically ultracontractive.
	Those further assumptions are discussed in Section \ref{sec:further_assumptions}.
	The assumptions made in this chapter are slightly weaker than the assumptions made in \cite{RenSongZhang2015Limit}.
	So all the examples in \cite{RenSongZhang2015Limit} satisfy our conditions.

\section{Perliminaries}

\subsection{Superprocesses}
	Let $E$ be a locally compact separable metric space, and $\mathcal M_f$ be the collection of all the finite Borel measures on $E$ equipped with topology of weak convergence.
	Let $\partial$ be a point not contained in $E$.
	$\partial$ will be interpreted as a cemetery point.
	We will use $E_\partial$ to denote $E\cup\{\partial\}$.
	Every function $f$ on $E$ is extended on $E_\partial$ by setting $f(\partial) = 0$.
	For any real function $\Gamma$ on $E \times \mathbb R$ and any real function $f$ on $E$, we write $ \Gamma (f)(x) := \Gamma(x,f(x))$ for each $ x\in E$.
	We will use $b\mathscr B_E$ and $p\mathscr B_E$ to denote the collection of all the bounded Borel functions and positive Borel functions on $E$ respectively.
	We write $bp\mathscr B_E$ for $b\mathscr B_E \cap p\mathscr B_E$.
	For any functions $f,g$ and measure $\mu$ on $E$, we write $\|f\|_\infty := \sup_{x\in E} |f(x)|$, $\mu(f) := \int_E f d\mu$, $\langle \mu,f \rangle := \int_E f d\mu$ and $\langle f,g \rangle_\mu := \int_E fg d\mu$ as long as they have meanings.
	We use $\mathbf 0$ to denote the null measure and use $f\equiv 0$ to mean that $f$ is the zero function.
	If $g(t,x)$ is a function on $[0,\infty)\times E$, we say $g$ is locally bounded if $\sup_{t\in [0,T],x\in E} |g(t,x)|<\infty$ for every $T\geq 0$.
\par
	Let the \emph{spatial motion} $\{(\xi_t)_{t\geq 0}; (\mathbb P_x)_{x\in E}\}$ be an $E$-valued Hunt process with $\zeta :=\inf\{t>0:\xi_t=\partial\}$ being its lifetime and $(P_t)_{t\geq 0}$ being its transition semigroup.
	Let the \emph{branching mechanism} $\Psi$ be a function on $E\times [0,\infty)$ defined by
\[
	\Psi(x,z)
	:=-\beta(x)z+\alpha(x)z^2+\int_{(0,\infty)}(e^{-zy}-1+zy)\pi(x,dy),
	\quad x\in E,z\geq 0,
\]
	with $\beta\in b\mathscr B_E,\alpha\in bp\mathscr B_E$ and $\pi(x,dy)$ being a kernel from $E$ to $(0,\infty)$ satisfying that $\sup_{x\in E} \int_{(0,\infty)} (y\wedge y^2) \pi(x,dy) < \infty$.
\par
	Let $\{(X_t)_{t\geq 0};(\mathbf P_\mu)_{\mu\in \mathcal M_f}\}$ be a \emph{superprocess with spatial motion $\{\xi;\mathbb P\}$ and branching mechanism $\Psi$.}
	That is to say, $\{X;\mathbf P\}$ is an $\mathcal M_f$-valued Markov process such that $\mathbf P_\mu[e^{-X_t(f)}] = e^{-\mu(V_tf)}$ for each $t \geq 0,\mu \in \mathcal M_f$ and
	$f \in p\mathscr B_E$.
	Here, for any $f\in bp\mathscr B_E$, the function $(t,x) \mapsto V_tf(x)$ on $[0,\infty) \times E$ is the unique locally bounded positive solution to the equation
\begin{equation}\label{eq:FKPP_in_definition}
	V_t f(x) + \mathbb P_x \big[ \int_0^t \Psi (V_{t-s} f)(\xi_s) ds \big]
	=\mathbb P_x[f(\xi_t)], \quad t \geq 0, x\in E.
\end{equation}
	We refer our readers to \cite{Dawson1993Measure-valued, Dynkin1993Superprocesses} and \cite[Section 2.3]{Li2011Measure-valued} for detailed discussions about the definition and the existence of such processes.
	Notice that we always have $\mathbf P_{\mathbf 0}(X_t = \mathbf 0) = 1$ for each $t\geq 0$.
	The process $\{(X_t)_{t\geq 0};(\mathbf P_\mu)_{\mu\in \mathcal M_f}\}$ will sometimes be referred to as a $(\xi, \Psi)$-superprocess.
\par
	According to \cite[Theorem 2.21]{Li2011Measure-valued}, the family $(V_t)_{t\geq 0}$ of operators on $p\mathscr B_E$ satisfies the semigroup property in the sense that, for any $f\in p\mathscr B_E, t, s \geq 0$ and $x \in E$, $V_t V_sf(x) = V_{t+s} f(x)$.
	$(V_t)_{t \geq 0}$ is referred to as the \emph{cumulant semigroup} of $\{X;\mathbf P\}$.
	Let $(S_t)_{t \geq 0}$ be a Feynman-Kac semigroup of $\{\xi; \mathbb P\}$ defined by
\[
	S_t f(x)
	:= \mathbb P_x[e^{\int_0^t \beta(\xi_s) ds} f(\xi_t)],
	\quad x \in E,t \geq 0,f \in b\mathscr B_E.
\]
	Then $(S_t)_{t \geq 0}$ is the \emph{mean semigroup} of the superprocess $\{X; \mathbf P\}$.
	It is well known, see \cite[Proposition 2.27]{Li2011Measure-valued} for example, that
\[
	\mathbf P_\mu[X_t(f)]
	= \mu(S_t f),
	\quad \mu \in \mathcal M_f, t \geq 0,f \in b\mathscr B_E.
\]
	It is also known, see \cite[Theorem 2.23]{Li2011Measure-valued} for example, that \eqref{eq:FKPP_in_definition} can be rewritten as
\begin{equation}\label{eq:mean-fkpp}
	V_t f(x) + \int_0^t S_{t-s} \Psi_0(V_s f)(x) ds
	= S_t f(x),
	\quad f \in bp\mathscr B_E, t \geq 0,x \in E,
\end{equation}
	where
\[
	\Psi_0(x,z)
	:= \alpha(x) z^2 + \int_{(0,\infty)} (e^{-z y} - 1 + z y) \pi(x,dy),
	\quad x \in E,z \geq 0.
\]
	We exclude the degenerate case that $\Psi_0\equiv 0$.

\subsection{Kuznetsov measures}
	In this chapter, we make the following assumption which not only allows us to study the extinction probability via the cumulant semigroup $(V_t)_{t\geq 0}$ but also guarantees the existence of the Kuznetsov measures $(\mathbb N_x)_{x\in E}$.
\begin{asp}\label{asp:1}
	$\mathbf P_{\delta_x}(X_t = \mathbf 0) > 0$ for each $x \in E$ and $t>0$.
\end{asp}
	Under Assumption \ref{asp:1}, we define,
\begin{equation}\label{eq:V_infty}\begin{split}
	v_t(x)
	&:= \lim_{\theta \to \infty} V_t(\theta \mathbf 1_E)(x)
	= - \lim_{\theta \to \infty} \log \mathbf P_{\delta_x}[e^{-X_t(\theta \mathbf 1_E)}]\\
	&= - \log \mathbf P_{\delta_x}(X_t = \mathbf 0)
	< \infty,
	\quad x \in E, t> 0.
\end{split}\end{equation}
	For convention, we write $-\log 0 = +\infty$.
	By monotonicity,
\begin{equation}\label{eq:linearity}\begin{split}
	-\log \mathbf P_\mu(X_t = \mathbf 0)
	&= -\lim_{\theta \to \infty} \log \mathbf P_{\mu}[e^{-X_t(\theta 1_E)}]\\
	& = \lim_{\theta\to \infty} \langle \mu, V_t(\theta\mathbf 1_E) \rangle
	= \langle \mu, v_t \rangle,
	\quad \mu \in \mathcal M_f, t > 0.
\end{split}\end{equation}
	It is also clear that
\[\begin{split}
	V_t v_s(x)
	&= -\log \mathbf P_{\delta_x}[e^{-\langle X_t,\lim_{\theta\to\infty } V_s(\theta \mathbf 1_E)\rangle}]
	= -\lim_{\theta \to \infty} \log \mathbf P_{\delta_x}[e^{-\langle X_t, V_s(\theta \mathbf 1_E)\rangle}]\\
	&= - \lim_{\theta\to\infty} V_t V_s(\theta \mathbf 1_E)(x)
	= v_{t+s}(x),
	\quad s,t>0, x\in E.
\end{split}\]
\par
	Denote by $(\mathbb N_x)_{x\in E}$
	the Kuznetsov measures of $\{X;\mathbf P\}$ which are $\sigma$-finite measures on $\mathcal W$, the space of all the $\mathcal M_f$-valued right-continuous trajectories on
	$[0,\infty)$ and having the null measure as a trap on $(0,\infty)$,
	equipped with $\sigma$-algebra on $\mathcal W$ generated by the coordinate process $(w_t)_{t\geq 0}$.
	We refer our readers to \cite{DynkinKuznetsov2004mathbb} and \cite[Section 8.4]{Li2011Measure-valued} for detailed discussions of the definition and existence of such measures.

	According to \cite[Theorem 8.22]{Li2011Measure-valued}, $\mathbb N_x$-a.e., we have $w_0 = \mathbf 0$.
	For any $\mu \in \mathcal M_f$, define
\[
	\mathbb N_\mu(dw)
	:=\int_E \mu(dx)\mathbb N_x(dw),
	\quad w\in \mathcal W.
\]	
\begin{lem}[{\cite[Theorem 8.24]{Li2011Measure-valued}}]\label{lem:lislem}
	Under Assumption \ref{asp:1}, for any $\mu\in \mathcal M_f$,
	if $\{\mathcal N; \mathbf P_\mu\}$ is a Poisson random measure with mean measure $\mathbb N_\mu$,
	then the process defined by
\[
	\widetilde X_0 := \mu;
	\quad \widetilde X_t(\cdot)
	:=\mathcal N[w_t(\cdot)],
	\quad t>0,
\]
	is a realization of the superprocess $\{X; \mathbf P_{\mu}\}$.
\end{lem}
	To simplify our discussion, we will always assume $X$ is given by the realization $\widetilde X$.
\par
	For any $T>0$, we write $(K,f)\in \mathcal K_T$ if $f: (s,x) \mapsto f_s(x)$ is a bounded non-negative Borel function on $(0,T] \times E$, and $K$ is an atomic measure on $(0,T]$ with finitely many atoms.
	For any $(K,f)\in\mathcal K_T$ and any $\mathcal M_f$-valued  process $(Y_t)_{t> 0}$, define the random variable
\[	
	K_{(s, T]}^f(Y)
	:= \int_{(s,T]} Y_{r-s}(f_r) K(dr),
	\quad s\in [0,T].
\]	
	It is clear that two $\mathcal M_f$-valued processes $(Y_t)_{t>0}$ and $(X_t)_{t>0}$ have same finite-dimensional distributions if and only if
\[
	\mathbf E[e^{-K_{(0,T]}^f(X)}]
	=\mathbf E[e^{-K^f_{(0,T]}(Y)}],
	\quad (K,f)\in \mathcal K_T, T>0.
\]
\begin{lem}[{\cite[Theorem 5.15 and Theorem 8.23]{Li2011Measure-valued}}]\label{lem:EquatDescNmeas}
	Under Assumption \ref{asp:1}, for any fixed $T> 0$ and $(K,f) \in \mathcal K_T$, we have
\[
	\mathbb N_\mu \big[ 1 - e^{-K_{(s, T]}^f(w)} \big]
	= \langle \mu, u_s \rangle
	= -\log \mathbf P_{\mu} \big[ e^{-K_{(s, T]}^f(X)} \big],
	\quad s\in [0,T], \mu \in \mathcal M_f,
\]
	where the function $u: (s,x) \mapsto u_s(x)$ on $[0,T] \times E$ is the unique bounded positive solution to the following integral equation:
\[
	u_s(x)
    =\mathbb P_x \Big[\int_{(s,T]} f_r(\xi_{r-s}) K(dr) - \int_s^T \Psi(u_r)(\xi_{r-s}) dr \Big],
    \quad s \in [0,T], x \in E.
\]
\end{lem}

\subsection{Poisson random measure and its size-biased transform}
	In this subsection, we digress briefly from superprocesses and discuss the size-biased transforms of Poisson random measures.
	Let $(S, \mathscr S)$ be a measurable space with a $\sigma$-finite measure $N$.
	Let $\{\mathcal N; P\}$ be a Poisson random measure on $(S, \mathscr S)$ with mean measure $N$.
	Campbell's theorem, see \cite[Proof of Theorem 2.7]{Kyprianou2014Fluctuations} for example, characterizes the law of $\{ \mathcal N; P \}$ by its Laplace functionals:
\[
	P [e^{-\mathcal N(g)}]
	= e^{-N(1 - e^{-g})},
	\quad g\in p\mathscr S.
\]
	According to \cite[Theorem 2.7]{Kyprianou2014Fluctuations}, we also have that $P [\mathcal N(g)] = N(g)$ for each $g\in \mathscr S$ with $N(|g|) < \infty.$
	By monotonicity, one can verify that
\[
	P [\mathcal N(g)] = N(g),
	\quad g\in p\mathscr S.
\]
\begin{lem}\label{lem:size-biased-lemma}
	If $g\in L^1(N)$ and $ f\in p\mathscr S$, then $\mathcal N(g) e^{-\mathcal N(f)}$ is integrable and
\begin{equation}\label{eq:size-biased-equation}
	P[\mathcal N(g) e^{-\mathcal N(f)}]
	= P[e^{-\mathcal N(f)}] N[g e^{-f}].
\end{equation}
	Furthermore, \eqref{eq:size-biased-equation} is true for each $g,f\in p\mathscr S$ if we allow extended values.	
\end{lem}
\begin{proof}
	Since $N$ is a $\sigma$-finite measure on $(S, \mathscr S)$, there exists a strictly positive measurable function $h$ on $S$ such that $N(h)<\infty$.
	According to \cite[Theorem 2.7.]{Kyprianou2014Fluctuations}, $\mathcal N(h)$ has finite mean.
	For any $g\in bp\mathscr S^h :=\{g\in p\mathscr S: \|h^{-1}g\|_\infty <\infty\}$ and $f\in p\mathscr S$, it is clear that $\mathcal N(g)$ and $\mathcal N(g)e^{-\mathcal N(f)}$ are integrable.
	Therefore, by the dominated convergence theorem, we deduce that
\[\begin{split}
	P[\mathcal N(g) e^{-\mathcal N(f)}]
	&=P[-\partial_\theta|_{\theta=0} e^{-\mathcal N(f+\theta g)}]
	= -\partial_\theta|_{\theta=0} P[e^{-\mathcal N(f+\theta g)}]\\
	&= -\partial_\theta|_{\theta=0} e^{-N(1-e^{-(f+\theta g)})}
	= e^{- N(1-e^{-f})} \partial_\theta|_{\theta=0}N(1-e^{-(f+\theta g)})\\
	&= P[e^{-\mathcal N(f)}]N[ge^{-f}].
\end{split}\]
	For any $g\in p\mathscr S$ and $s\in S$, define $g^{(n)}(s) := h(s) \min\{h(s)^{-1}g(s),n\}$.
	Then $(g^{(n)})_{n\in \mathbb N}$ is a $bp\mathscr S^h$-sequence which increasingly
	converges to $g$ pointwisely.
	Note that \eqref{eq:size-biased-equation} is true for each $g^{(n)}$ and $f$.
	Letting $n\to\infty$, by monotonicity, we see that if we allow extended values, then \eqref{eq:size-biased-equation} is true for each $g,f\in p\mathscr S$.
	In the case when $g\in L^1(N)$, we simply consider its positive and negative parts.
\end{proof}
\begin{thm}\label{prop:sizBaisPoissRandMeas}
	Let $g \in p \mathscr S$ with $N(g)\in (0,\infty).$
	Denote by $N^g$ and $P^{\mathcal N(g)}$ the $g$-transform of $N$ and the $\mathcal N(g)$-transform of $P$, respectively.
	Let $\{\vartheta;Q\}$ be an $S$-valued random element  with law $N^g$, defined in a different probability space.
	Then, we have
\[
	\{\mathcal N;P^{\mathcal N(g)}\}
	\overset{law}{=} \{\mathcal N + \delta_\vartheta;P\otimes Q\}.
\]
\end{thm}
\begin{proof}
	By Lemma \ref{lem:size-biased-lemma}, it is easy to see that, for any $f\in p\mathscr S$,
\[\begin{split}
	P^{\mathcal N(g)}[e^{-\mathcal N(f)}]
	&= N(g)^{-1} P[\mathcal N(g) e^{-\mathcal N(f)}]
	= N(g)^{-1} P[e^{-\mathcal N(f)}] N[ge^{-f}]\\
	&=  P[e^{-\mathcal N(f)}] N^g[e^{-f}]
	= (P\otimes Q)[e^{-\mathcal N(f) - f(\vartheta)}]
	= (P\otimes Q)[e^{-(\mathcal N + \delta_\vartheta)(f)}].
\end{split}\]
	which completes the proof.
\end{proof}
\begin{lem}\label{lem:covPoissRandMeas}
	For all $g, f \in L^1(N) \cap L^2(N)$, $\mathcal N(g) \mathcal N(f)$ is integrable and
\begin{equation}\label{eq:2rdMomPoissRandMeas}
	P [\mathcal N(g) \mathcal N(f)]
	= N(g) N(f) + N(g f).
\end{equation}
	Furthermore, \eqref{eq:2rdMomPoissRandMeas} is true for all $g,f\in p\mathscr S$ if we allow extended values.
\end{lem}
\begin{proof}
	Since $N$ is a $\sigma$-finite measure on $(S, \mathscr S)$, there exists a strictly positive measurable function $\tilde h$ on $S$ such that $N(\tilde h)<\infty$.
	Define $h(s) := \min\{ \tilde h(s), \tilde h(s)^{1/2} \}$ for each $s \in S$.
	It is clear that $h$ is a strictly positive measurable functionon on $S$ such that $N(h) < \infty$ and $N(h^2) < \infty$.
	According to \cite[Theorem 2.7]{Kyprianou2014Fluctuations}, $\mathcal N(h)$ has finite 1st and 2nd moments.
	For any $g, f \in bp\mathscr S^h := \{g \in p\mathscr S: \|h^{-1} g\|_\infty < \infty\}$, it is easy to see that $\mathcal N(g), \mathcal N(f), \mathcal N(f) \mathcal N(g)$ are integrable.
	Thus, using Lemma \ref{lem:size-biased-lemma} and the dominated convergence theorem, we have
\[\begin{split}
	P [\mathcal N(g) \mathcal N(f)]
	&= - P[\partial_\theta|_{\theta = 0} \mathcal N(g) e^{-\mathcal N(\theta f)}]
	= - \partial_\theta|_{\theta = 0} P[\mathcal N(g) e^{-\mathcal N(\theta f)}]\\
	&= - \partial_\theta|_{\theta = 0} P[e^{-\mathcal N(\theta f)}]N(ge^{-\theta f})\\
	&= - N[g] \partial_\theta|_{\theta = 0} P[e^{-\mathcal N(\theta f)}] - \partial_\theta|_{\theta = 0} N(ge^{- \theta f}) \\
	&= - N(g) P[ \partial_\theta|_{\theta = 0} e^{-\mathcal N(\theta f)}] - N( \partial_\theta|_{\theta = 0} g e^{-\theta f}) \\
	&= N(g) N(f) + N(g f).
\end{split}\]
	For any $g,f\in p\mathscr S$ and $s \in S$, define $g^{(n)}(s) := h(s) \min\{h(s)^{-1}g(s),n\}$.
	Then $(g^{(n)})_{n\in \mathbb N}$ is a $bp\mathscr S^h$-sequence which increasingly converges to $g$ pointwisely.
	Define $f^{(n)}$ similarly.
	Then from what we have proved, \eqref{eq:2rdMomPoissRandMeas} is true for $g^{(n)}$ and $f^{(n)}$.
	Letting $n\to\infty$, by monotonicity, \eqref{eq:2rdMomPoissRandMeas} is true for each $g,f\in p\mathscr S$ if we allow extended values.
	In the case when $g,f\in L^1(N) \cap L^2(N)$ we simply consider their positive and negative parts.
\end{proof}

\subsection{Size-biased transform of the superprocesses.}

	Recall that, for any $\mu\in \mathcal M_f$, $\{\mathcal N; \mathbf P_\mu\}$ is a Poisson random measure
	with mean measure $\mathbb N_\mu$,
	and the $(\xi, \Psi)$-superprocess $(X_t)_{t\ge 0}$ is defined as
\[
	X_0 := \mu;
	\quad X_t(\cdot)
	:=\mathcal N[w_t(\cdot)],
	\quad t>0.
\]
\begin{thm}\label{prop:sizBiasDecSupProc}
	Suppose that Assumption \ref{asp:1} holds.
	Let $\mu \in \mathcal M_f$, and let $F$ be a non-negative measurable function on $\mathcal W$
	with $\mathbb N_\mu(F)\in (0,\infty)$.
	Denote by $\mathbb N^F$ and $\mathbf P_\mu^{\mathcal N(F)}$ the $F$-transform of $\mathbb N$ and $\mathcal N(F)$-transform of $\mathbf P_\mu$, respectively.
	Let $\{(Y_t)_{t\geq 0}; \mathbf Q_\mu\}$ be a $\mathcal W$-valued random element with law $\mathbb N^F_\mu$ defined in a different probability space.
	Then, we have
\[
	\{(X_t)_{t\geq 0}; \mathbf P_\mu^{\mathcal N(F)}\}
	\overset{f.d.d.}{=} \{(X_t + Y_t)_{t\geq 0}; \mathbf P_\mu \otimes \mathbf Q_\mu\}.
\]
\end{thm}
\begin{proof}
	Notice that, under $\mathbf P_\mu^{\mathcal N(F)}$, $X_0\overset{\text{a.s.}}{=}\mu$ is deterministic,
	and so is $X_0+Y_0$ under  $\mathbf P_\mu \otimes \mathbf Q_\mu$ since $X_0+Y_0\overset{\text{a.s.}}=\mu$.
	Therefore, we only have to show that,
\[
	\{(X_t)_{t > 0}; \mathbf P_{\mu}^{\mathcal N(F)}\}
	\overset{f.d.d.}{=} \{(X_t + Y_t)_{t > 0}; \mathbf P_\mu\otimes \mathbf Q_\mu\}.
\]
	Since $\mathbb N_\mu(F) \in (0,\infty)$, it follows from Campbell's formula that $\mathcal N(F)$ is almost surely absolutely convergent with mean $\mathbb N_\mu(F)$ under the probability $\mathbf P_\mu$.
	It then immediately follows from Theorem \ref{prop:sizBaisPoissRandMeas} that
\[
	\{\mathcal N; \mathbf P^{\mathcal N(F)}_\mu\}
	\overset{law}{=} \{\mathcal N + \delta_Y; \mathbf P_\mu\otimes \mathbf Q_\mu\}.
\]
	This completes the proof since for any $T > 0$ and $(K,f) \in \mathcal K_T$,
\[\begin{split}
	\mathbf P_\mu^{\mathcal N(F)} [e^{-K_{(0, T]}^f(X)}]
	&= \mathbf P_\mu^{\mathcal N(F)} [e^{-\mathcal N[K_{(0, T]}^f(w)]}]
	= (\mathbf P_\mu\otimes \mathbf Q_\mu) [e^{-(\mathcal N+\delta_Y)[K_{(0, T]}^f(w)]}]\\
	&= (\mathbf P_\mu\otimes \mathbf Q_\mu) [e^{-K_{(0, T]}^f(X+Y)}]
\end{split}\]
\end{proof}

\section{Spine decomposition of superprocesses}
	The spine decomposition theorem characterizes the superprocess $\{X,\mathbf P\}$ after a martingale change of measure, and has been investigated in the literature in different situations, see \cite{EckhoffKyprianouWinkel2015Spines, EnglanderKyprianou2004Local,  LiuRenSong2009Llog} for example.
	The martingale that is used for the change of measure is defined by $M_t := e^{-\lambda t}X_t(\phi)$ where $\phi$ is the principal eigenfunction of the generator of the mean semigroup of $X$ with $\lambda$ being the corresponding eigenvalue.
	After this martingale change of measure, the transformed process preserves the Markov property, and thus, to prove the spine decomposition theorem, one only needs to focus on the one-dimensional distribution of the transformed process.
\par
	In this section, we generalize this classical result by considering the $X_T(g)$-transform of the superprocess $X$, where $g$ is a non-negative Borel function on $E$ such that $\phi^{-1} g$ is bounded from above.
	If $g$ is not equal to $\phi$, the $X_T(g)$-transformed process is typically not a Markov process.
	So we have to use a different method to develop the theorem.
	Thanks to Theorem \ref{prop:sizBiasDecSupProc},
	we only have to consider the $w_T(g)$-transform of the Kuznetsov measures.

\subsection{Principal eigenfunction of the mean semigroup}
	In the remainder of this chapter, we will always assume that the following assumption is satisfied.
\begin{asp}{2}\label{asp:2}
	There exist a $\sigma$-finite Borel measure $m$ with full support on $E$ and a family of strictly positive, bounded continuous functions $\{ p(t,\cdot,\cdot): t > 0 \}$ on $E \times E$ such that,
\begin{equation}\label{eq:kernMeanSemGroup}
	P_t f(x)
	= \int_E p(t,x,y) f(y) m(dy),
	\quad t>0, x \in E,f \in b\mathscr B_E,
\end{equation}
\[
	\int_E p(t,y,x)m(dy)
	\leq 1,
	\quad t>0,x\in E,
\]
	and
\[
	\int_E \int_E p(t,x,y)^2 m(dx) m(dy)
	<\infty,
	\quad t> 0.
\]
	Moreover, $x \mapsto \int_E p(t,x,y)^2 m(dy)$ and $x \mapsto \int_E p(t,y,x)^2 m(dy)$ are both continuous on $E$.
\end{asp}
	Under Assumption \ref{asp:2}, it is proved in \cite{RenSongZhang2015Limit} and \cite{RenSongZhang2017Central} that the semigroup $(P_t)_{t \geq 0}$ and its adjoint semigroup $(P^*_t)_{t \geq 0}$ are both strongly continuous semigroups of compact operators in $L^2(E,m)$.
	According to \cite[Lemma 2.1]{RenSongZhang2015Limit}, there exists a function $q(t,x,y)$ on $(0,\infty) \times E \times E$ which is continuous in $(x,y)$ for each $t>0$ such that
\[
	e^{-\|\beta\|_\infty t} p(t,x,y)
	\leq q(t,x,y)
	\leq e^{\|\beta\|_\infty t} p(t,x,y),
	\quad t>0, x, y\in E,
\]
	and that for any $t>0, x\in E$ and $f \in b\mathscr B_E$,
\[
	S_t f(x)
	= \int_E q(t,x,y) f(y) m(dy).
\]
	Define a family of transition kernels $(S^*_t)_{t \geq 0}$ on $E$ such that
\[
	S^*_0 = I;
	\quad S^*_t f(x)
	:= \int_E q(t,y,x) f(y) m(dy),
	\quad t>0, x\in E, f\in b\mathscr B_E.
\]
	It is clear that $(S^*_t)_{t \geq 0}$ is the adjoint semigroup of $(S_t)_{t \geq 0}$ in $L^2(E,m)$.
	It is proved in \cite{RenSongZhang2015Limit} and \cite{RenSongZhang2017Central} that $(S_t)_{t \geq 0}$ and $(S^*_t)_{t \geq 0}$ are also strongly continuous semigroups of compact operators in $L^2(E,m)$.
	Let $L$ and $L^*$ be the generators of the semigroups $(S_t)_{t \geq 0}$ and $(S^*_t)_{t \geq 0}$, respectively.
	Denote by $\sigma(L)$ and $\sigma(L^*)$ the spectra of $L$ and $L^*$, respectively.
	According to \cite[Theorem V.6.6.]{Schaefer1974Banach}, $\lambda := \sup \text{Re}(\sigma(L)) = \sup \text{Re}(\sigma(L^*))$ is a common eigenvalue of multiplicity $1$ for both $L$ and $L^*$.
	Using the argument in \cite{RenSongZhang2015Limit}, the eigenfunctions $\phi$ of $L$ and $\phi^*$ of $L^*$ associated with the eigenvalue $\lambda$ can be chosen to be strictly positive and continuous everywhere on $E$.
	We further normalize $\phi$ and $\phi^*$ so that $\langle\phi, \phi\rangle_m = \langle\phi,\phi^*\rangle_m = 1$.
	Moreover, for each $t\geq 0,x\in E$, we have $S_t \phi(x) = e^{\lambda t} \phi(x)$ and $S^*_t \phi(x) = e^{\lambda t} \phi^*(x)$.
	Since our main interest in this article is on critical superprocesses, we assume that $\lambda = 0$.

\subsection{Spine decomposition}
	Denote by $\mathcal M_f^\phi$ the collection of all the measures $\mu \in \mathcal M_f$ such that $\mu(\phi) \in (0,\infty)$.
	Denote by $b\mathscr B_E^\phi$ and $bp\mathscr B_E^\phi$ the collection of all the Borel functions $g$ on $E$ such that $\phi^{-1} g\in b\mathscr B_E$ and $\phi^{-1} g\in bp\mathscr B_E$, respectively.
	By monotonicity, one can verify that for each $\mu \in \mathcal M^\phi_f, t \geq 0$ and $g \in bp\mathscr B_E^\phi$, $\mathbf P_\mu[X_t(g)] = \mu(S_t g) \leq \mu(\phi) \|\phi^{-1} g\|_\infty < \infty$.
\par
	It is clear that $\{(\phi(\xi_t) e^{\int_0^t \beta(\xi_s) ds})_{t \geq 0}; \mathbb P\}$ is a non-negative martingale.
	Let $\{\xi; \dot{\mathbb P}\}$ be the martingale transform of $\{\xi; \mathbb P\}$ via this martingale in the sense that
\[
	\frac {d \dot{\mathbb P}_x|_{\mathscr F_t^\xi}} {d \mathbb P_x|_{\mathscr F_t^\xi}}
	:= \frac {\phi(\xi_t)} {\phi(x)} e^{\int_0^t \beta(\xi_s) ds},
	\quad x \in E,t \geq 0.
\]
	One can verify that $\{\xi; \dot{\mathbb P}\}$ is a time homogeneous Markov process and that its transition density with respect to the measure $m$ is
\[
	\dot{q}(t,x,y)
	:= \frac {\phi(y)} {\phi(x)} q(t,x,y),
	\quad x, y\in E, t\geq 0.
\]
	One can also verify that $\phi(x)\phi^*(x)m(dx)$ is a invariant measure for $\{\xi; \dot{\mathbb P}\}$.
\par
	Fix a measure $\mu \in \mathcal M^\phi_f$.
	Define $M_t := X_t(\phi)$ for each $t \geq 0$.
	It is clear that $\{(M_t)_{t\geq 0}; \mathbf P_\mu\}$ is a non-negative martingale.
	Let $\{(X_t)_{t\geq 0}; \mathbf P_\mu^M\}$ be the martingale transform of $\{(X_t)_{t\geq 0}; \mathbf P_\mu\}$ via this martingale in the sense that
\[
	\frac {d \mathbf P_\mu^M |_{\mathscr F_t^X}} {d \mathbf P_\mu |_{\mathscr F_t^X}}
	:= \frac {M_t} {\mu(\phi)},
	\quad t \geq 0.
\]
\par
	The \emph{spine immigration $\{(Y_t)_{t\geq 0}, (\xi_t)_{t\geq 0}, \mathbf n; \dot {\mathbf P}_\mu\}$ corresponding to $\{X; \mathbf P_\mu\}$}
	is defined as follows:
\begin{itemize}
\item
	The \emph{spine process} $\{\xi; \dot{\mathbf P}_\mu\}$ is a copy of $\{\xi; \dot{\mathbb P}_{\mu\phi}\}$, where
\[
	\dot {\mathbb P}_{\mu\phi}(\cdot)
	:= \mu(\phi)^{-1} \int \mu(dx) \phi(x) \dot{\mathbb P}_x(\cdot).
\]
\item
    Conditioned on $\xi$, the measure $\mathbf n$ is a Poisson random measure on $[0,\infty) \times \mathcal W$ with mean measure
\[
	\mathbf m^\xi(ds,dw)
	:= 2 \alpha(\xi_s) ds \cdot \mathbb N_{\xi_s}(dw) + ds \cdot \int_{(0,\infty)} y \mathbf P_{y \delta_{\xi_s}}(dw) \pi(\xi_s,dy).
\]
\item
	The \emph{immigration} $\{Y; \dot{\mathbf P}_\mu\}$ is an $\mathcal M_f$-valued process defined by
\[
	Y_t
	:= \int_{(0,t] \times \mathcal W} w_{t-s} \mathbf n(ds,dw),
	\quad t \geq 0.
\]
\end{itemize}
\begin{thm}[{Spine decomposition, \cite{EckhoffKyprianouWinkel2015Spines, EnglanderKyprianou2004Local, LiuRenSong2009Llog}}]\label{thm:spinDec}
    Suppose that Assumptions \ref{asp:1} and \ref{asp:2} hold.
	Let $\mu \in \mathcal M_f^\phi$, and let $\{Y; \dot {\mathbf P}_\mu\}$ be the spine immigration corresponding to $\{X; \mathbf P_\mu\}$ defined above.
	Then, we have
\[
    \{(X_t)_{t\geq 0}; \mathbf P_\mu^M\}
    \overset{f.d.d.}{=} \{(X_t + Y_t)_{t\geq 0}; \mathbf P_\mu \otimes\dot{\mathbf P}_\mu\}.
\]
\end{thm}
\par
	We will prove Theorem \ref{thm:spinDec} by providing a more general result.
	Fixing a time $T>0$ and a function $g \in bp\mathscr B^\phi_E$ with $m(g)>0$, we define a process $\{(\xi_t)_{0 \leq t \leq T}; \mathbb P^{(g, T)}\}$ as the $(g(\xi_T) e^{-\int_0^T \beta(\xi_s) ds})$-transform of process $\{(\xi_t)_{0\leq t\leq T}; \mathbb P\}$ in the sense that
\[
	\frac {d \mathbb P^{(g,T)}_x} {d \mathbb P_x}
	:= \frac {g(\xi_T) e^{-\int_0^T \beta(\xi_s) ds}} {S_T g(x)},
	\quad x \in E.
\]
	Notice that, since $m(g) > 0$ and $q(T,\cdot,\cdot)$ is strictly positive on $E \times E$, we have
\[
	\mathbb P_x[ g(\xi_T) e^{-\int_0^T \beta(\xi_s) ds} ]
	= S_T g(x)
	= \int_E q(T,x,y) g(y) m(dy)
	> 0,
	\quad x \in E,
\]
	and therefore the probability $\mathbb P_x^{(g,T)}$ is well defined.
\par
	We define the \emph{$(g,T)$-spine immigration $\{(Y_t)_{0\leq t\leq T}, (\xi_t)_{0\leq t\leq T}, \mathbf n_T; \dot {\mathbf P}^{(g,T)}_\mu\}$ corresponding to $\{X; \mathbf P_\mu\}$} as follows:
\begin{itemize}
\item
	The \emph{spine process} $\{(\xi_t)_{0\leq t\leq T}; \dot{\mathbf P}^{(g,T)}_\mu\}$ is a copy of $\{(\xi_t)_{0\leq t\leq T}; \mathbb P^{(g,T)}_{\mu S_T g}\}$, where
\[
	\mathbb P_{\mu S_T g}^{(g,T)}(\cdot)
	:= \langle\mu, S_T g\rangle^{-1} \int_{x\in E} \mu(dx) S_T g(x) \mathbb P^{(g,T)}_{x} (\cdot).
\]
\item
    Conditioned on $(\xi_t)_{0 \leq t\leq T}$, the measure $\mathbf n_T$ is a Poisson random measure on $[0,T] \times \mathcal W$ with mean measure
\begin{equation}\label{eq:meanMeasImmigr}
	\mathbf m^\xi_T(ds,dw)
	:= 2 \alpha(\xi_s) ds \cdot \mathbb N_{\xi_s}(dw) + ds \cdot \int_{(0,\infty)} y \mathbf P_{y\delta_{\xi_s}}(dw) \pi(\xi_s,dy).
\end{equation}
\item
	The \emph{immigration} $\{(Y_t)_{0\leq t\leq T}; \dot{\mathbf P}^{(g,T)}_\mu\}$ is an $\mathcal M_f$-valued process defined by
\begin{equation}\label{eq:defSpinImmigr}
	Y_t
	:= \int_{(0,t] \times \mathcal W} w_{t-s} \mathbf n_T(ds,dw),
	\quad 0 \leq t\leq T.
\end{equation}
\end{itemize}
	Note that, by this construction, $\{(Y_t)_{0 \leq t \leq T}; \dot {\mathbf P}_\mu\} \overset{f.d.d}{=} \{(Y_t)_{0\leq t\leq T}; \dot{\mathbf P}^{(\phi,T)}_\mu\}$.
\begin{thm}[Spine decomposition]\label{prop:sizBiasNMeas}
	Suppose that Assumptions \ref{asp:1} and \ref{asp:2} hold.
	Fix a time $T>0$, a measure $\mu \in \mathcal M_f^\phi$ and a function $g \in bp\mathscr B^\phi_E$ with $m(g) > 0$.
	Let $\{Y; \dot {\mathbf P}^{(g,T)}_\mu\}$ be the $(g,T)$-spine immigration corresponding to $\{X; \mathbf P_\mu\}$ defined above.
	Then we have
\[
	\{(Y_t)_{t\leq T}; \dot{\mathbf P}^{(g,T)}_\mu\}
	\overset{f.d.d.}{=} \{(w_t)_{t\leq T}; \mathbb N_\mu^{w_T(g)}\},
\]
	where $\mathbb N_{\mu}^{w_T(g)}$ is the $w_T(g)$-transform of the Kuznetsov measure $\mathbb N_\mu$.
\end{thm}
	The proof of Theorem \ref{prop:sizBiasNMeas} relies on the following lemmas:
\begin{lem}\label{lem:relSpinNMeas}
	For any $x\in E, T>0, (K,f) \in \mathcal K_T$ and $g \in p\mathscr B_E$, we have
\begin{equation}\label{eq:relSpinNMeas}
	\mathbb N_x[w_T(g) e^{-K_{(0, T]}^f(w)}]
	= \mathbb P_x[g(\xi_T) e^{-\int_0^T \Psi'(u_s)(\xi_s) ds}],
\end{equation}
	where
\[
	\Psi'(x,z)
	:= \partial_z \Psi(x,z)
	= - \beta(x) + 2 \alpha(x) z + \int_{(0,\infty)} (1 - e^{-yz}) y \pi(x,dy),
	\quad x \in E, z \geq 0,
\]
	and $u: (s,x) \mapsto u_s(x)$ on $[0,T] \times E$ is defined in Lemma \ref{lem:EquatDescNmeas}.
\end{lem}
\begin{proof}
	We first prove assertion \eqref{eq:relSpinNMeas} in the case when $g \in bp\mathscr B_E$.
	Throughout this proof, we fix $(K,f) \in \mathcal K_T$ and consider $0 \leq \theta \leq 1$.
	Define
\begin{equation}\label{eq:mm-1}
	u_s^\theta(x)
	:=	\mathbb N_x \big[ 1 - e^{ - K_{(s, T]}^f(w) - w_{T-s}(\theta g)} \big],
	\quad s \geq 0, x\in E.
\end{equation}
	Let
\[\begin{split}
	\tilde K(dr)
	&:=	\mathbf 1_{0\leq r<T} K(dr) + \delta_T (dr), \\
	\tilde f_r
	&:=	\mathbf 1_{0\leq r<T} f_r + \mathbf 1_{r=T} \big( K(\{T\}) f_T + \theta g \big).
\end{split}\]
	Then $(\tilde K, \tilde f) \in \mathcal K_T$ and \eqref{eq:mm-1} can be rewritten as
\[
	u_s^\theta(x)
	:=	\mathbb N_x \big[ 1 - e^{ - \tilde K_{(s, T]}^{\tilde f} (w)} \big],
	\quad s \geq 0, x\in E.
\]
	It follows from Lemma \ref{lem:EquatDescNmeas} that, for any
$\theta \geq 0$, $(s,x) \mapsto u^\theta_s(x)$ is the unique bounded positive solution to the equation
\[	
	u_s^\theta(x)
	= \mathbb P_{x} \Big[ \int_{(s,T]} \tilde f_r(\xi_{r-s}) \tilde K(dr) - \int_s^T \Psi (u_r^\theta) (\xi_{r-s}) dr \Big],
	\quad s\in [0,T],x\in E,
\]
	which is equivalent to
\begin{equation}\label{eq:mewth}
	u_s^\theta(x)
    = \mathbb P_{x} \Big[\int_{(s,T]} f_r(\xi_{r-s}) K(dr) + \theta g(\xi_{T-s}) - \int_s^T \Psi(u^\theta_r)(\xi_{r-s}) dr\Big].
\end{equation}
	We claim that $u_s^\theta(x)$ is differentiable in $\theta$ at $\theta = 0$.
	In fact, since
\begin{equation}\label{eq:sbN-1}
	\frac {|e^{-K_{(s, T]}^f(w)-w_{T-s}(\theta g)} - e^{-K_{(s, T]}^f(w)}|} {\theta}
	\leq w_{T-s}(g),
	\quad 0 < \theta \leq 1,
\end{equation}
	and
\begin{equation}
\label{eq:sbN-2}
	\mathbb N_x[w_{T-s}(g)]
	= S_{T-s} g (x)
	= \mathbb P_x[e^{\int_0^{T-s} \beta(\xi_r) dr} g(\xi_{T-s})]
	\leq e^{T \|\beta\|_\infty} \|g\|_\infty,
\end{equation}
	it follows from \eqref{eq:mm-1} and the dominated convergence theorem that
\begin{equation}
\label{eq:mm-3}
	\dot u_s(x)
	:= \partial_\theta|_{\theta=0} u_s^\theta(x)
	= \mathbb N_x[w_{T-s}(g) e^{-K^f_{(s,T]}(w)}]
	\leq e^{T \|\beta\|_\infty} \|g\|_\infty.
\end{equation}
	From \eqref{eq:mm-1}, we also have the following upper bound for $u_s^\theta (x)$ with $0 \leq \theta \leq 1$:
\begin{equation}\label{eq:uppBoundUThet}\begin{split}
	u_s^\theta(x)
	&\leq \mathbb N_x \Big[\int_{(s,T]} w_{r-s}(f_r) K(dr) + w_{T-s}(\theta g)\Big]\\
	&= \int_{(s,T]} \mathbb N_x[w_{r-s}(f_r)] K(dr ) + \mathbb N_x[w_{T-s}(\theta g)]\\
	&\leq e^{T\|\beta\|_\infty} \big( \|f\|_\infty K((0,T])+\|g\|_\infty \big)
	= : L_0.
\end{split}\end{equation}
	By elementary analysis, one can verify that, for each $L>0$, there exists a constant $C_{\Psi,L}>0$ such that for each $x\in E$ and $0\leq z, z_0 \leq L$,
\begin{equation}\label{eq:lipschPhi}
	|\Psi(x,z_0) - \Psi(x,z)|
	\leq C_{\Psi,L} |z - z_0|.
\end{equation}
	In fact one can choose
	$C_{\Psi,L} := \|\beta\|_\infty + 2L\|\alpha\|_\infty + \max\{L,1\} \sup_{x\in E} \int_{(0,\infty)} (y \wedge y^2) \pi(x,dy)$.
	This upper bound also implies that
\[
	|\Psi' (x,z)|
	\leq C_{\Psi,L},
	\quad x \in E, 0 \leq z \leq L.
\]
	Therefore, we can verify that $\mathbb P_x[\int_s^T \Psi(u^\theta_r)(\xi_{r-s}) dr]$ is differentiable in $\theta$ at $\theta = 0$.
	In fact, by \eqref{eq:lipschPhi}, \eqref{eq:uppBoundUThet}, \eqref{eq:mm-1}, \eqref{eq:sbN-1} and \eqref{eq:sbN-2}, we have
\[\begin{split}
	\frac {|\Psi(u_r^\theta)(x)- \Psi(u^0_r)(x)|} {\theta}
	&\leq C_{\Psi,L_0} \frac {|u_r^\theta(x) - u_r^0(x)|} {\theta}\\
	&\leq C_{\Psi,L_0} \cdot e^{T \|\beta\|_\infty} \|g\|_\infty,
	\quad 0 \leq \theta \leq 1.
\end{split}\]
	Therefore, by the bounded convergence theorem, we have
\begin{equation}\label{eq:mm-5}\begin{split}
	\partial_\theta|_{\theta =0} \mathbb P_x \Big[ \int_s^T \Psi(u_r^\theta)(\xi_{r-s}) dr\Big]
	= \mathbb P_x \Big[ \int_s^T \big(\Psi'(u_r^0) \cdot \dot u_r \big)(\xi_{r-s}) dr \Big].
\end{split}\end{equation}
	Now, taking $\partial_\theta|_{\theta=0}$ on the both sides of \eqref{eq:mewth}, we obtain from \eqref{eq:mm-5} that
\begin{equation}\label{eq:mm-2}
	\dot u_s (x)
	= \mathbb P_x\Big[ g(\xi_{T-s})- \int_s^T \big(\Psi'(u_r^0)\cdot \dot u_r\big)(\xi_{r-s}) dr \Big],
	\quad s\in [0,T], x\in E.
\end{equation}
	Notice that the function $\dot u:(s,x) \mapsto \dot u_s(x)$ is bounded on $[0,T] \times E$ by $e^{T \|\beta\|_\infty} \|g\|_\infty$;
	$g$ is bounded on $E$ by $\|g\|_\infty$;
	and $\Psi'(u_r^0)$ is bounded on $E$ by $C_{\Psi,L_0}$.
	These bounds allow us to apply the classical Fymann-Kac fromula, see \cite[Lemma A.1.5]{Dynkin1993Superprocesses} for example, to equation \eqref{eq:mm-2} and get that
\begin{equation}\label{eq:third}
	\dot u_0(x)
	= \mathbb P_{x} \Big[g(\xi_T) e^{-\int_0^T \Psi'(u_s)(\xi_s) ds}\Big].
\end{equation}
	The desired result when $g \in bp\mathscr B_E$ then follows from \eqref{eq:mm-3} and \eqref{eq:third}.
\par
	In the case when $g \in p\mathscr B_E$, we write $g^{(n)}(x) := \min\{g(x), n\}$ for $x \in E$ and $n \in \mathbb N$.
	Then, from what we have proved, we know that
\[
	\mathbb N_x[w_T(g^{(n)}) e^{- K_{(0,T]}^f(w)}]
	= \mathbb P_x[g^{(n)}(\xi_T) e^{-\int_0^T \Psi'(u_s)(\xi_s)ds}],
	\quad n \in \mathbb N.
\]
	Letting $n \to \infty$ we complete the proof.
\end{proof}
\begin{lem}\label{lem:spinImigrCondSpin}
	Let $T>0, k\in[0, T], (K,f) \in \mathcal K_T, \mu\in \mathcal M^\phi_f$ and $g\in bp\mathscr B^\phi_E$.
	We have
\begin{equation}\label{eq:spinImigrCondSpin}
    -\log \dot{\mathbf P}^{(g,T)}_\mu[e^{-K^f_{(k, T]}(Y)}|\xi]
	=\int_k^T \Psi'_0(u_s)(\xi_{s-k}) ds,
\end{equation}
	where
\[
	\Psi'_0(x,z)
	:= \partial_z\Psi_0(x,z)
	= 2 \alpha(x) z + \int_{(0,\infty)}(1 - e^{-yz})y \pi(x,dy),
	\quad x\in E, z\geq 0,
\]
	and the function $u$ is defined in Lemma \ref{lem:EquatDescNmeas}.
\end{lem}
\begin{proof}
	Throughout this proof,
we denote by $\mathbf n_{T-k}$ and $\mathbf m^\xi_{T-k}$
	the restriction of $\mathbf n_T$ and $\mathbf m^\xi_T$ on $[0,T-k] \times \mathcal W$ respectively.
	It follows from the properties of Poisson random measures that, conditioned on $\xi$, $\mathbf n_{T-k}$ is a Poisson random measure with mean measure $\mathbf m^\xi_{T-k}$.
\par
	It follows from \eqref{eq:defSpinImmigr} and Fubini's theorem that
\begin{equation}\label{eq:kfy_is_nkf}\begin{split}
    &K_{(k,T]}^f(Y)
    = \int_{(k,T]} Y_{r-k}(f_r) K(dr) \\
	&\quad = \int_{(k,T]} K(dr) \int_{(0,r-k] \times \mathcal M_f} w_{(r-k)-s}(f_r) \mathbf n_T(ds,dw)\\
	&\quad = \int_{(0,T-k] \times \mathcal M_f} \mathbf n_T(ds,dw) \int_{(k+s,T]} w_{r-(k+s)}(f_r) K(dr)\\
    &\quad = \int K^f_{(k + s, T]}(w)\mathbf n_{T-k}(ds,dw).
\end{split}\end{equation}
	Conditioned on $\xi$, it follows from Campbell's formula and Lemma \ref{lem:EquatDescNmeas} that
\[\begin{split}
    &-\log \dot{\mathbf P}^{(g,T)}_\mu[e^{-K^f_{(k, T]}(Y)}|\xi]
    = -\log \dot{\mathbf P}^{(g,T)}_\mu\big[e^{-\int K^f_{(k + s, T]}(w)\mathbf n_{T-k}(ds,dw)}\big|\xi\big]\\
	&\quad = \int(1 - e^{-K_{(k + s, T]}^f (w)})\mathbf m^\xi_{T-k}(ds,dw)\\
	&\quad = \int_0^{T-k} \Big(2\alpha(\xi_s) \mathbb N_{\xi_s}[1 - e^{-K_{(k + s, T]}^f(w)}] \\
	&\qquad\qquad\qquad + \int_{(0,\infty)} y \mathbf P_{y \delta_{\xi_s}}[1 - e^{-K_{(k + s, T]}^f(X)}] \pi(\xi_s,dy)\Big) ds\\
	&\quad = \int_0^{T-k} \Big(2\alpha(\xi_s) u_{k+s}(\xi_s) + \int_{(0,\infty)} (1 - e^{-yu_{k+s}(\xi_s)})y\pi(\xi_s,dy)\Big) ds\\
	&\quad =\int_0^{T-k} \Psi'_0(u_{s+k})(\xi_s) ds
	=\int_k^T \Psi'_0(u_s)(\xi_{s-k}) ds,
\end{split}\]
	as desired.
\end{proof}
\begin{proof}[Proof of Theorem \ref{prop:sizBiasNMeas}]
    We only need to prove that
\[
	\{(Y_t)_{0<t\le T}; \dot{\mathbf P}^{(g,T)}_\mu\}
	\overset{f.d.d.}{=} \{(w_t)_{0<t\le T}; \mathbb N_\mu^{w_T(g)}\},
\]
	since both $\{Y_0; \dot{\mathbf P}^{(g,T)}_\mu\}$ and $\{w_0; \mathbb N_\mu^{w_T(g)}\}$ are deterministic with common value $\mathbf 0$.
	By Lemma \ref{lem:relSpinNMeas} and \ref{lem:spinImigrCondSpin}, we have
\[\begin{split}
    &\mathbb N_\mu^{w_T(g)}\big[e^{-K_{(0, T]}^f(w)}\big]
	=\mathbb N_\mu[w_T(g)]^{-1} \mathbb N_\mu \big[w_T(g) e^{-K_{(0, T]}^f(w)}\big]\\
	&\quad =\langle S_Tg,\mu \rangle^{-1} \int_E \mu(dx) \mathbb N_x \big[w_T(g) e^{-K_{(0, T]}^f(w)}\big]\\
	&\quad =\langle S_Tg,\mu \rangle^{-1} \int_E \mu(dx) \mathbb P_x\big[g(\xi_T) e^{-\int_0^T \Psi'(u_s)(\xi_s) ds}\big]\\
	&\quad =\langle S_Tg,\mu \rangle^{-1} \int_E \mu(dx) S_Tg(x) \mathbb P^{(g,T)}_x \big[e^{-\int_0^T \Psi'_0(u_s)(\xi_s) ds}\big]\\
	&\quad =\mathbb P^{(g,T)}_{\mu S_Tg}[e^{-\int_0^T \Psi'_0(u_s)(\xi_s) ds}]
	=\dot {\mathbf P}^{(g,T)}_\mu\big[ \dot{\mathbf P}^{(g,T)}_\mu[e^{-K_{(0, T]}^f(Y)}|\xi] \big]\\
	&\quad = \dot{\mathbf P}^{(g,T)}_\mu[e^{-K_{(0, T]}^f(Y)}].
\end{split}\]
	The proof is complete.
\end{proof}
\begin{proof}[Proof of Theorem \ref{thm:spinDec}]
	Fix $T>0$.
	We only need to show that
\[
	\{(X_t)_{t\leq T}; \mathbf P_\mu^M\}
	\overset{f.d.d.}{=} \{(X_t + Y_t)_{t\leq T}; \mathbf P_\mu\otimes\dot{\mathbf P}_\mu\}.
\]
	Actually, it follows from Theorem \ref{prop:sizBiasNMeas} that
\[
	\{(Y_t)_{t\leq T}; \dot{\mathbf P}_\mu\}
	\overset{f.d.d.}{=} \{(Y_t)_{t\leq T}; \dot{\mathbf P}^{(\phi,T)}_\mu\}
	\overset{f.d.d.}{=} \{(w_t)_{t\leq T}; \mathbb N_\mu^{w_T(\phi)}\}.
\]
	So, applying Theorem \ref{prop:sizBiasDecSupProc}
	with $F(w) := w_T(\phi)$, we have
\[
	\{(X_t+Y_t)_{t\leq T}; \mathbf P_\mu\otimes\dot{\mathbf P}_\mu\}
	\overset{f.d.d.}{=} \{(X_t)_{t\leq T}; \mathbf P_\mu^{X_T(\phi)}\}
	\overset{f.d.d.}{=} \{(X_t)_{t\leq T}; \mathbf P_\mu^M\}
\]
	as desired.
\end{proof}

\section{2-spine decomposition of critical superprocesses}
	In this section, we develop a 2-spine decomposition for critical superprocesses.
	We will take $F = w_T(\phi)^2$ in Theorem \ref{prop:sizBiasDecSupProc}
	and investigate the $w_T(\phi)^2$-transform of the Kuznetsov measure $\mathbb N_\mu$.
	In order to do so, some moment assumptions are needed to ensure that $\mathbb N_\mu[w_T(\phi)^2] \in (0,\infty)$.
\begin{asp}\label{asp:3}
	The function $A\phi$ is bounded on $E$ where
\[
	A(x)
	:= 2\alpha(x) + \int_{(0,\infty)} y^2\pi(x,dy),
	\quad x\in E.
\]
\end{asp}
	Assumption \ref{asp:3} is satisfied, for example, when $\phi$ and $A$ are bounded on $E$.
	These conditions appeared in the literature and was used by \cite{RenSongZhang2015Limit} in the proof of the Kolmogrov type result \eqref{eq:Mot-3} and the Yaglom type result \eqref{eq:Mot-4} for critical superprocesses.
	All the example in \cite{RenSongZhang2015Limit} satisfy these conditions.
	However, our results are valid under the weaker condition of Assumption \ref{asp:3}.

\subsection{Second moment formula}
\begin{lem}\label{lem:1stMomSizBiasSupProc}
	Suppose that Assumption  \ref{asp:1}, \ref{asp:2} and \ref{asp:3} hold, then
\[
	\mathbf P_{\mu}^{(g,t)} [Y_t(f)|\xi]
    = \int_0^t ( A S_{t-s} f) (\xi_s) ds
	\leq t \|A \phi\|_{\infty} \|\phi^{-1}f\|_\infty,
	\quad \mathbf P^{(g,t)}_{\mu} \text{-a.s.},
\]
	for each $g,f\in bp\mathscr B_E^\phi, \mu \in \mathcal M^\phi_f$ and $t\geq 0$.
\end{lem}
\begin{proof}
	Define $G(s,w) := \mathbf 1_{s\leq t} w_{t-s}(f)$ for each $s \geq 0$ and $w \in \mathcal W$.
	Under Assumption \ref{asp:3}, it is clear from \eqref{eq:meanMeasImmigr} that
\[\begin{split}
	\mathbf m_t^\xi(G)
	&= \int_0^t 2 \alpha(\xi_s) \mathbb N_{\xi_s}[w_{t-s}(f)]ds + \int_0^t ds \int_{(0,\infty)} y \mathbf P_{y \delta_{\xi_s}}[X_{t-s}(f)] \pi(\xi_s,dy) \\
	&= \int_0^t 2 \alpha S_{t-s}f(\xi_s) ds + \int_0^t ds \int_{(0,\infty)} y^2 S_{t-s} f(\xi_s) \pi(\xi_s, dy)\\
    &= \int_0^t (A S_{t-s} f) (\xi_s) ds
	\leq t \cdot \|A \phi\|_{\infty} \cdot \|\phi^{-1}f\|_\infty.
\end{split}\]
	Since, conditioned on $\xi$, $\{\mathbf n_t;\mathbf P^{(g,t)}_\mu\}$ is a
	Poisson random measure on $[0,t] \times \mathcal W$
	with mean measure $\mathbf m_t^\xi$, we conclude from Campbell's theorem that
\[
	\mathbf P^{(g,t)}_{\mu}[Y_t(f)|\xi]
	= \mathbf P^{(g,t)}_{\mu} [\mathbf n_t(G)|\xi]
	= \mathbf m_t^\xi(G)
    = \int_0^t (A S_{t-s} f) (\xi_s) ds,
	\quad \mathbf P^{(g,t)}_{\mu} \text{-a.s.}.
\]
	Noticing that
\[
    \int_0^t (A S_{t-s} f)(\xi_s)ds
	 = \int_0^t [(A \phi) \phi^{-1} S_{t-s}(\phi \cdot \phi^{-1} f)](\xi_s)ds
	\leq t \|A\phi\|_\infty \|\phi^{-1}f\|_\infty,
\]
	we have our result as desired.
\end{proof}
\begin{prop}\label{prop:covanrance}
	Under Assumptions \ref{asp:1}, \ref{asp:2} and \ref{asp:3}, for each $g,f\in b\mathscr B^\phi_E$, $\mu \in \mathcal M^\phi_f$ 
	and $t\geq 0$, 
	we have that $X_t(g)X_t(f)$ is integrable with respect to $\mathbf P_\mu$ and
\begin{equation}\label{eq:covanrance}
	\mathbf P_\mu[X_t(g) X_t( f)]
	= \langle\mu, S_t g \rangle \langle\mu, S_t f\rangle + \langle\mu, \phi\rangle \dot{\mathbb P}_{\mu\phi} \Big[\phi^{-1} g(\xi_t) \int_0^t( A S_{t-s} f)(\xi_s) ds\Big].
\end{equation}
\end{prop}
\begin{proof}
	We first consider the case when $g,f\in bp\mathscr B^\phi_E$.
	In this case, the right hand of \eqref{eq:covanrance} is positive and finite.
	Actually,
	by Lemma \ref{lem:1stMomSizBiasSupProc},
	the right side of \eqref{eq:covanrance} is less than or equal to
\[\begin{split}
	&\langle \mu, S_tg \rangle \langle\mu,S_t f \rangle + \langle \mu,\phi\rangle\dot{\mathbb P}_{\mu\phi}\big[\phi^{-1}g(\xi_t)\big] t \|A\phi\|_\infty \|\phi^{-1}f\|_\infty\\
	&\quad \leq \big(\langle \mu,\phi \rangle^2 + \langle \mu,\phi\rangle t \|A\phi\|_\infty\big)\|\phi^{-1}g\|_\infty\|\phi^{-1}f\|_\infty
	< \infty.
\end{split}\]
	We can also assume that $m(g)>0$.
	Since if $m(g)=0$,
then both sides of \eqref{eq:covanrance} are $0$.
\par
	Now in the case when $g,f\in bp\mathscr B^\phi_E$ and $m(g)>0$, from
    Theorem \ref{prop:sizBiasNMeas} and
	Lemma \ref{lem:1stMomSizBiasSupProc} we know that, for each $x\in E$,
\[\begin{split}
	\mathbb N_x^{w_t(g)}[w_t(f)]
	&= \mathbf P_{\delta_x}^{(g,t)}[Y_t(f)]
	= \mathbf P_{\delta_x}^{(g,t)} \big[\mathbf P_{\delta_x}^{(g,t)}[Y_t(f) | \xi]\big]\\
	&= \mathbf P_{\delta_x}^{(g,t)} \Big[\int_0^t  (A S_{t-s}  f)(\xi_s) ds\Big]
	= \mathbb P_x^{(g,t)} \Big[\int_0^t  (A S_{t-s} f)(\xi_s) ds\Big]\\
	&= S_t g(x)^{-1} \mathbb P_{x} \big[g(\xi_t) e^{\int_0^t \beta(\xi_s) ds} \int_0^t (A S_{t-s} f)(\xi_s) ds\big].
\end{split}\]
	Therefore,
\[\begin{split}
	\mathbb N_x[w_t(g) w_t( f)]
	&= \mathbb N_x[w_t(g)] \mathbb N_x^{w_t(g)}[w_t(f)]\\
	&= \mathbb P_x \big[g(\xi_t) e^{\int_0^t \beta(\xi_s) ds} \int_0^t( A S_{t-s} f)(\xi_s) ds\big]\\
	&= \phi(x) \dot{\mathbb P}_x \big[(\phi^{-1} g)(\xi_t) \int_0^t( A S_{t-s} f)(\xi_s)ds \big].
\end{split}\]
	Integrating with $\mu \in \mathcal M_f^{\phi}$, we have
\[
	\mathbb N_\mu[w_t(g)w_t(f)]
	= \langle\mu, \phi\rangle \dot{\mathbb P}_{\mu\phi} \big[(\phi^{-1} g)(\xi_t) \int_0^t (A S_{t-s} f)(\xi_s) ds\big].
\]
	It then follows from Lemmas \ref{lem:lislem} and \ref{lem:covPoissRandMeas} that
\[\begin{split}
	\mathbf P_\mu[ X_t( g) X_t( f)]
	&= \mathbb N_\mu[ w_t( g)] \mathbb N_\mu[ w_t( f)] + \mathbb N_\mu[ w_t( g) w_t( f)]\\
	&= \langle \mu, S_t g\rangle \langle \mu, S_t f\rangle + \langle \mu, \phi\rangle \dot{\mathbb P}_{\mu\phi} \big[ ( \phi^{-1} g)( \xi_t) \int_0^t (A S_{t-s} f)( \xi_s) ds \big]
\end{split}\]
	as desired.
	For the more general case when $g,f\in b\mathscr B^\phi_E$, we only need to consider their positive and negative parts.
\end{proof}

\subsection{Immigrations along a 2-spine skeleton}
\label{size-biased-equation}
	Fix an arbitrary $\mu \in \mathcal M_f^\phi$.
	Consider the spine immigration $\{Y, \xi, \mathbf n; \dot{\mathbf P}_\mu\}$ corresponding to $\{X; \mathbf P_\mu\}$.
	We first state a property of $\{Y; \dot{\mathbf P}_\mu\}$, which is needed later.
\begin{lem}\label{lem:Y-is-immortal}
	$\dot{\mathbf P}_\mu(Y_t = \mathbf 0) = 0$ for each $\mu \in \mathcal M_f^\phi$ and $t>0$.
\end{lem}
\begin{proof}
	According to Theorem \ref{prop:sizBiasNMeas}, we have
\[
	\dot{\mathbf P}_\mu (Y_t = 0)
	= \mathbb N^{w_t(\phi)}_\mu ( w_t(\phi) = 0 )
	= \langle \mu, \phi \rangle^{-1} \mathbb N_\mu [ w_t(\phi) \mathbf 1_{w_t(\phi) = 0}]
	= 0.
\]
\end{proof}
\par
	We now introduce the \emph{2-spine immigration}
\[
	\{(Z_t)_{t\leq T}, (\xi_t)_{t\leq T}, \kappa, (\xi_t')_{\kappa\leq t\leq T}; \ddot {\mathbf P}_\mu^{(T)}\}
\]
	\emph{corresponding to the process} $\{X; \mathbf P_\mu\}$  as follows:
\begin{itemize}
\item
	\emph{The main spine} $\{(\xi_t)_{t\leq T}; \ddot{\mathbf P}_\mu^{(T)}\}$ is a copy of $\{(\xi_t)_{t\leq T}; \ddot{\mathbb P}_{\mu\phi}^{(T)}\}$, where
\begin{equation}\label{eq:definition-of-main-spine}
	\frac {d \ddot{\mathbb P}_{\mu\phi}^{(T)}} {d \dot{\mathbb P}_{\mu\phi}}
	:= \frac{\int_0^T (A \phi)(\xi_s) ds}{\dot{\mathbb P}_{\mu\phi}[ \int_0^T (A \phi)(\xi_s) ds]}.
\end{equation}
\item
	Conditioned on $(\xi_t)_{t \leq T}$, the \emph{splitting time} $\kappa$ is a random variable taking values in $[0,T]$ with law proportion to $\int_\cdot \mathbf 1_{s\leq T} (A \phi)(\xi_s) ds$ in the sense that
\[
	\ddot{\mathbf P}_\mu\big(\kappa \in ds\big|(\xi_t)_{t\leq T}\big)
	=\frac {\mathbf 1_{s\leq T} (A\phi)(\xi_s) ds} {\int_0^T (A\phi)(\xi_r) dr}.
\]
\item
	Conditioned on $(\xi_t)_{t \leq T}$ and $\kappa$, we define the \emph{auxiliary spine} $(\xi'_t)_{\kappa \leq t \leq T}$ such that
\begin{equation}\label{eq:defAuxilSpin}
	\{(\xi'_{\kappa+t})_{0 \leq t \leq T - \kappa}; \ddot{\mathbf P}_\mu^{(T)}(\cdot | \xi,\kappa) \}
	\overset{law}{=} \{(\xi_t)_{0 \leq t \leq T - \kappa}; \dot{\mathbb P}_{\xi_\kappa} \}.
\end{equation}
	We refer to $\{(\xi_t)_{t\leq T}, \kappa, (\xi_t')_{\kappa \leq t\leq T}; \ddot {\mathbf P}_\mu^{(T)}\}$ as \emph{the 2-spine skeleton} and define
\[
	\mathscr G
	:= \sigma \{ (\xi_t)_{t \leq T}, \kappa, (\xi_t')_{\kappa \leq t \leq T} \}.
\]
\item
	Conditioned on $\mathscr G$, \emph{the main immigration} $(Y_t)_{0 \leq t\leq T}$ is defined by
\[
	Y_t
	:= \int_{(0,t] \times \mathcal W} w_{t-s} \mathbf n_T(ds, dw),
	\quad t\in [0,T],
\]
	where $\mathbf n_T$ is a
	Poisson random measure on $[0,T] \times \mathcal W$ with mean measure
\[
	\mathbf m_T^\xi (ds,dw)
	:= 2 \alpha(\xi_s) ds \cdot \mathbb N_{\xi_s}(dw) + ds \cdot \int_{(0,\infty)} y \mathbf P_{y\delta_{\xi_s}}(dw) \pi(\xi_s,dy).
\]
\item
	Conditioned on $\mathscr G$, \emph{the auxiliary immigration} $(Y'_t)_{\kappa \leq t \leq T}$ is defined by
\[
	Y'_t
	:= \int_{(\kappa,t] \times \mathcal W} w_{t-s} \mathbf n'_T(ds,dw),
	\quad t \in [\kappa,T],
\]
	where $\mathbf n'_T$ is a
	Poisson random measure on $[\kappa,T] \times \mathcal W$ with mean measure
\[
	\mathbf m^{\xi'}_{\kappa,T}(ds,dw)
	:= 2 \alpha(\xi'_s)ds \cdot \mathbb N_{\xi'_s}(dw) + ds \cdot \int_{(0,\infty)} y \mathbf P_{y\delta_{\xi'_s}}(dw) \pi(\xi'_s,dy).
\]
\item
	Conditioned on $\mathscr G$, \emph{the splitting-time immigration}
$(X'_t)_{\kappa \leq t \leq T}$ is defined by
\[
	\{(X'_{\kappa+t})_{0\leq t\leq T-\kappa}; \ddot{\mathbf P}_\mu(\cdot | \mathscr G)\}
	\overset{law}{=} \{(X_t)_{0 \leq t \leq T-\kappa}; \widetilde{\mathbf P}_{\xi_\kappa}\},
\]
where, for each $x\in E$, the probability measure $\widetilde{\mathbf P}_{x}$ is defined by

\begin{equation}\label{eq:def-tilde-P}
	\widetilde{\mathbf P}_{x}(\cdot)
	:=
\begin{cases}
	\frac{2\alpha(x) \mathbf P_{\mathbf 0}(\cdot)+\int_{(0,\infty)}y^2\mathbf P_{y\delta_x}(\cdot)\pi(x,dy)}{2\alpha(x)+\int_{(0,\infty)}y^2\pi(x,dy)},
	&\quad \mbox{if } A(x)>0,\\
	\mathbf P_{\mathbf 0}(\cdot),
	&\quad \mbox{if } A(x)=0.
\end{cases}
\end{equation}
\item
	We assume that, conditioned on $\mathscr G$,
	$\{Y,\mathbf n_T\}$, $\{Y',\mathbf n'_T\}$ and $X'$ are independent.
	We also define $Y'_t = \mathbf 0$ and $X'_t = \mathbf 0$ for every $t\leq \kappa$.
	Finally, we define  the \emph{total immigration $(Z_t)_{0\leq t\leq T}$} by
\[
	Z_t
	:= Y_t + Y_t' + X_t',
	\quad 0\leq t\leq T.
\]
\end{itemize}
\begin{thm}\label{prop:2-spine-decomposition}
	Suppose that Assumptions \ref{asp:1}, \ref{asp:2} and \ref{asp:3} hold.
	Fix  $\mu\in\mathcal M_f^\phi$ and $T>0$.
	Let $\{Z;\ddot {\mathbf P}_\mu^{(T)}\}$ be the 2-spine immigration process defined above.
	Then
\[
	\{(Z_t)_{t\leq T}; \ddot {\mathbf P}^T_\mu\}
	\overset{f.d.d.}{=} \{(w_t)_{t\leq T}; \mathbb N^{w_T(\phi)^2}_\mu\},
\]
	where $\mathbb N^{w_T(\phi)^2}_\mu$ is the $w_T(\phi)^2$-transform of the Kuznetsov measure $\mathbb N_\mu$.
\end{thm}
\begin{rem}\label{rem:chain-rule}
	Note that $\mathbb N^{w_T(\phi)^2}_\mu$ is also the $w_T(\phi)$-transform of $\mathbb N^{w_T(\phi)}_\mu$.
	In fact, the size-biased transforms satisfy the following chain rule:
	If $g,f$ are non-negative measurable functions on some measure space $(D,\mathscr F_D,\mathbf D)$ with $\mathbf D(g) \in (0,\infty)$ and $\mathbf D(gf) \in (0,\infty)$.
	Denoted by $\mathbf D^g$ the $g$-transform of $\mathbf D$, then $(\mathbf D^g)^f = \mathbf D^{gf}$, i.e., the $f$-transform of $\mathbf D^g$ is the $gf$-transform of $\mathbf D$.
	This is true because it is easy to see that
\[
	\mathbf D^{gf}(ds)
	:= \frac{g(s) f(s) \mathbf D(ds)}{\mathbf D[gf]}
	= \frac{f(s) \mathbf D^g(ds)}{\mathbf D^g[f]}
	=: (\mathbf D^g)^f (ds),
	\quad s\in S.
\]
\end{rem}
\par
	The proof of Theorem \ref{prop:2-spine-decomposition} relies on the following lemma:
\begin{lem}
\label{lem:key-lemma}
	For any $\mu\in \mathcal M_f^\phi$, $T>0$ and $(K,f)\in \mathcal K_T$, we have
\[\begin{split}
    &\dot{\mathbf P}_\mu [Y_T(\phi) e^{-K_{(0, T]}^f(Y)}|\xi]\\
    &\quad =\dot{\mathbf P}_\mu[e^{-K_{(0, T]}^f(Y)}|\xi]\int_0^T  (A\phi)(\xi_s)\dot{\mathbf P}_{\delta_{\xi_s}}[e^{-K_{(s, T]}^f(Y)}]\widetilde {\mathbf P}_{\xi_s}[e^{-K^f_{(s, T]}(X)}]ds,
\end{split}\]
where $\widetilde {\mathbf P}_{x}$ is defined by \eqref{eq:def-tilde-P} for each $x\in E$.
\end{lem}
\begin{proof}
	Define $G(s,w) := \mathbf 1_{s\leq T}w_{T-s}(\phi)$ for each $s\geq 0$ and $w\in\mathcal W$.
	Notice that from \eqref{eq:kfy_is_nkf}, under the probability $\dot {\mathbf P}_\mu$,
	we have $Y_T(\phi) = \mathbf n(G)$ and
	$K_{(0, T]}^f(Y) = \mathbf n(K^f_{s}(w))$.
	From Lemmas \ref{lem:1stMomSizBiasSupProc} and \ref{lem:Y-is-immortal} we know that
\[
	0
	< \dot{\mathbf P}_{\mu}[Y_T(\phi)|\xi]
	< \infty,
	\quad \dot{\mathbf P}_\mu \text{-a.s.}.
\]
	Therefore, we can apply Lemma \ref{lem:size-biased-lemma} to the conditioned
	Poisson random measure $\mathbf n$, and get
\begin{equation}\label{eq:condSizBiasEquatSpinImmigr}
    \dot{\mathbf P}_\mu [\mathbf n(G) e^{-\mathbf n(K^f_{(s, T]}(w))}|\xi]
	=\dot{\mathbf P}_\mu[e^{-\mathbf n(K^f_{(s, T]}(w))}|\xi]\mathbf m^\xi[Ge^{-K^f_{(s, T]}(w)}].
\end{equation}
	It is clear from the definitions of $\mathbf m^\xi$, $\mathbb N^{w_t(\phi)}$ and $\mathbf P^M$ that
\begin{equation}\label{eq:represent-mGeKf}\begin{split}
    \mathbf m^\xi[Ge^{-K_{(s, T]}^f(w)}]
	&=\int_0^T \Big( 2\alpha(\xi_s)\mathbb N_{\xi_s}[w_{T-s}(\phi)e^{-K^f_{(s,T]}(w)}]\\
	&\qquad\qquad +\int_{(0,\infty)}y\mathbf P_{y\delta_{\xi_s}}[X_{T-s}(\phi)e^{-K^f_{(s,T]}(X)}]\pi(\xi_s,dy)\Big)ds\\
	&=\int_0^T \Big(2\alpha\phi(\xi_s)\mathbb N^{w_{T-s}(\phi)}_{\xi_s}[e^{-K^f_{(s,T]}(w)}] \\
	&\qquad\qquad + \int_{(0,\infty)}y^2\phi(\xi_s)\mathbf P^M_{y\delta_{\xi_s}}[e^{-K^f_{(s,T]}(X)}]\pi(\xi_s,dy)\Big)ds.
\end{split}\end{equation}
	According to  Theorem \ref{prop:sizBiasNMeas}, we have
\begin{equation}
\label{eq:represent-NxwTsphi}
	\mathbb N_x^{w_{T-s}(\phi)}[e^{-K^f_{(s,T]}(w)}]
	= \dot{\mathbf P}_{\delta_x}[e^{-K^f_{(s,T]}(Y)}]
	= \dot{\mathbf P}_{\delta_x}[e^{-K^f_{(s,T]}(Y)}]\mathbf P_{\mathbf 0}[e^{-K^f_{(s,T]}(X)}],
\end{equation}
	where we used the fact that $\mathbf P_{\mathbf 0}(X_t=\mathbf 0,\mbox{ for any }t\ge 0)=1$.
	According to Theorem \ref{thm:spinDec}, we have
\begin{equation}\label{eq:represent-PMydeltax}
	\mathbf P^M_{y\delta_x}[e^{-K^f_{(s,T]}(X)}]
	= \dot{\mathbf P}_{y\delta_x}[e^{-K^f_{(s,T]}(X+Y)}]
	=\dot{\mathbf P}_{\delta_x}[e^{-K^f_{(s,T]}(Y)}]\mathbf P_{y\delta_x}[e^{-K^f_{(s,T]}(X)}]
\end{equation}
	for any $s\in [0,T],x\in E$, and $y\in (0,\infty)$.
	Plugging \eqref{eq:represent-NxwTsphi} and \eqref{eq:represent-PMydeltax} back into \eqref{eq:represent-mGeKf} and rearranging terms, we have that
\begin{equation}\label{eq:fifth}
\begin{split}
    &\mathbf m^\xi[Ge^{-K_{(s, T]}^f}(w)]\\
	&\quad=\int_0^T \Big(2\alpha\phi(\xi_s)\dot{\mathbf P}_{\delta_{\xi_s}}[e^{-K^f_{(s,T]}(Y)}]\mathbf P_{\mathbf 0}[e^{-K^f_{(s,T]}(X)}] \\
	&\qquad\qquad\quad + \int_{(0,\infty)}y^2\phi(\xi_s)\dot{\mathbf P}_{\delta_{\xi_s}}[e^{-K_s^f(Y)}]\mathbf P_{y\delta_{\xi_s}}[e^{-K^f_{(s,T]}(X)}]\pi(\xi_s,dy)\Big)ds.\\
	&\quad=\int_0^T \phi(\xi_s)\dot{\mathbf P}_{\delta_{\xi_s}}[e^{-K^f_{(s,T]}(Y)}] \\
	&\qquad\qquad\times\Big(2\alpha(\xi_s)\mathbf P_{\mathbf 0}[e^{-K^f_{(s,T]}(X)}] + \int_{(0,\infty)}y^2\mathbf P_{y\delta_{\xi_s}}[e^{-K^f_{(s,T]}(X)}]\pi(\xi_s,dy)\Big)ds\\
	&\quad =\int_0^T (A\phi)(\xi_s)\dot{\mathbf P}_{\delta_{\xi_s}}[e^{-K^f_{(s,T]}(Y)}]\widetilde {\mathbf P}_{\xi_s}[e^{-K^f_{(s,T]}(X)}]ds.
\end{split}\end{equation}
	Plugging \eqref{eq:fifth} back into \eqref{eq:condSizBiasEquatSpinImmigr}, we get the desired result.
\end{proof}
\begin{proof}[Proof of Theorem \ref{prop:2-spine-decomposition}]
	By Remark \ref{rem:chain-rule}, and noticing that $\{Z_0;\ddot{\mathbf P}^{(T)}_\mu\}$ and $\{w_0;\mathbb N^{w_T(\phi)^2}_\mu\}$ are both deterministic with common value $\mathbf 0$,
	we only need to show that
$\{(Z_t)_{0< t\leq T};\ddot{\mathbf P}^{(T)}_\mu\}$ is the $w_T(\phi)$-transform of $\{(w_t)_{0< t\leq T}; \mathbb N^{w_T(\phi)}_\mu\}$.
	Note that, by Theorem\ref{prop:sizBiasNMeas},
%	we already have a process $\{(Y_t)_{0\leq t\leq T}; \dot {\mathbf P}_\mu\}$ which is a realization of $\{(w_t)_{0< t\leq T}; \mathbb N^{w_T(\phi)}_\mu\}$.
	we already have a process $\{(Y_t)_{0\leq t\leq T}; \dot {\mathbf P}_\mu\}$ which is a realization of $\{(w_t)_{0\leq t\leq T}; \mathbb N^{w_T(\phi)}_\mu\}$.
	Therefore, in order to prove Theorem \ref{prop:2-spine-decomposition},
	we only need to show that
$\{(Z_t)_{0< t\leq T};\ddot{\mathbf P}^{(T)}_\mu\}$ is the $Y_T(\phi)$-transform of process $\{(Y_t)_{0< t\leq T}; \dot {\mathbf P}_\mu\}$.
\par
	Similar to \eqref{eq:kfy_is_nkf}, on the probability space where $\ddot {\mathbf P}^{(T)}_\mu$ is defined, we have $K_{(r, T]}^f(Y) = \mathbf n_T[K^f_{(r+\cdot, T]}]$ and $K_{(r, T]}^f(Y') = \mathbf n'_T[K^f_{(r+\cdot, T]}]$ for each $r\leq T$.
	Therefore, by Campbell's theorem, using an argument similar to that used in the proof of Lemma \ref{lem:spinImigrCondSpin}, one can verify that
\begin{equation}\label{eq:mainImmigrCondSkel}\begin{split}
	- \log\ddot {\mathbf P}_\mu [e^{-K^f_{(0, T]}(Y)}|\mathscr G]
	= \int_0^T\Psi'_0(u_s)(\xi_s)ds
\end{split}\end{equation}
	and
\begin{equation}\label{eq:auxilImmigrCondSkel}
	- \log\ddot {\mathbf P}_\mu [e^{-K^f_{(0, T]}(Y')}|\mathscr G]
	= \int_\kappa^T\Psi'_0(u_s)(\xi'_s)ds,
\end{equation}
	where $u:(s,x)\mapsto u_s(x)$ is the function on $[0,T]\times E$ defined in Lemma \ref{lem:EquatDescNmeas}.
	It is then clear from \eqref{eq:auxilImmigrCondSkel}, \eqref{eq:defAuxilSpin} and Lemma \ref{lem:spinImigrCondSpin} that
\begin{equation}\label{eq:auxilImmigrCondMainSpin}\begin{split}
    \ddot{\mathbf P}_\mu[e^{-K^f_{(0, T]}(Y')}|\xi,\kappa]
	&= \ddot{\mathbf P}_\mu[e^{-\int_\kappa^T\Psi'_0(u_s)(\xi'_s)ds}|\xi,\kappa]\\
	&= \dot{\mathbb P}_{\xi_r}[e^{-\int_r^T\Psi'_0(u_s)(\xi_{s-r})ds}]|_{r=\kappa}
	= \dot{\mathbf P}_{\delta_{\xi_r}} [e^{-K^f_{(r, T]}(Y)}]|_{r=\kappa}.
\end{split}\end{equation}
	By the construction of the splitting immigration $X'$ at time $\kappa$, we also have
\begin{equation}\label{eq:oneTimImmigrCondSkel}
    \ddot {\mathbf P}_\mu[e^{-K^f_{(0, T]}(X')}|\mathscr G ]
	= \widetilde{\mathbf P}_{\xi_r}[e^{-K^f_{(r, T]}(X)}]|_{r=\kappa}.
\end{equation}
	Using \eqref{eq:mainImmigrCondSkel}, \eqref{eq:auxilImmigrCondMainSpin}, \eqref{eq:oneTimImmigrCondSkel} and the construction of the 2-spine immigration, we deduce that
\[\begin{split}
	&\ddot {\mathbf P}_\mu[e^{-K^f_{(0, T]}(Z)}|\xi,\kappa]
	=\ddot {\mathbf P}_\mu\big[ \ddot {\mathbf P}_\mu[e^{-K^f_{(0, T]}(Z)}|\mathscr G ] \big | \xi,\kappa\big]\\
	&\quad=\ddot {\mathbf P}_\mu\Big[ \ddot {\mathbf P}_\mu[e^{-K^f_{(0, T]}(Y)}|\mathscr G ] \ddot {\mathbf P}_\mu[e^{-K^f_{(0, T]}(Y')}|\mathscr G ] \ddot {\mathbf P}_\mu[e^{-K^f_{(0, T]}(X')}|\mathscr G ] \Big | \xi,\kappa\Big]\\
	&\quad=e^{-\int_0^T\Psi'_0(u_s)(\xi_s)ds}\dot {\mathbf P}_{\delta_{\xi_r}}[ e^{-K^f_{(r, T]}(Y)}]\widetilde{\mathbf P}_{\xi_r}[e^{-K^f_{(r, T]}(X)}]\big |_{r=\kappa}.
\end{split}\]
	Therefore, from the conditioned law of $\kappa$ given $\xi$, we have
\begin{equation}\label{eq:totImmigrCondSpin}\begin{split}
	&\ddot {\mathbf P}_\mu[e^{-K^f_{(0, T]}(Z)}|\xi]\\
	&\quad = \frac{e^{-\int_0^T\Psi'_0(u_s)(\xi_s)ds}}{\int_0^T (A\phi)(\xi_r)dr}\int_0^T  (A\phi)(\xi_r)\dot {\mathbf P}_{\delta_{\xi_r}}[ e^{-K^f_{(r, T]}(Y)}]\widetilde{\mathbf P}_{\xi_r}[e^{-K^f_{(r, T]}(X)}] dr.
\end{split}\end{equation}
	Taking expectation, we get that
\[\begin{split}
	&\ddot {\mathbf P}_\mu[e^{-K^f_{(0,T]}(Z)}]\\
	&\quad \overset{\text{\eqref{eq:totImmigrCondSpin}}}{=}\ddot {\mathbb P}_{\mu\phi}^{(T)}\Big\{ \frac{e^{-\int_0^T\Psi'_0(u_s)(\xi_s)ds}}{\int_0^T (A\phi)(\xi_r)dr}\int_0^T  (A\phi)(\xi_r)\dot {\mathbf P}_{\delta_{\xi_r}}[ e^{-K^f_{(r, T]}(Y)}]\widetilde{\mathbf P}_{\xi_r}[e^{-K^f_{(r, T]}(X)}] dr\Big\}\\
	&\quad \overset{\text{\eqref{eq:definition-of-main-spine}}}{=}\dot{\mathbb P}_{\mu\phi}\Big\{ \frac{e^{-\int_0^T\Psi'_0(u_s)(\xi_s)ds}}{\dot{\mathbb P}_{\mu\phi}[\int_0^T (A\phi)(\xi_r)dr]}\int_0^T  (A\phi)(\xi_r)\dot {\mathbf P}_{\delta_{\xi_r}}[ e^{K^f_{(r, T]}(Y)}]\widetilde{\mathbf P}_{\xi_r}[e^{-K^f_{(r, T]}(X)}] dr\Big\}\\
	&\quad \overset{\text{\eqref{eq:spinImigrCondSpin}}}{=} \dot{\mathbf P}_{\mu}\Big\{ \frac{\dot {\mathbf P}_\mu[e^{-K^f_{(0,T]}(Y)}|\xi]}{\dot{\mathbf P}_{\mu}[\int_0^T (A\phi)(\xi_r)dr]}\int_0^T  (A\phi)(\xi_r)\dot {\mathbf P}_{\delta_{\xi_r}}[ e^{-K^f_{(r, T]}(Y)}]\widetilde{\mathbf P}_{\xi_r}[e^{-K^f_{(r, T]}(X)}] dr\Big\}\\
	&\quad \overset{\text{Lemma \ref{lem:key-lemma}}}{=} \dot{\mathbf P}_{\mu}\Big\{ \frac{\dot {\mathbf P}_\mu[Y_T(\phi)e^{-K^f_{(0, T]}(Y)}|\xi]}{\dot{\mathbf P}_{\mu}[Y_T(\phi)]}\Big\}=\frac{\dot {\mathbf P}_\mu[Y_T(\phi)e^{-K^f_{(0, T]}(Y)}]}{\dot{\mathbf P}_{\mu}[Y_T(\phi)]},
\end{split}\]
	which says that $(Z_t)_{0< t\leq T}$ is the $Y_T(\phi)$-transform of the process $\{(Y_t)_{0< t\leq T}; \dot {\mathbf P}_\mu\}$, as desired.
\end{proof}

\section{The asymptotic behavior of a class of critical superprocesses.}\label{sec:asymptotic}

\subsection{Further assumptions}\label{sec:further_assumptions}
	In this section, we study the limit behavior of a class of critical superprocesses whose mean semigroups satisfy the following stronger regularity condition:
\begin{asp}
\label{asp:2'}
	In addition to Assumption \ref{asp:2}, we further assume that the mean semigroup $(S_t)_{t\geq 0}$ is intrinsically ultracontractive,
	that is, for each $t>0$ there exists $c_t>0$ such that for all $x,y\in E$, we have $q(t,x,y)\leq c_t\phi(x)\phi^*(y)$.
\end{asp}
	Under Assumption \ref{asp:2'}, it was shown in \cite{KimSong2008Intrinsic} there exist positive constants $c, \gamma$ such that
\begin{equation}
\label{eq:IU}
	\sup_{x,y\in E} \Big| \frac {q(t,x,y)} {\phi(x)\phi^*(y)} - 1\Big|
	= \sup_{x,y\in E} \Big| \frac { \dot{q}(t,x,y)} {\phi(y)\phi^*(y)} - 1\Big|
	\leq c e^{-\gamma t},
	\quad t>1.
\end{equation}
\par
	Throughout this chapter, we define $\nu(dy) := \phi^*(y)m(dy)$.
	Under Assumption \ref{asp:2'}, $\nu(dy)$ is a finite measure on $E$ (see \cite{RenSongZhang2015Limit}).
	In fact, $\phi^*(y) \leq q(t,x,y)  (1-c e^{-\gamma t})^{-1}\phi^{-1}(x)$ for all $t>1$ where, clearly, the right hand of this inequality is integrable in $y$ with respect to measure $m$.
    Therefore, we can consider a superprocess $X$ with initial configuration $\nu$.
\par
	According to \eqref{eq:V_infty} and \eqref{eq:linearity}, the limit behavior of the extinction probability $\mathbf P_\mu(X_t = \mathbf 0)$ can be described by the functions $(v_t)_{t > 0}$.
	Taking $f=V_s(\theta \mathbf 1_E)$ in \eqref{eq:mean-fkpp} and letting $\theta \to \infty$, by monotonicity, we get that $(v_t)_{t\geq 0}$ satisfy the following equation
\begin{equation}\label{eq:reason-for-asp2'}
	v_{t+s}(x) + \int_0^tS_{t-r}\Psi_0(v_{r+s})(x) dr
	= S_tv_s(x),
	\quad x\in E,t\geq 0.
\end{equation}
	However, in order to take advantage of this equation, we need that $S_t v_s(x)$ takes finite value at least for some large $s,t>0$ and for some $x\in E$.
	The following lemma addresses this need.
\begin{lem}\label{lem:discuss-of-assumption2'}
	Under Assumption \ref{asp:2'}, the following statements are equivalent:
\begin{enumerate}
\item[$(1)$]
	$S_tv_s(x)<\infty$ for some $s>0,t>1$ and $x\in E$.
\item[$(1')$]
	There is an $s_0>0$ such that for any $s\geq s_0$, $t>1$ and $x\in E$, we have $S_tv_s(x)<\infty$.
\item[$(1'')$]
	There is an $s_0>0$ such that for any $s\geq s_0$, $t>0$ and $x\in E$, we have $S_tv_s(x)<\infty$.
\item[$(2)$]
	$ \langle v_s, \phi^* \rangle_m < \infty$ for some $s>0$.
\item[$(2')$]
	There is an $s_0>0$ such that for any $s\geq s_0$, we have $\langle v_s, \phi^* \rangle_m < \infty$.
\item[$(3)$]
	There is an $s_0>0$ such that for any $s\geq s_0$, we have $v_s\in bp\mathscr B^\phi_E$.
\item[$(4)$]
	$\mathbf P_\nu (X_t = \mathbf 0) > 0$ for some $t>0$.
\item[$(5)$]
	$\phi^{-1}v_t$ converges to $0$ uniformly when $t\to\infty$.
\item[$(6)$]
	For any $\mu \in \mathcal M^\phi_f$, $\mathbf P_\mu(\exists t>0,~s.t.~X_t = \mathbf 0)
	= 1$.
\end{enumerate}
\end{lem}
\begin{proof}
	We first give some estimates.
	By the definition of $\{X;\mathbf P\}$, it is easy to verify that
\begin{equation}\label{eq:assertion1}
\begin{split}
	\langle v_{s_0},\phi^* \rangle_m
	&= -\log \mathbf P_\nu ( X_{s_0} = \mathbf 0 )\\
	& \geq -\log \mathbf P_\nu ( X_s = \mathbf 0 )
	= \langle v_s,\phi^* \rangle_m,
	\quad 0 < s_0\leq s.
\end{split}\end{equation}
	Let $c,\gamma > 0$ be the constants in \eqref{eq:IU}.
	By \eqref{eq:kernMeanSemGroup}, we have
\begin{equation}\label{eq:assertion2}\begin{split}
	\phi(x) \langle v_s , \phi^* \rangle_m (1 - ce^{-\gamma t})
	&\leq S_t v_s(x) \\
	&\leq \phi(x) \langle v_s , \phi^* \rangle_m (1 + ce^{-\gamma t}),
	\quad s>0,t>1,x\in E.
\end{split}\end{equation}
	Notice that $\phi$ is strictly positive, using \eqref{eq:mean-fkpp}, one can verify that
\begin{equation}\label{eq:inequality-with-IU-assumption}\begin{split}
	\frac{V_t f(x)}{\phi(x)}
	\leq \frac{S_t f(x)}{\phi(x)}
	\leq (1+ ce^{-\gamma t}) \langle f,\phi^*\rangle,
	\quad f\in bp\mathscr B_E,x\in E,t>1.
\end{split}\end{equation}
	Taking $f = V_s(\theta \mathbf 1_E)$ in \eqref{eq:inequality-with-IU-assumption} and letting $\theta \to \infty$, by \eqref{eq:V_infty} we have that,
\begin{equation}\label{eq:assertion3}
	\frac{v_{t+s}(x)}{\phi(x)}
	\leq (1+ ce^{-\gamma t}) \langle v_s,\phi^*\rangle_m,
	\quad x\in E,s>0,t>1.
\end{equation}
	We can also verify that
\begin{equation}\label{eq:assertion4}
	S_t v_s(x)
	\leq \|\phi^{-1}v_s\|_\infty  S_t \phi(x)
	= \|\phi^{-1}v_s\|_\infty \phi(x)
	\quad s,t>0,x\in E.
\end{equation}
\par
	Now, we are ready to give the proof of this lemma using the following steps: $(1'')\Rightarrow (1')\Rightarrow (1)\Rightarrow (2)\Rightarrow (2')\Rightarrow (3)\Rightarrow (1'')$ and $(2)\Rightarrow (5)\Rightarrow (6)\Rightarrow (4)\Rightarrow (2)$.
	In fact, it is obvious that $(1'')\Rightarrow (1')\Rightarrow (1)$.
	For $(1)\Rightarrow (2)$ we use \eqref{eq:assertion2}.
	For $(2)\Rightarrow (2')$ we use \eqref{eq:assertion1}.
	For $(2')\Rightarrow (3)$ we use \eqref{eq:assertion3}.
	For $(3)\Rightarrow (1'')$ we use \eqref{eq:assertion4}.
\par
	For $(2)\Rightarrow (5)$, we follow the argument in \cite[Lemma 3.3]{RenSongZhang2015Limit}.
	Note that, from what we have proved, $(2)$ is equivalent to $(1),(1'),(1''),(2')$ and $(3)$.
	Integrating \eqref{eq:mean-fkpp} with respect to the measure $\nu$,
	by Fubini's theorem and monotonicity, we have that, for any $f\in p\mathscr B_E$ and $t\geq 0$,
\begin{equation}\label{eq:FKPP_mild}\begin{split}
	\langle f,\phi^*\rangle_m
	&=\langle f,S_t^*\phi^*\rangle_m
	=\langle S_tf,\phi^*\rangle_m\\
	&=\langle V_tf,\phi^*\rangle_m + \int_0^t\langle S_{t-r}\Psi_0(V_rf),\phi^*\rangle_m dr\\
	&=\langle V_tf,\phi^*\rangle_m + \int_0^t \langle \Psi_0(V_rf),\phi^*\rangle_mdr.
\end{split}\end{equation}
	Define
\[
 	v(x)
 	:= \lim_{t\to\infty} v_t(x)
 	= \lim_{t\to\infty}(-\log\mathbf P_{\delta_x}(X_t=\mathbf 0))
 	= -\log \mathbf P_{\delta_x}(\exists t>0, ~\text{s.t.}~ X_t=\mathbf 0).
 \]
	Since $v_t(x)=-\log\mathbf P_{\delta_x}(X_t=\mathbf 0)$ is non-increasing in $t$, and by $(3)$, we know that $v_t\in bp\mathscr B^\phi_E$ for $t$ large enough, we have $v\in bp\mathscr B^\phi_E\subset L^2(E,m)$.
	Taking $f=V_s(\theta \mathbf 1_E)$ in \eqref{eq:FKPP_mild} and letting $\theta\to\infty$, by monotonicity and $(2')$, we have that, there is an $s_0>0$ such that
\begin{equation}\label{eq:extinction-vt-phi}
	\int_0^t \langle \Psi_0(v_{r+s}),\phi^*\rangle_mdr
	=\langle v_s,\phi^*\rangle_m-\langle v_{t+s},\phi^*\rangle_m,
	\quad s\geq s_0, t\geq 0.
\end{equation}
	Letting $s\to\infty$, by monotonicity, we have
\[
	\int_0^t \langle \Psi_0(v), \phi^* \rangle_m dr
	= t \langle \Psi_0(v), \phi^* \rangle_m
	= \langle v, \phi^* \rangle_m - \langle v, \phi^* \rangle_m
	= 0.
\]
	Since $\phi^*$ is strictly positive on $E$, we must have $\Psi_0(v) = 0, m\text{-a.e.}$.
	This, with \eqref{eq:kernMeanSemGroup}, implies that $S_t\Psi_0 (v)\equiv 0$ for any $t>0$.
	By $(1'')$, we know that $S_t v_s(x)$ take finite value for $s$ large enough.
	Letting $s\to\infty$ in the \eqref{eq:reason-for-asp2'}, by monotonicity, we have
\[
	v(x)
	=S_tv(x)-\int_0^t S_{t-r}\Psi_0(v)(x)dr=S_tv(x),
	\quad x\in E, t\geq 0.
\]
	Since $v\in L^2(E,m)$, there is a constant $c\in \mathbb R$, such that $v(x) = c\phi(x)$ for all $x\in E$.
	So with $\Psi_0 (v) \equiv 0, m\text{-a.e.},$ we must have $v\equiv 0$.
	Using the fact that $v_t(x)$ converges to $0$ pointwisely, by monotonicity and \eqref{eq:assertion3}, we can verify the desired result $(5)$.
\par
	For $(5)\Rightarrow (6)$, note that, by \eqref{eq:linearity}, for any $\mu\in\mathcal M_f^\phi$, we have
\[\begin{split}
	&-\log\mathbf P_\mu[\exists t>0,\text{ s.t. }X_t=0]
	=\lim_{t\to \infty}(-\log\mathbf P_\mu(X_t = \mathbf 0))
	=\lim_{t\to\infty}\langle\mu, v_t\rangle
	= 0.
\end{split}\]
\par
	Finally, note that $(6)\Rightarrow (4)$ and $(4)\Rightarrow (2)$ are obvious.
\end{proof}
	As we mentioned earlier, in order to study the limit behavior of $(v_t)_{t>0}$, we assume that one of the assertions in Lemma \ref{lem:discuss-of-assumption2'} is true:
\begin{asp}
\label{asp:1'}
	In addition to Assumption \ref{asp:1}, we further assume that $\mathbf P_\nu (X_t = \mathbf 0) > 0$ for some $t>0$.
\end{asp}

\subsection{Kolmogorov type result}
\begin{thm}\label{thm:Kolmogorov-type-of-theorem}
	Under Assumptions \ref{asp:1'}, \ref{asp:2'} and \ref{asp:3}, we have
\[
	t\mathbf P_\mu(X_t\neq \mathbf 0)
	\xrightarrow[t\to\infty]{} \frac{\langle \mu,\phi\rangle} {\frac{1}{2}\langle  A \phi,\phi \phi^*\rangle_m},
%	\quad \mu\in \mathcal M^\phi_f,x\in E.
	\quad \mu\in \mathcal M^\phi_f.
\]
\end{thm}
	We first give some lemmas.
\begin{lem}\label{lem:Kolmogorov-1}
	Under Assumptions  \ref{asp:1'}, \ref{asp:2'} and \ref{asp:3}, we have
\[
	\sup_{x\in E}\Big | \frac{v_t(x)}{\langle v_t, \phi^*\rangle_m \phi(x)} - 1\Big |
	\xrightarrow[t\to\infty]{} 0.
\]
\end{lem}
\begin{proof}
	We use an argument similar to that used in \cite{Powell2015An-invariance} for critical branching diffusions.
	For each $\mu\in\mathcal M_f^\phi$, consider $\{Y, \mathbf n, \xi;\dot{\mathbf P}_\mu\}$, the spine immigration corresponding to $\{X;\mathbf P_\mu\}$.
	According to Theorem \ref{prop:sizBiasNMeas}, we have
\begin{equation}\label{eq:vt-and-Y}\begin{split}
	\langle \mu,\phi\rangle \dot {\mathbf P}_\mu [Y_t(\phi)^{-1}]
	&= \langle \mu,\phi\rangle \mathbb N_\mu^{w_t(\phi)} [ w_t(\phi)^{-1}] \\
	&= \mathbb N_{\mu}[\mathbf 1_{w_t(\phi)>0}]
	= -\log \mathbf P_{\mu} [ X_t = \mathbf 0] \\
	& = \langle \mu,v_t \rangle,\quad \mu\in\mathcal M_f^\phi,t > 0.
\end{split}\end{equation}
	Taking $\mu = \delta_x$ in \eqref{eq:vt-and-Y}, we get $v_t(x)/\phi(x)=\dot{\mathbf P}_{\delta_x}[Y_t(\phi)^{-1}]$.
	Taking $\mu = \nu$, we get $\langle v_t, \phi^*\rangle_m = \dot {\mathbf P}_{\nu} [Y_t(\phi)^{-1}]$.
	Therefore, to complete the proof, we only need to show that
\[
	\sup_{x\in E}\Big | \frac {\dot{\mathbf P}_{\delta_x}[Y_t(\phi)^{-1}]} {\dot {\mathbf P}_\nu [Y_t(\phi)^{-1}]}-1\Big|
	\xrightarrow[t\to\infty]{} 0.
\]
	For any Borel subset $G\subset (0,t]$, define
\[
	Y^G_t
	:= \int_{G\times \mathcal W} w_{t-s} \mathbf n(ds,dw).
\]
	Then we have the following decomposition of $Y$:
\begin{equation}\label{eq:decomposition-on-Y}
	Y_t
	= Y^{(0,t_0]}_t + Y^{(t_0,t]}_t,
	\quad 0 < t_0 < t < \infty.
\end{equation}
	It is easy to see, from the construction and the Markov property of the spine immigration $\{Y,\xi; \dot {\mathbf P}\}$, that for any $0 < t_0 < t < \infty$,
\[
	\dot{\mathbf P} [Y_t^{(t_0,t]}(\phi)^{-1}|\mathscr F^\xi_{t_0}]
	= \dot{\mathbf P}_{\delta_{\xi_{t_0}}}[Y_{t-t_0}(\phi)^{-1}]
	= (\phi^{-1}v_{t-t_0})(\xi_{t_0}).
\]
	Therefore, we have
\[\begin{split}
	\dot{\mathbf P}_\nu[Y_t^{(t_0,t]}(\phi)^{-1}]
	= \dot{\mathbb P}_{\nu\phi}[(\phi^{-1}v_{t-t_0})(\xi_{t_0}) ]
	= \langle v_{t-t_0},\phi^* \rangle_m
\end{split}\]
	and
\begin{equation}
\label{eq:Yt0t}
	\dot{\mathbf P}_{\delta_x}[Y_t^{(t_0,t]}(\phi)^{-1}]
	= \dot{\mathbb P}_x[(\phi^{-1}v_{t-t_0})(\xi_{t_0}) ]
	=  \int_E  \dot{q}(t_0,x,y)(\phi^{-1}v_{t-t_0})(y) m(dy).
\end{equation}
	By the decomposition \eqref{eq:decomposition-on-Y}, we have
\begin{equation}\label{eq:vt-equation}\begin{split}
	\phi^{-1}v_t(x)
	&= \dot {\mathbf P}_{\delta_x} [Y_t(\phi)^{-1}]\\
	&= \dot {\mathbf P}_\nu [Y^{(t_0,t]}_t(\phi)^{-1}] + \big( \dot {\mathbf P}_{\delta_x} [Y^{(t_0,t]}_t(\phi)^{-1}] - \dot {\mathbf P}_\nu [Y^{(t_0,t]}_t(\phi)^{-1}] \big) \\
	&\quad + \big( \dot{\mathbf P}_{\delta_x}[Y_t(\phi)^{-1} - Y^{(t_0,t]}_t(\phi)^{-1}] \big)\\
	&=: \langle v_{t-t_0},\phi^* \rangle_m + \epsilon_x^1(t_0,t) +\epsilon_x^2(t_0,t).
\end{split}\end{equation}
	Suppose that $t_0 >1$, and let $c,\gamma > 0$ be the constants in \eqref{eq:IU}, we have
\begin{equation}\label{eq:epsilon-1}\begin{split}
	|\epsilon_x^1(t_0,t)|
	& = \big| \dot {\mathbf P}_{\delta_x} [Y^{(t_0,t]}_t(\phi)^{-1}] - \dot {\mathbf P}_\nu [Y^{(t_0,t]}_t(\phi)^{-1}] \big| \\
	& = \big|  \int_E  \dot{q}(t_0,x,y)(\phi^{-1}v_{t-t_0})(y) m(dy) - \langle v_{t-t_0},\phi^* \rangle_m \big|\\
	& \leq \int_{y\in E} \big| \dot{q}(t_0,x,y) - (\phi\phi^*)(y) \big| (\phi^{-1}v_{t-t_0})(y) m(dy)\\
	& \leq ce^{-\gamma t_0}\langle v_{t-t_0},\phi^* \rangle_m .
\end{split}\end{equation}
	We also have
\begin{equation}\label{eq:epsilon-2}\begin{split}
	|\epsilon_x^2(t_0,t)|
	&= \big| \dot{\mathbf P}_{\delta_x}[Y_t(\phi)^{-1} - Y^{(t_0,t]}_t(\phi)^{-1}] \big| \\
	&= \dot{\mathbf P}_{\delta_x}[Y_t^{(0,t_0]}(\phi)\cdot Y_t(\phi)^{-1}\cdot Y^{(t_0,t]}_t(\phi)^{-1}]\\
	&\leq \dot{\mathbf P}_{\delta_x}[\mathbf 1_{Y_t^{(0,t_0]}(\phi)>0}\cdot Y^{(t_0,t]}_t(\phi)^{-1}]\\
	&= \dot{\mathbf P}_{\delta_x} \big[\dot{\mathbf P}_{\delta_x}[\mathbf 1_{Y_t^{(0,t_0]}(\phi)>0}|\mathscr F^\xi_{t_0}] \cdot \dot{\mathbf P}_{\delta_x}[ Y^{(t_0,t]}_t(\phi)^{-1}|\mathscr F^\xi_{t_0}] \big].
\end{split}\end{equation}
	Notice that, by Campbell's formula, one can verify that
\[
	\dot{\mathbf P}_{\delta_x}[e^{-\langle Y_t^{(0,t_0]},\theta \mathbf 1_E\rangle}|\mathscr F^\xi_{t_0}]
	= e^{-\int_0^{t_0}\Psi'_0(V_{t-s}(\theta\mathbf 1_E))(\xi_s)ds}.
\]
	Letting $\theta \to \infty$ we have
\[
	\dot {\mathbf P}_{\delta_x} [ \mathbf 1_{Y_t^{(0,t_0]}=\mathbf 0} | \mathscr F^\xi_{t_0}]
	= e^{-\int_0^{t_0}\Psi'_0(v_{t-s})(\xi_s)ds}.
\]
	We also have
\[\begin{split}
	\Psi'_0(v_{t-s})(x)
	&= 2\alpha(x)v_{t-s}(x) +\int_{(0,\infty)} (1-e^{-yv_{t-s}(x)})y\pi(x,dy)\\
	&\leq \big( 2\alpha (x)+\int_{(0,\infty)}y^2\pi(x,dy) \big) v_{t-s}(x)\\
	&= A v_{t-s}(x) \leq \| A\phi\|_\infty \|\phi^{-1}v_{t-s}\|_\infty.
\end{split}\]
	Therefore
\begin{equation}\label{eq:firstpart-of-Y}
	\dot{\mathbf P}_{\delta_x}[\mathbf 1_{Y_t^{(0,t_0]}\neq \mathbf 0}|\mathscr F^\xi_{t_0}]
	= 1-e^{-\int_0^{t_0}\Psi'_0(v_{t-s})(\xi_s)ds}
	\leq t_0\| A\phi\|_\infty \|\phi^{-1}v_{t-t_0}\|_\infty.
\end{equation}
	Plugging \eqref{eq:firstpart-of-Y} into \eqref{eq:epsilon-2}, using \eqref{eq:Yt0t} and letting $c,\gamma > 0$ be the constants in \eqref{eq:IU}, we have that
\begin{equation}\label{eq:epsilon-2-final}\begin{split}
	|\epsilon_x^2(t_0,t)|
	& \leq t_0\| A\phi\|_\infty \|(\phi^{-1}v_{t-t_0})\|_\infty \dot{\mathbf P}_{\delta_x}[ Y^{(t_0,t]}_t(\phi)^{-1}|\mathscr F^\xi_{t_0}] \\
	& \leq t_0\|  A \phi\|_\infty\|(\phi^{-1}v_{t-t_0}) \|_\infty \int_{E} \dot{q} (t_0,x,y)(\phi^{-1}v_{t-t_0})(y) m(dy)\\
	& \leq t_0\| A\phi\|_\infty \| \phi^{-1}v_{t-t_0}\|_\infty (1+ce^{-\gamma t_0}) \langle v_{t-t_0},\phi^* \rangle_m.
\end{split}\end{equation}
	Combining \eqref{eq:vt-equation}, \eqref{eq:epsilon-1} and \eqref{eq:epsilon-2-final}, we have that
\begin{equation}\label{vts-inequality}\begin{split}
	\Big|\frac{\phi^{-1}v_t(x)}{\langle v_{t-t_0},\phi^* \rangle_m}-1 \Big|
	&\leq \frac{|\epsilon_x^1(t_0,t)|}{\langle v_{t-t_0},\phi^* \rangle_m} + \frac{|\epsilon_x^2(t_0,t)|}{\langle v_{t-t_0},\phi^* \rangle_m}\\
	&\leq ce^{-\gamma t_0} +t_0\| A\phi\|_\infty \| \phi^{-1}v_{t-t_0}\|_\infty (1+ce^{-\gamma t_0}).
\end{split}\end{equation}
	Since we know from
	Lemma \ref{lem:discuss-of-assumption2'}(5)
	that $\| \phi^{-1}v_t\|_\infty\to 0$ when $t\to\infty$, there exists a map $t\mapsto t_0(t)$ such that,
\[
	t_0(t)
	\xrightarrow[t\to\infty]{} \infty;
	\quad t_0(t)\| \phi^{-1}v_{t-t_0(t)}\|_\infty
	\xrightarrow[t\to\infty]{} 0.
\]
	Plugging this choice of $t_0(t)$ back into \eqref{vts-inequality}, we have that
\begin{equation}\label{eq:k1}
	\sup_{x\in E}\Big|\frac{\phi^{-1}v_t(x)}{\langle v_{t-t_0(t)},\phi^* \rangle_m}-1 \Big|
	\xrightarrow[t\to\infty]{} 0.
\end{equation}
	Now notice that
\begin{equation}\label{eq:k2}\begin{split}
	\Big |\frac {\langle v_t, \phi^*\rangle_m} {\langle v_{t-t_0(t)} , \phi^*\rangle_m} - 1 \Big |
	&\leq \int \Big | \frac{\phi^{-1}v_t(x)}{\langle v_{t-t_0(t)} , \phi^*\rangle} - 1 \Big| \phi \phi^*(x) m(dx)\\
	&\leq \sup_{x\in E}\Big|\frac{\phi^{-1}v_t(x)}{\langle v_{t-t_0(t)},\phi^* \rangle_m}-1 \Big|
	\xrightarrow[t\to\infty]{} 0.
\end{split}\end{equation}
	Finally, by \eqref{eq:k1}, \eqref{eq:k2} and the property of uniform convergence,
\[
	\sup_{x\in E}\Big|\frac{\phi^{-1}v_t(x)}{\langle v_{t},\phi^* \rangle_m}-1 \Big|
	\xrightarrow[t\to\infty]{} 0,
\]
	as desired.
\end{proof}
\begin{lem}\label{lem:Kolmogorov-2}
	Under Assumptions \ref{asp:1'}, \ref{asp:2'} and \ref{asp:3}, we have
\[
	\frac{1}{t\langle v_t,\phi^*\rangle_m}
	\xrightarrow[t\to\infty]{} \frac{1}{2}\langle  A\phi,\phi\phi^*\rangle_m.
\]
\end{lem}
\begin{proof}
	We use an argument similar to that used in \cite{Powell2015An-invariance} for critical branching diffusions.
	According to \cite{RenSongZhang2015Limit}, we have that, for any $x\in E$ and $z\geq 0$,
\[\begin{split}
	R(x,z)
	:=\Psi_0(x,z)-\frac{1}{2} A(x)z^2
	\leq e(x,z)z^2,
\end{split}\]
	where
\[
	e(x,z)
	:=\int_{(0,\infty)}y^2\big(1\wedge \frac{1}{6}yz\big)\pi(x,dy)
	\leq  A(x).	
\]
	By monotonicity, we have that
\begin{equation}\label{equation:reason2}
	e(x,z)
	\xrightarrow[z\to 0]{} 0,
	\quad x\in E.
\end{equation}
	Taking $b(t):=\langle v_t,\phi^*\rangle_m$ and writing $l_t(x):=v_t(x)-b(t)\phi(x)$, Lemma \ref{lem:Kolmogorov-1} says that,
\begin{equation}\label{equation:reason1}
	\sup_{x\in E}\Big|\frac{l_t(x)}{b(t)\phi(x)}\Big|
	\xrightarrow[t\to\infty]{} 0.
\end{equation}
	Now, taking $s_0>0$ as in \eqref{eq:extinction-vt-phi},
	we have that $t\mapsto b(t)$ is differentiable on the set
\[
	C
	=\{t> s_0: \text{the function}~ t \mapsto \langle\Psi_0(v_t),\phi^*\rangle_m~ \text{is continuous at}~ t \}
\]
	and that
\begin{equation}\label{equation:b(t)}\begin{split}
	\frac{d}{dt}b(t)
	&= -\langle\Psi_0(v_t),\phi^*\rangle_m
	= -\big\langle \frac{1}{2} A \cdot v_t^2+R (v_t),\phi^*\big\rangle_m \\
	&= -\big\langle \frac{1}{2}  A \cdot \big(b(t)\phi+l_t\big)^2+R (v_t),\phi^*\big\rangle_m\\
	&= -b(t)^2\big[\frac{1}{2} \langle  A\phi,\phi \phi^*\rangle_m+g(t)\big],
	\quad t\in C,
\end{split}\end{equation}	
	where
\[\begin{split}
	g(t)
	&= \Big\langle \frac{l_t}{b(t) \phi}, A\phi^2\phi^*\Big\rangle_m + \frac{1}{2}\Big\langle \Big(\frac{l_t}{b(t) \phi}\Big)^2, A\phi^2\phi^*\Big\rangle_m + \Big\langle \frac{R(v_t)}{b(t)^2 \phi^2},\phi^2\phi^*\Big\rangle_m \\
	&=: g_1(t) + g_2(t) + g_3(t).
\end{split}\]
	From \eqref{equation:reason1}, we have $g_1(t)\to 0$ and $g_2(t)\to 0$ as $t\to\infty$.
	From
\[\begin{split}
	\frac{R(v_t)(x)}{b(t)^2 \phi(x)^2}
	\leq \frac{e(v_t)(x)\cdot v_t(x)^2}{ b(t)^2\phi(x)^2}
	= e(v_t)(x)\Big(1+\frac{l_t(x)}{ b(t) \phi(x)}\Big)^2,
\end{split}\]
	using \eqref{equation:reason1}, \eqref{equation:reason2}, Lemma \ref{lem:discuss-of-assumption2'}(5) and the dominated convergence theorem ($e(v_t)$ is dominated by $ A$), we conclude that $g_3(t)\to 0$ as $t\to\infty$.
\par
	Finally, from \eqref{equation:b(t)} we can write
\begin{equation}\label{eq:derivative-of-b(t)-1}
	\frac{d}{dt} \Big(\frac{1}{b(t)}\Big)
	= -\frac{d b(t)}{b(t)^2dt}
	= \frac{1}{2}\langle  A\phi,\phi\phi^*\rangle_m + g(t),
	\quad t\in C.
\end{equation}
	Notice that, since the function $t\mapsto \langle\Psi_0(v_t),\phi^*\rangle_m$ is non-increasing in $t$, $C^c$ has at most countably many elements.
	Therefore, using \eqref{eq:extinction-vt-phi} and \eqref{eq:derivative-of-b(t)-1}, one can verify that $t\mapsto \frac{1}{b(t)}$ is absolutely continuous on the interval $[s_0,t_0]$ as long as $s_0$ and $t_0$ are large enough.
	This allows us to integrate \eqref{eq:derivative-of-b(t)-1} on the interval $[s_0,t_0]$ with respect to the Lebesgue measure, and get that
\[
	\frac{1}{b(t_0)}
	= \frac{1}{b(s_0)} + \frac{1}{2}\langle A\phi,\phi\phi^*\rangle_m(t_0-s_0) + \int_{s_0}^{t_0} g(s)ds,
	\quad \text{for } 0\leq s_0\leq t_0 \text{ large enough}.
\]
	Dividing by $t_0$ and letting $t_0\to\infty$ in the above equation, we have
\[
	\frac{1}{b(t)t}
	\xrightarrow[t\to\infty]{} \frac{1}{2}\langle  A\phi,\phi\phi^*\rangle_m
\]
	as desired.
\end{proof}
\begin{proof}[Proof of Theorem \ref{thm:Kolmogorov-type-of-theorem}]
	For $\mu \in \mathcal M^\phi_f$, from Lemma \ref{lem:discuss-of-assumption2'}(5) we know that
\begin{equation}\label{eq:kol-1}
	\langle \mu ,v_t\rangle
    = \int_E v_t(x) \mu(dx)
	= \int_E \frac{v_t(x)}{\phi(x)} \phi(x)\mu(dx)
	\xrightarrow[t\to\infty]{} 0.
\end{equation}
	From Lemma \ref{lem:Kolmogorov-1} we know that
\begin{equation}\label{eq:kol-2}
     \frac {\langle\mu, v_t\rangle}{ \langle v_t,\phi^*\rangle_m}
	= \int_E \frac{v_t(x)}{\langle v_t, \phi^* \rangle_m \phi(x)}\phi(x)\mu(dx)
	\xrightarrow[t\to\infty]{} \langle \mu,\phi\rangle.
\end{equation}
	It then follows from \eqref{eq:kol-1}, \eqref{eq:kol-2} and Lemma \ref{lem:Kolmogorov-2} that
\[\begin{split}
	t\mathbf P_{\mu}(X_t\neq \mathbf 0)
	&= t (1-e^{-\langle \mu, v_t \rangle})
	= t \langle v_t,\phi^*\rangle\frac{\langle \mu,v_t\rangle }{\langle v_t,\phi^*\rangle_m} \frac {1-e^{-\langle \mu, v_t \rangle}} {\langle \mu, v_t \rangle}\\
	&\xrightarrow[t\to\infty]{} \frac{\langle \mu,\phi\rangle} {\frac{1}{2}\langle  A \phi,\phi \phi^*\rangle_m},
	\quad x\in E.
\end{split}\]
\end{proof}

\subsection{Yaglom type result}
	In this section we give the proof of the following Yaglom type result:
\begin{thm}\label{thm:Yaglom-type-theorem}
	Suppose that Assumptions  \ref{asp:1'}, \ref{asp:2'} and \ref{asp:3} hold.
	If $f\in bp\mathscr B^\phi_E$ and $\mu\in\mathcal M^\phi_f$, then we have
\[\begin{split}
	\big\{t^{-1}X_t(f);\mathbf P_\mu(\cdot | X_t\neq \mathbf 0)\big\}
	\xrightarrow[t\to\infty]{law} \frac{1}{2}\langle \phi^*, f\rangle_m\langle \phi A, \phi\phi^*\rangle_m \mathbf e,
\end{split}\]
	where $\mathbf e$ is an exponential random variable with mean 1.
\end{thm}
	We will need the following lemmas:
\begin{lem}
\label{lem:ergodic}
	Suppose that Assumption \ref{asp:2'} holds. Suppose that $F(x,u,t)$ is a bounded Borel function on $E\times [0,1]\times [0,\infty)$ such that $F(x,u):= \lim_{t\to\infty} F(x,u,t)$ exists for each $x\in E$ and $u\in [0,1]$.
	Then we have,
\[
	\int_0^1 F(\xi_{ut},u,t)du
	\xrightarrow[t\to\infty]{L^2(\dot{\mathbb P}_x)} \int_0^1 \langle F(\cdot ,u),\phi\phi^* \rangle_m du,
	\quad x\in E.
\]
\end{lem}
\begin{proof}
\par
	We first show that
\begin{equation}\label{eq:ergodic-lemma-1}
	\dot{\mathbb P}_x[F(\xi_{ut},u,t)]
	\xrightarrow[t\to\infty]{} \langle F(\cdot,u),\phi\phi^*\rangle_m,
      \quad x\in E, u\in(0,1).
\end{equation}
	In fact,
\[
	\dot{\mathbb P}_x[F(\xi_{ut},u,t)]
	=\int_E \frac{\dot{q}(ut,x,y)}{(\phi\phi^*)(y)}F(y,u,t)(\phi\phi^*)(y)m(dy).
\]
	Note that $\int_\cdot (\phi\phi^*)(y)m(dy)$ is a finite measure, $(y,t)\mapsto \frac{\dot{q}(ut,x,y)}{(\phi\phi^*)(y)}F(y,u,t)$ is bounded by $(1+ce^{-\gamma ut})\|F\|_\infty$ for $t>u^{-1}$, and $\frac{\dot{q}(ut,x,y)}{(\phi\phi^*)(y)}F(y,u,t)\xrightarrow[t\to\infty]{}F(y,u)$.
	Using the bounded convergence theorem, we get \eqref{eq:ergodic-lemma-1}.
	By Fubini's theorem,
\[
	\dot{\mathbb P}_x\big[\int_0^1F(\xi_{ut},u,t)du\big]
	=\int_0^1\dot{\mathbb P}_x[F(\xi_{ut},u,t)]du,
	\quad x\in E.
\]
	Since $\dot{\mathbb P}_x[F(\xi_{ut},u,t)]$ is bounded by $\|F\|_\infty$ and $\dot{\mathbb P}_x[F(\xi_{ut},u,t)]\xrightarrow[t\to\infty]{}\langle F(\cdot, u),\phi\phi^*\rangle_m$, by the bounded convergence theorem, we get
\[
	\dot{\mathbb P}_x\big[\int_0^1F(\xi_{ut},u,t)du\big]
	\xrightarrow[t\to\infty]{} c_F
	:=\int_0^1\langle F(\cdot,u),\phi\phi^*\rangle_mdu.
\]
	With a similar argument, using \eqref{eq:IU}, one can verify that for any $0< u< v\leq 1$,
\[\begin{split}
	&\dot{\mathbb P}_x[F(\xi_{ut},u,t)F(\xi_{vt},v,t)]\\
	&\quad=\int_E\int_E  \dot{q}(ut,x,y) \dot{q}((v-u)t,y,z)F(y,u,t)F(z,v,t)m(dy)m(dz)\\
	&\quad\xrightarrow[t\to\infty]{} \langle F(\cdot,u),\phi\phi^*\rangle_m \langle F(\cdot,v),\phi\phi^*\rangle_m.
\end{split}\]
	The above convergence is also true for $0< v < u\leq 1$ since the limit is symmetric in $u$ and $v$.
	We have again, by Fubini's theorem and the bounded convergence theorem,
\[\begin{split}
	\dot{\mathbb P}_x\Big[\big(\int_0^1 F(\xi_{ut},u,t)du\big)^2\Big]
	&=\int_0^1du\int_0^1\dot{\mathbb P}_x[F(\xi_{ut},u,t)F(\xi_{vt},v,t)]dv\xrightarrow[t\to\infty]{} c_F^2.
\end{split}\]
	Finally, we have
\[\begin{split}
	&\dot{\mathbb P}_x\Big[\big(\int_0^1 F(\xi_{ut},u,t)du - c_F\big)^2\Big]\\
	&\quad =\dot{\mathbb P}_x\Big[\big(\int_0^1 F(\xi_{ut},u,t)du\big)^2\Big] -2c_F \dot{\mathbb P}_x\Big[\int_0^1 F(\xi_{ut},u,t)du\Big]+ c_F^2\\
	&\quad \xrightarrow[t\to\infty]{} 0,
\end{split}\]
	as desired.
\end{proof}
\begin{prop}\label{prop:yaglTheorSpinImmigr}
	Suppose that Assumptions \ref{asp:1}, \ref{asp:2'}  and \ref{asp:3} hold.
	Let $\{Y,\xi,\mathbf n;\dot{\mathbf P}\}$ be the corresponding spine immigration.
	Then
\[
	\dot{\mathbf P}_\mu[e^{-\theta W_t}]
	\xrightarrow[t\to\infty]{}\frac{1}{(1 + c_0 \theta)^2},
	\quad \theta \geq 0,\mu\in \mathcal M^\phi_f,
\]
	where $W_t:= \frac{Y_t(\phi)}{t}$ and $c_0 := \frac {1} {2} \langle \phi  A, \phi \phi^* \rangle_m$.
\end{prop}
\begin{proof}
	We first prove the result when $\mu = \delta_x$.
	To simplify  notation, for any $x\in E,\theta\geq 0$ and $t\geq 0$, we write
\[\begin{split}
	J(x,\theta,t)
	&:=\phi A(x)\dot{\mathbf P}_{\delta_{x}}[e^{-\theta W_t}]\widetilde{\mathbf P}_{x}[e^{-X_t(\frac{\theta\phi}{t})}],\\
	J_0(x,\theta,t)
	&:=\phi A(x)\dot{\mathbf P}_{\delta_x}[e^{-\theta W_t}]
\end{split}\]
	and
\[\begin{split}
	M(x,\theta,t)
	&:=\Big|\frac{1}{(1+c_0\theta)^2}-\dot{\mathbf P}_{\delta_x}[e^{-\theta W_t}]\Big|.
\end{split}\]
	\emph{Step 1.  We will show that}
\begin{equation}\label{eq:yaglTheorSpinImmigrStep1}
	\dot{\mathbf P}_{\delta_x}[e^{-\theta W_t}]
	=\dot{\mathbf P}_{\delta_x}[ e^{-\int_0^1 du\int_0^\theta d\rho\cdot J(\xi_{ut},\rho(1-u),t(1-u))} ].
\end{equation}
	In fact, we have
\[
	\frac{\partial}{\partial \theta}\dot{\mathbf P}_{\delta_x}[e^{-\theta W_t}|\xi]
	= -\dot{\mathbf P}_{\delta_x}[W_te^{-\theta W_t}|\xi],
	\quad t\geq 0,\theta \geq 0.
\]
	Applying Lemma \ref{lem:key-lemma} with $K(dr)=\delta_t(dr)$ and $f_t=\frac{\theta\phi}{t}$, for each $\theta \geq 0$, we have
\[\begin{split}
	-\frac{\partial}{\partial \theta}\log \dot{\mathbf P}_{\delta_x}[e^{-\theta W_t}|\xi]
	&=\frac{\dot{\mathbf P}_{\delta_x} [W_t e^{-\theta W_t}|\xi]}{\dot{\mathbf P}_{\delta_x}[e^{-\theta W_t}|\xi]}\\
	&=\frac{1}{t}\int_0^t  (A\phi)(\xi_s)\dot{\mathbf P}_{\delta_{\xi_s}}[e^{-(\theta \frac{t-s}{t})W_{t-s}}]\widetilde{\mathbf P}_{\xi_s}[e^{-X_{t-s}(\frac{\theta\phi}{t})}]ds\\
	&=\int_0^1 J(\xi_{ut},\theta(1-u),t(1-u)) du.
\end{split}\]
	Integrating both sides of the above equation yields that
\[
	-\log \dot{\mathbf P}_{\delta_x}[e^{-\theta W_t}|\xi]
	=\int_0^1 du\int_0^\theta J(\xi_{ut},\rho(1-u),t(1-u)) d\rho,
\]
	which implies \eqref{eq:yaglTheorSpinImmigrStep1}.
\par
	\emph{Step 2. We will show that}
\begin{equation}\label{eq:yaglTheorSpinImmigrStep2}
	\int_0^1 du\int_0^\theta (J_0-J)(\xi_{ut},\rho(1-u),t(1-u)) d\rho
	\xrightarrow[t\to\infty]{L^2(\dot{\mathbf P}_{\delta_x})} 0,\quad \theta\geq 0.
\end{equation}
	To get this result, we will apply Lemma \ref{lem:ergodic} with
\begin{equation}\label{eq:F}\begin{split}
	F(x,u,t)
	&:=\int_0^\theta d\rho\cdot (J_0-J)(x,\rho(1-u),t(1-u))\\
	&=\int_0^\theta d\rho\cdot  (A\phi)(x)\dot{\mathbf P}_{\delta_{x}}[e^{-\rho(1-u)W_{t(1-u)}}]\widetilde{\mathbf P}_{x}[1-e^{-X_{t(1-u)}(\frac{\rho\phi}{t})}].
\end{split}\end{equation}
	Firstly note that $F(x,u,t)$ is bounded by $\theta\|\phi A\|_\infty$ on $E\times[0,1]\times[0,\infty)$.
	Secondly note that $F(x,u,t)\xrightarrow[t\to\infty]{} 0$ for each $x\in E$ and $u\in[0,1]$, since $|J_0-J|$ is bounded by $\|\phi A\|_\infty$ and
\[\begin{split}
	\big|(J_0-J)(x,\theta,t)\big|
	&=  (A\phi)(x)\dot{\mathbf P}_{\delta_{x}}[e^{-\theta W_{t}}]\widetilde{\mathbf P}_{x}[1-e^{-X_t(\frac{\theta\phi}{t})}]\\
	&\leq  (A\phi)(x)\widetilde{\mathbf P}_{x}(X_t\neq \mathbf 0) \\
	&=  (A\phi)(x)\frac{2\alpha(x)\mathbf P_{\mathbf 0}(X_t\neq \mathbf 0)+\int_{(0,\infty)}y^2\mathbf P_{y\delta_x}(X_t\neq \mathbf 0)\pi(x,dy)}{2\alpha(x)+\int_{(0,\infty)}y^2\pi(x,dy)}\\
	&\xrightarrow[t\to\infty]{} 0, \quad x\in E,\theta\geq 0.
\end{split}\]
	Therefore, we can apply Lemma \ref{lem:ergodic} with $F(x,u,t)$ given by \eqref{eq:F}, and get \eqref{eq:yaglTheorSpinImmigrStep2}.
\par
	\emph{Step 3. We will show that}
\begin{equation}\label{eq:yaglTheorSpinImmigrStep3}
	\frac{1}{(1+c_0\theta)^2}
	= \lim_{t\to\infty} \dot {\mathbf P}_{\delta_x} \Big[e^{- \int_0^1 du \int_0^\theta d\rho \frac{ (A\phi)(\xi_{ut})}{(1+c_0\rho(1-u))^2} }\Big],
	\quad \theta\geq 0.
\end{equation}
	By elementary calculus, the following map
\[
	(x,u)
	\mapsto\int_0^\theta \frac{ (A\phi)(x)}{(1+c_0\rho(1-u))^2} d\rho
	= \frac{ (A\phi)(x)\theta}{1+c_0\theta(1-u)}
\]
	is bounded by $\theta\| A\phi\|_\infty$ on $E\times[0,1]$.
	According to Lemma \ref{lem:ergodic}, we have that
\[\begin{split}
	\int_0^1 du \int_0^\theta \frac{ (A\phi)(\xi_{ut})}{\big(1+c_0\rho(1-u)\big)^2} d\rho
	&\xrightarrow[t\to\infty]{L^2(\dot{\mathbf P}_{\delta_x})} \int_0^1 \big\langle \frac{\theta A\phi}{1+c_0\theta(1-u)},\phi\phi^* \big\rangle_m du\\
	&= \langle  A\phi,\phi\phi^*\rangle_m\int_0^1 \frac{\theta}{1+c_0\theta(1-u)}du \\
	&=2\log(1+c_0\theta).
\end{split}\]
	Therefore, by the bounded convergence theorem, we get \eqref{eq:yaglTheorSpinImmigrStep3}.
\par
	\emph{Step 4. We will show that}
\begin{equation}\label{eq:yaglTheorSpinImmigrStep4}
	M(x,\theta)
	:=\limsup_{t\to\infty}M(x,\theta,t)=0,
	\quad x\in E,\theta\geq 0.
\end{equation}
	In fact,
\begin{equation}\label{eq:separate-M-into-3-parts}
	M(x,\theta,t)
	\leq I_1+I_2+I_3,
\end{equation}
	where
\[
	I_1
	:=\Big|\frac{1}{(1+c_0\theta)^2}- \dot {\mathbf P}_{\delta_x} \big[e^{- \int_0^1 du \int_0^\theta \frac{ (A\phi)(\xi_{ut})}{[1+c_0\rho(1-u)]^2}d\rho }\big]\Big|
	\xrightarrow[t\to\infty]{\text{by \eqref{eq:yaglTheorSpinImmigrStep3}}} 0,
\]
\[\begin{split}
	I_2
	&:=\Big| \dot {\mathbf P}_{\delta_x} [e^{- \int_0^1 du \int_0^\theta  \frac{ (A\phi)(\xi_{ut})}{(1+c_0\rho(1-u))^2} d\rho}]-\dot{\mathbf P}_{\delta_x}[e^{-\int_0^1 du\int_0^{\theta} J_0(\xi_{ut},\rho(1-u),t(1-u)) d\rho}]\Big|\\
	&\leq\dot{\mathbf P}_{\delta_x}\Big[\int_0^1du\int_0^\theta (A\phi)(\xi_{ut})M(\xi_{ut},\rho(1-u),t(1-u))d\rho \Big]\\
	&=\int_0^1du\int_0^\theta d\rho \int_{E} \dot{q}(ut,x,y) (A\phi)(y)M(y,\rho(1-u),t(1-u)) m(dy),
\end{split}\]
	and by \eqref{eq:yaglTheorSpinImmigrStep1} and \eqref{eq:yaglTheorSpinImmigrStep2},
\[\begin{split}
	I_3
	&:=\big|\dot{\mathbf P}_{\delta_x}[e^{-\int_0^1 du\int_0^{\theta} J_0(\xi_{ut},\rho(1-u),t(1-u)) d\rho }]-\dot{\mathbf P}_{\delta_x}[e^{-\theta W_t}]\big|\\
	&=\big|\dot{\mathbf P}_{\delta_x}[e^{-\int_0^1 du\int_0^{\theta} J_0(\xi_{ut},\rho(1-u),t(1-u)) d\rho }]-\dot{\mathbf P}_{\delta_x}[e^{-\int_0^1 du\int_0^{\theta} J(\xi_{ut},\rho(1-u),t(1-u)) d\rho }]\big|\\
	&\leq \dot{\mathbf P}_{\delta_x}\Big[\Big|\int_0^1du\int_0^\theta (J_0-J)(\xi_{ut},\rho(1-u),t(1-u))d\rho\Big|\Big]
	\xrightarrow[t\to\infty]{} 0.
\end{split}\]
	Therefore, taking $\limsup_{t\to\infty}$ in \eqref{eq:separate-M-into-3-parts}, by the reverse Fatou lemma, we get
\begin{equation}\label{eq:the-key-inequiality}
	M(x,\theta)
	\leq \int_0^1du\int_0^\theta \langle  A\phi M(\cdot,\rho(1-u)),\phi\phi^*\rangle_m d\rho,
	\quad x\in E,\theta\geq 0.
\end{equation}
	Integrating with respect to the finite measure $(A\phi\phi\phi^*)(x)m(dx)$ yields that
\[
	\langle  A\phi M(\cdot,\theta),\phi\phi^*\rangle_m
	\leq \langle A\phi,\phi\phi^*\rangle_m\int_0^1du\int_0^\theta \langle  A\phi M(\cdot,\rho(1-u)),\phi\phi^*\rangle_m d\rho,
	\quad \theta\geq 0.
\]
	According to \cite[Lemma 3.1]{RenSongSun2018A-2-spine}, this inequality implies that $\langle A\phi M(\cdot,\theta),$ $\phi\phi^*\rangle_m$ $ =0$ for each $ \theta\geq 0$.
	This and \eqref{eq:the-key-inequiality} imply \eqref{eq:yaglTheorSpinImmigrStep4}, which completes the proof when $\mu=\delta_x$.
\par
	Finally, for any $\mu \in \mathcal M^\phi_f$, since
\[\begin{split}
	\langle \mu,\phi\rangle\dot {\mathbf P}_\mu [e^{-\theta W_t}]
	&=\langle\mu,\phi\rangle\mathbb N^{w_t(\phi)}_\mu[e^{-\theta \frac{w_t(\phi)}{t}}]
	=\mathbb N_\mu [w_t(\phi)e^{-\theta \frac{w_t(\phi)}{t}}]\\
	&=\int_E\mu(dx)\mathbb N_x[w_t(\phi)e^{-\theta\frac{w_t(\phi)}{t}}]
	=\int_E \mu(dx) \phi(x) \dot{\mathbf P}_{\delta_x}[e^{-\theta W_t}],
\end{split}\]
	we have that, by the bounded convergence theorem,
\[
	\big|\dot {\mathbf P}_\mu[e^{-\theta W_t}] - \frac{1}{(1+c_0\theta)^2}\big|
	\leq \int_E \big|\dot {\mathbf P}_{\delta_x}[e^{-\theta W_t}]-\frac{1}{(1+c_0\theta)^2}\big| \frac{\phi(x)\mu(dx)}{\langle\mu,\phi\rangle}
	\xrightarrow[t\to\infty]{} 0,
\]
	as desired.	
\end{proof}
\begin{lem}\label{lem:Yaglom-for-phi}
	Suppose that Assumptions \ref{asp:1'}, \ref{asp:2'} and \ref{asp:3} hold.
	Then we have
\[\begin{split}
	\mathbf P_\mu[e^{-\theta t^{-1}X_t(\phi)} |X_t\neq\mathbf 0]
	&\xrightarrow[t\to\infty]{}\frac{1}{1+c_0\theta},
	\quad \theta\geq 0, \mu\in \mathcal M_f^\phi,
\end{split}\]
	where $c_0 = \frac{1}{2}\langle\phi A,\phi\phi^*\rangle_m$.
\end{lem}
\begin{proof}
	Notice that, by Lemma \ref{lem:discuss-of-assumption2'}(6), we have
\[
	\{t^{-1}X_t(\phi);\mathbf P_\mu\}
	\xrightarrow[t\to\infty]{law} 0.
\]
	Therefore, by Theorem \ref{thm:spinDec} and Proposition \ref{prop:yaglTheorSpinImmigr}, we have
\[
	\mathbf P^M_\mu[e^{-\theta t^{-1}X_t(\phi)}]
	= (\mathbf P_\mu \otimes \dot {\mathbf P}_\mu)[e^{-\theta t^{-1}(X_t+Y_t)(\phi)}]
	\xrightarrow[t\to\infty]{}\frac{1}{(1+c_0\theta)^2}.
\]
	Also notice that, by elementary calculus
\[
	\frac{1-e^{-\theta u}}{u}
	=\int_0^\theta e^{-\rho u} d\rho,
	\quad u> 0.
\]
	From Theorem \ref{thm:spinDec} and Lemma \ref{lem:Y-is-immortal} we know that $\mathbf P^M_\mu(X_t=\mathbf 0)=0$.
	Therefore by the bounded convergence theorem, we have
\[\begin{split}
	\mathbf P^M_\mu \Big[\frac{1-e^{-\theta t^{-1}X_t(\phi)}}{t^{-1}X_t(\phi)}\Big]
	&= \mathbf P^M_\mu\Big[\int_0^\theta e^{-\rho t^{-1}X_t(\phi) }d\rho\Big]
	= \int_0^\theta \mathbf P^M_\mu [ e^{-\rho t^{-1}X_t(\phi) }] d\rho\\
	&\xrightarrow[t\to\infty]{} \int_0^\theta \frac{1}{(1+c_0\rho)^2}d\rho
	= c_0^{-1} (1-\frac{1}{1+c_0\theta}).
\end{split}\]
	Hence, by Theorem \ref{thm:Kolmogorov-type-of-theorem} we have
\[\begin{split}
	\mathbf P_\mu[1-e^{-\theta t^{-1}X_t(\phi)} |X_t\neq \mathbf 0]
	&= \mathbf P_\mu(X_t\neq \mathbf 0)^{-1}\mathbf P_\mu[(1-e^{-\theta t^{-1}X_t(\phi)})\mathbf 1_{X\neq \mathbf 0}]\\
	&= \mathbf P_\mu(X_t\neq \mathbf 0)^{-1}\mathbf P_\mu\Big[(1-e^{-\theta t^{-1}X_t(\phi)})\frac{X_t(\phi)}{X_t(\phi)}\Big]\\
	&= (t\mathbf P_\mu(X_t\neq \mathbf 0))^{-1} \langle\mu,\phi\rangle \mathbf P^M_\mu \Big[\frac{1-e^{-\theta t^{-1}X_t(\phi)}}{t^{-1}X_t(\phi)}\Big]\\
	&\xrightarrow[t\to\infty]{}1-\frac{1}{1+c_0\theta},
\end{split}\]
	which completes the proof.
\end{proof}
\begin{lem}\label{lem:general-lemma}
	Suppose that Assumptions \ref{asp:1'}, \ref{asp:2'} and \ref{asp:3} hold.
	If $\tilde f\in b\mathscr B^\phi_E$ satisfies $\langle \tilde f, \phi^*\rangle = 0$, then we have, for any $\mu \in \mathcal M^\phi_f$,
\[\begin{split}
	\big\{ t^{-1} X_t(\tilde f) ; \mathbf P_\mu(\cdot|X_t \neq \mathbf 0)\big\}
	&\xrightarrow[t\to\infty]{} 0,
	\quad\text{in probability}.
\end{split}\]
\end{lem}
\begin{proof}
	If we can show that $\mathbf P_\mu\big[\big(t^{-1}X_t(\tilde f)\big)^2|X_t \neq \mathbf 0\big]\xrightarrow[t\to\infty]{} 0$, then the desired result follows by the Chebyshev's inequality
\[
	\mathbf P_\mu \big( | t^{-1} X_t(\tilde f) | \geq \epsilon \big | X_t \neq \mathbf 0\big)
	\leq \epsilon^{-2}\mathbf P_\mu \big[ \big(t^{-1} X_t(\tilde f)\big)^2 \big | X_t \neq \mathbf 0 \big].
\]
	By Proposition \ref{prop:covanrance} we have that
\begin{equation}\label{eq:general-lemma-1}\begin{split}
	&\mathbf P_\mu\big[\big(t^{-1}X_t(\tilde f)\big)^2\big|X_t \neq \mathbf 0\big]
	= t^{-2} \mathbf P_\mu (X_t\neq \mathbf 0)^{-1}\mathbf P_\mu\big[X_t(\tilde f)^2\mathbf 1_{X_t\neq \mathbf 0}\big] \\
	&\quad= t^{-1} \mathbf P_\mu (X_t\neq \mathbf 0)^{-1} \Big(\frac{\langle \mu,S_t\tilde f\rangle^2}{t} + \langle \mu,\phi \rangle\dot{\mathbb P}_{\mu\phi}\big[(\phi^{-1}\tilde f)(\xi_t)\frac{1}{t}\int_0^t A S_{t-s} \tilde f(\xi_s)ds\big]\Big).
\end{split}\end{equation}
	Letting $c,\gamma > 0$ be the constants in \eqref{eq:IU}, we know that
\begin{equation}\label{eq:ll1}\begin{split}
	| S_t\tilde f (x) - \langle \phi^* , \tilde f \rangle_m \phi(x)|
	&= \Big | \int_{E} \big(q(t,x,y) - \phi(x)\phi^*(y)\big) \tilde f (y) m(dy) \Big | \\
	&\leq \int_{E} \big|\frac{q(t,x,y)}{\phi(x)\phi^*(y)} - 1 \big| \cdot |\phi(x)\phi^*(y) \tilde f (y) | m(dy) \\
	& \leq ce^{-\gamma t} \phi(x) \|\phi^{-1}\tilde f\|_\infty \int_{E} (\phi\phi^*)(y) m(dy)\\
	& \xrightarrow[t\to\infty]{} 0,
	\quad x\in E.
\end{split}\end{equation}
Therefore, by the dominated convergence thoerem,
\[
	\langle \mu,S_t\tilde f\rangle
	\xrightarrow[t\to \infty]{} \langle \phi^*, \tilde f\rangle_m\langle \mu, \phi\rangle
	= 0.
\]
	Hence,	
\begin{equation}\label{eq:S_t}
	\frac{ \langle \mu,S_t\tilde f\rangle}{t}
	\xrightarrow [t\to\infty]{} 0,
	\quad x\in E.
\end{equation}
	By \eqref{eq:ll1} and Lemma \ref{lem:ergodic}, we know that
\[\begin{split}
	\frac{1}{t}\int_0^t A S_{t-s} \tilde f(\xi_s)ds
	&= \int_0^1 A S_{t-ut} \tilde f(\xi_{ut})d{u}\\
    &\xrightarrow[t\to\infty]{L^2(\dot{\mathbb P}_x)} \int_0^1\langle A\phi,\phi\phi^*\rangle_m \langle\phi^*,\tilde f\rangle_m du
	= 0.
\end{split}\]
	Hence, by Lemma \ref{lem:1stMomSizBiasSupProc} and the bounded convergence theorem we have that
\begin{equation}\label{eq:int}\begin{split}
	&\big| \langle \mu , \phi \rangle \dot{\mathbb P}_{\mu\phi}\big[(\phi^{-1}\tilde f)(\xi_t)\frac{1}{t}\int_0^t A S_{t-s} \tilde f(\xi_s)ds\big]\big|\\
	&\quad\leq \int \mu(dx)\phi(x) \big|\dot{\mathbb P}_x \big[(\phi^{-1}\tilde f)(\xi_t)\frac{1}{t}\int_0^t A S_{t-s} \tilde f(\xi_s)ds\big]\big| \\
	&\quad\leq \|\phi^{-1} \tilde f\|_\infty \cdot \int \mu(dx)\phi(x) \dot{\mathbb P}_{x}\Big[ \big| \frac{1}{t}\int_0^t A S_{t-s} \tilde f(\xi_s)ds \big|^2 \Big]^{\frac{1}{2}}\\
	&\quad \xrightarrow[t\to\infty]{} 0.
\end{split}\end{equation}
	Finally, using  Theorem \ref{thm:Kolmogorov-type-of-theorem} and combining  \eqref{eq:general-lemma-1}, \eqref{eq:S_t} and  \eqref{eq:int},   we have that
\[
	\mathbf P_{\mu}\big[\big(t^{-1}X_t(\tilde f)\big)^2\big|X_t \neq \mathbf 0\big]
	\xrightarrow[t\to\infty]{} 0
\]
	as required.
\end{proof}
\begin{proof}[Proof of Theorem \ref{thm:Yaglom-type-theorem}]
	Define a function $\tilde f$ by
\begin{equation}\label{eq:yy1}
	\tilde f(x)
	:=f(x) - \langle \phi^*,f\rangle_m \phi(x),
	\quad x\in E.
\end{equation}
	It is easy to see that $\tilde f\in b\mathscr B^\phi_E$ and $\langle\tilde f,\phi^* \rangle_m = 0$.
	It then follows from Lemma \ref{lem:Yaglom-for-phi} that
\begin{equation}\label{eq:yy2}
	\big\{ t^{-1}X_t(\langle \phi^*,f\rangle_m \phi);\mathbf P_\mu(\cdot | X_t\neq \mathbf 0)\big\}
	\xrightarrow[t\to\infty]{law} \frac{1}{2}\langle \phi^*,f\rangle_m\langle \phi A, \phi\phi^*\rangle_m \mathbf e,
\end{equation}
	and from Lemma \ref{lem:general-lemma} that
\begin{equation}\label{eq:yy3}
	\big\{ t^{-1} X_t(\tilde f) ; \mathbf P_\mu(\cdot|X_t \neq \mathbf 0)\big\}
	\xrightarrow[t\to\infty]{\text{in probability}}0.
\end{equation}
	The desired result then follows from \eqref{eq:yy1}, \eqref{eq:yy2}, \eqref{eq:yy3} and Slutsky's theorem.
\end{proof}
%%% end


% vim:ts=4:sw=4


	% 正文中的附录部分。
	\appendix
	% 排版参考文献列表。bibintoc 选项使“参考文献”出现在目录中;
	% 如果同时要使参考文献列表参与章节编号,可将“bibintoc”改为“bibnumbered”。
	\printbibliography[heading = bibintoc]
	% 各附录。
	% Copyright (c) 2014,2016 Casper Ti. Vector
% Public domain.

\chapter{Some appendix}
Test text.

% vim:ts=4:sw=4


	% 以下为正文之后的部分,默认不进行章节编号。
	\backmatter
	% 致谢。
	% Copyright (c) 2014,2016 Casper Ti. Vector
% Public domain.

\chapter{Acknowledgements}
Test text.

% vim:ts=4:sw=4

	% 原创性声明和使用授权说明。
	\include{chap/origin}
\end{document}

% vim:ts=4:sw=4
