\chapter{Spine decompositions of non-presistent superprocesses: characteristic functions}
	Some of the contents in this chpter are still in preparation.
\subsection{Analytic facts}
    In this subsection, we present some analytic fasts that will be useful.
\begin{lem}
\label{lem: estimate of exponential remaining}
    Suppose that $z\in \mathbb C_+:= \{x+iy: x \in [0,\infty), y \in \mathbb R\}$. Then
\begin{equation}
\label{eq: estimate of exponential remaining}
    \Big|e^{-z} - \sum_{k=0}^n \frac{(-z)^k}{k!} \Big|
    \leq \frac{|z|^{n+1}}{(n+1)!} \wedge \frac{2|z|^{n}}{n!}, \quad n\in \mathbb Z_+.
\end{equation}
\end{lem}
\begin{proof}
    Notice that $|e^{-z}| = e^{- \operatorname{Re} z} \leq 1$.
    Therefore, according to \cite[Theorem 7.20.]{Rudin2006Real},
\begin{equation}
    |e^{-z} - 1| = \Big| \int_0^1 e^{-\theta z} z d\theta\Big|
    \leq |z|.
\end{equation}
    Also, notice that $|e^{-z} - 1| \leq |e^{-z}|+1 \leq 2$.
    So we have \eqref{eq: estimate of exponential remaining} is true when $n = 0$.
    Now, suppose that \eqref{eq: estimate of exponential remaining} is true when $n = m$ for some $m \in \mathbb Z_+$.
    According to \cite[Theorem 7.20.]{Rudin2006Real},
\begin{align}
    &\Big|e^{-z} - \sum_{k=0}^{m+1} \frac{(-z)^k}{k!}\Big|
    = \Big| \int_0^1\Big(e^{-\theta z} - \sum_{k=0}^m \frac{(-\theta z)^k}{k!} \Big) z d\theta \Big|
    %\\&\quad \leq  \Big(\int_0^1 \frac{|\theta z|^{m+1}}{(m+1)!} z d\theta\Big) \wedge \Big(\int_0^1 \frac{2|\theta z|^{m}}{m!} z d\theta\Big)
    \\&\quad \leq  \Big(\int_0^1 \frac{|\theta z|^{m+1}}{(m+1)!} |z| d\theta\Big) \wedge \Big(\int_0^1 \frac{2|\theta z|^{m}}{m!} |z| d\theta\Big)
    = \frac{|z|^{m+2}}{(m+2)!} \wedge \frac{2|z|^{m+1}}{(m+1)!},
\end{align}
    which says that \eqref{eq: estimate of exponential remaining} is true for $n = m + 1$.
    Finally, use induction.
\end{proof}

\begin{lem}
\label{lem: extension lemma for branching mechanism}
    Suppose that  $\nu$ is a measure on $(0,\infty)$ such that $\int_{(0,\infty)} (u \wedge u^2) \nu(du)< \infty$. Then the following function $h: \mathbb C_+ \to \mathbb C$ is well defined:
\begin{equation}
    h (z) = \int_{(0,\infty)} (e^{-zu} - 1 + zu) \nu(du), \quad z \in \mathbb C_+.
\end{equation}
    Moreover, $h$ is continuous on $\mathbb C_+$ and is holomorphic on $\mathbb C_+^0:=\{x+iy:x \in (0,\infty), y\in \mathbb R\}$ with deriavative
\begin{equation}
\label{eq: deriavetive of the Poission part}
    h'(z) = \int_{(0,\infty)}(1- e^{-uz})u \nu(du).
\end{equation}
\end{lem}
\begin{proof}
    From Lemma \ref{lem: estimate of exponential remaining}, we know that $h$ is well defined on $\mathbb C_+$.
    According to \cite[Theorem 3.2. \& Theorem 3.5.]{SchillingSongVondracek2012Bernstein}, \eqref{eq: deriavetive of the Poission part} defines a continuous function $h'$ on $\mathbb C_+$ which is holomorphic on $\mathbb C_+^0$.
    Let $z_0, z \in \mathbb C_+$.
    Let $\gamma: [0,1]\mapsto \mathbb C_+$ be a $C^1$ path with $\gamma(0) = z_0$ and $\gamma(1) = z$.
    Notice that, according to Lemma \ref{lem: estimate of exponential remaining},
\begin{align}
    &\int_0^1 \int_0^\infty |1-e^{-u\gamma(\theta)}|u~\nu(du)~d\theta
    \\ &= \int_0^1~d\theta~ \Big( \int_0^1 |1-e^{-u\gamma(\theta)}|u~\nu(du) + \int_1^\infty |1-e^{-u\gamma(\theta)}|u~\nu(du) \Big)
    \\ &\leq \int_0^1~d\theta~ \Big( |\gamma(\theta)|\int_0^1 u^2~\nu(du) + 2\int_1^\infty u~\nu(du) \Big)
    < \infty.
\end{align}
    Therefore, according to Fubini's theorem and \cite[Theorem 7.20.]{Rudin2006Real},
\begin{align}
\label{eq: path integration representation of h}
    \int_\gamma h'(z)dz
    &=\int_0^1 h'(\gamma(\theta))\gamma'(\theta) ~d\theta
    = \int_0^1 \int_0^\infty (1-e^{-u\gamma(
    \theta)}) u~\nu(du)~ \gamma'(\theta) ~d\theta
    \\&= \int_0^\infty \int_0^1 (1-e^{-u\gamma(
    \theta)})~ \gamma'(\theta) u ~d\theta~ \nu(du)
    = h(z) - h(z_0).
\end{align}
    The rest of the proof follows the arguement in \cite[Section 10.14]{Rudin2006Real}.
\end{proof}

	For each $z\in \mathbb C\setminus (-\infty,0]$ define
\[
	\log z := \log |z| + i \arg z
\]
	where $\arg z \in (-\pi,\pi)$ is uniquely determined by
\[
	z = |z|e^{i \arg z}.
\] 	
	For each $z\in \mathbb C\setminus (-\infty,0]$ and $\gamma \in \mathbb C$ define
\[
	z^\gamma := e^{\gamma \log z}.
\]
	Then it is known, see \cite[Theorem 6.1]{SteinShakarchi2003Complex} for example, $z\mapsto \log z$ is holomophic on $\mathbb C\setminus (-\infty,0]$.
	Therefore, so is $z\mapsto z^\gamma$, for each $\gamma \in \mathbb C$.
	For convention, define $0^\gamma := \mathbf 1_{\gamma = 0}$ for each $\gamma \in \mathbb C$.
    From this definition we can verify that $(z_1z_0)^\gamma = z_1^\gamma z_0^\gamma$ provided $\arg (z_1z_0)=\arg z_1 + \arg(z_0)$.


    Recall $\Gamma$ the Gamma function defined by
\begin{equation}
    \Gamma (x) := \int_0^\infty t^{x-1} e^{-t}dt,
    \quad x>0.
\end{equation}
	It is known, see \cite[Theorem 6.1.3 and its following remark]{SteinShakarchi2003Complex} for example, function $\Gamma$ has an unique analytic extension on $\mathbb C\setminus\{0, -1,-2,\dots\}$ and that
\[
	\Gamma(z+1) = z \Gamma(z),\quad z\in \mathbb C\setminus\{0, -1,-2,\dots\}.
\]
	Using this recursively, one can verify that
\begin{align}
\label{eq: definition of Gamma function}
    \Gamma(x)
    := \int_0^\infty t^{x-1} \Big(e^{-t} - \sum_{k=0}^{n-1} \frac{(-t)^k}{k!}\Big) dt,
    \quad -n< x< -n+1, n\in \mathbb N.
\end{align}

    Fix a $\beta \in (0,1)$.
    Using the uniqueness of holomorphic extension and \cite[Theorem 3.2. \& Theorem 3.5.]{SchillingSongVondracek2012Bernstein}, we can verify that
\begin{equation}
    z^{\beta}
	= \int_0^\infty (e^{-zy}-1) \frac{dy}{\Gamma(-\beta)y^{1+\beta}},
    \quad z\in \mathbb C_+,
\end{equation}
	by showing that the both sides
\begin{itemize}
\item
    are extension of the real function $x\mapsto x^{\beta}$ on $[0,\infty)$;
\item
    are holomorphic on $\mathbb C_+^0:= \{x+iy:x\in (0,\infty), y\in \mathbb R\}$;
\item
    are continuous on $\mathbb C_+ :=\{x+iy: x\in [0,\infty), y\in \mathbb R\}$.
\end{itemize}
    Similarly, using Lemma \ref{lem: extension lemma for branching mechanism} in the Appendix, we can verify that
\begin{equation}
\label{eq: stable branching on C+}
    z^{1+\beta}
    = \int_0^\infty (e^{-zy}-1+zy)\frac{dy}{\Gamma(-1-\beta)y^{2+\beta}},
    \quad z\in \mathbb C_+.
\end{equation}
    Moreover, according to \eqref{eq: path integration representation of h}, for any $C^1$ path $\gamma:[0,1]\to \mathbb C_+$ ,
\begin{align}
\label{eq: integration formula for 1+beta-th power of z}
    &\gamma(1)^{1+\beta} - \gamma(0)^{1+\beta}
    = \int_0^1 \gamma'(\theta)d\theta \int_{(0,\infty)}(1-e^{-\gamma(\theta)y})\frac{ydy}{\Gamma(-1-\beta)y^{2+\beta}}
    \\&=\int_0^1 \gamma'(\theta)d\theta \int_{(0,\infty)}(1-e^{-\gamma(\theta)y})\frac{(-1-\beta)dy}{\Gamma(-\beta)y^{1+\beta}}
    = \int_0^1 (1+\beta) \gamma(\theta)^{\beta} \gamma'(\theta)d\theta.
\end{align}
    This also verifies that the derivative of $z\mapsto z^{1+\beta}$ is $z\mapsto (1+\beta)z^{\beta}$ on $\mathbb C^0_+$.
\begin{lem}
\label{lem: Lip of power function}
    There is a constant $(1+\beta)$ such that for any $z_0,z_1 \in \mathbb C_+$,
\begin{equation}
\label{eq: Lip of power function}
    |z_0^{1+\beta} - z_1^{1+\beta}|
    \leq (1+\beta)(|z_0|^{\beta}+|z_1|^{\beta})|z_0 - z_1|.
\end{equation}

\end{lem}
\begin{proof}
    Notice that
\begin{align}
\label{eq: upper bound for beta power of z}
	|z^\beta| 
	= |e^{\beta \log |z| +i\beta \operatorname {arg}z}| = e^{\beta \log |z|} = |z|^\beta,
	\quad z \in \mathbb C\setminus (-\infty, 0].
\end{align}
    Define a path $\gamma: [0,1] \to \mathbb C_+$ such that
\[
    \gamma(\theta)
    = z_0 (1-\theta) + \theta z_1,
    \quad \theta \in [0,1].
\]
    Then, according to \eqref{eq: integration formula for 1+beta-th power of z}, we have
\begin{align}
    |z_0^{1+\beta} - z_1^{1+\beta}|
    &\leq (1+\beta) \int_0^1 |\gamma(\theta)^{\beta}|\cdot |\gamma'(\theta)|d\theta
    \leq (1+\beta)  \sup_{\theta \in [0,1]} |\gamma(\theta)|^{\beta} \cdot |z_1-z_0|
    \\&\leq (1+\beta)  ( |z_1|^{\beta}+|z_0|^{\beta} ) |z_1-z_0|.
    \qedhere
\end{align}
\end{proof}

	Suppose that $\varphi(\theta)$ is a continuous function from $\mathbb R$ into $\mathbb C$ such that $\varphi(0) = 1$ and $\varphi(\theta) \neq 0$ for all $\theta \in \mathbb R$. 
	Then according to \cite[Lemma 7.6]{Sato2013Levy}, there is a unique continuous function $f(\theta)$ from $\mathbb R$ into $\mathbb C$ such that $f(0) = 0$ and $e^{f(\theta)} = \varphi(\theta)$. 
	Such function $f$ is called the distinguished logarithm of function $\varphi$ and is denoted as $\operatorname{Log} \varphi(\theta)$.
	In particular, let $\varphi$ be the characteristic function of an infinitely divisible random variable $Y$, then $h$ is called the Levy exponent of $Y$. 
	This distinguished logarithm should not be confused with the $\log$ function defined on $\mathbb C\setminus (-\infty, 0]$.

\subsection{Feynman-Kac formula with complex values}
\label{seq: complex Feynman-Kac transform}
%new added
    In this subsection we give a version of the Feynman-Kac formula with complex values.
%end new added
    Suppose that $\{(\xi_t)_{t \in [r,\infty)}; (\Pi_{r,x})_{r\in [0,\infty), x\in E}\}$ is a (possibly non-homogeneous) Hunt process in a locally compact seperable metric space $E$.
    %For brevity, for each $0< s< t< \infty$ and each bounded complex valued measurable function $\rho$ on $[0,\infty) \times E$, we write
    Fix a time $t >0$.
    For each $0< s< t$ and each bounded complex valued measurable function $\rho$ on $[0,t) \times E$, we write
\begin{equation}
    H^{(\rho)}_{(s,t)}:= \exp\Big\{\int_s^t \rho(u,\xi_u) du\Big\}.
\end{equation}
    Notice that
\begin{equation}
\label{eq: crucial for Feynman-Kac}
    \frac{\partial}{\partial s} H^{(\rho)}_{(s,t)}= -H^{(\rho)}_{(s,t)}\rho(s,\xi_s),
    %\quad s> 0.
    \quad s\in (0,t).
\end{equation}
    %Suppose that $\beta$, $\rho$ are complex valued bounded measurable functions on $[0,\infty) \times E$; $F$ is a complex valued bounded measurable function on $E$.
    Suppose that $\beta$, $\rho$ are complex valued bounded measurable functions on $[0,t) \times E$; $F$ is a complex valued bounded measurable function on $E$.
%delete
    %Fix a time $t >0$.
%end delete
    Define
\begin{equation}
    g(r,x) := \Pi_{r,x}[ H_{(r,t)}^{(\beta+\rho)} F(\xi_t)],\quad r \in [0,t), x\in E.
\end{equation}
    Notice that
\begin{align}
    \Pi_{r,x} \Big[ \int_r^t | H_{(r,t)}^{(\beta)}\rho(s,\xi_s) H_{(s,t)}^{(\rho)} F(\xi_t)| ~ds \Big]
    \leq  \int_r^t e^{(t-r)\|\beta\|_\infty}e^{(t-s)\|\rho\|_\infty}\|\rho\|_\infty\|F\|_\infty ~ds
    < \infty.
\end{align}
    Therefore, from the Markov property of $\xi$ and Fubini's theorem we can verify that
\begin{align}
    &\Pi_{r,x} \Big[ \int_r^tH_{(r,s)}^{(\beta)}~(\rho g)(s,\xi_s)~ds \Big]
    =\Pi_{r,x} \Big[ \int_r^t H_{(r,s)}^{(\beta)}\rho(s,\xi_s) \Pi_{s,\xi_s}[ H_{(s,t)}^{(\beta+\rho)} F(\xi_t)]~ds \Big]
    \\&= \Pi_{r,x} \Big[ \int_r^t H_{(r,t)}^{(\beta)}\rho(s,\xi_s) H_{(s,t)}^{(\rho)} F(\xi_t) ~ds \Big]
    = \Pi_{r,x} [ H_{(r,t)}^{(\beta)}F(\xi_t)(H_{(r,t)}^{(\rho)} - 1)]
    \\&= g(r,x) - \Pi_{r,x} [ H_{(r,t)}^{(\beta)} F(\xi_t)].
\end{align}

\subsection{Non-presistent superprocesses}
\label{sec: definition of superprocess}
    We say $X=\{(X_t)_{t\geq 0}; (\mathbf P_\mu)_{\mu \in \mathcal M^1_E}\}$ is a $(\xi,\psi)$-superprocess if
\begin{itemize}
\item
    The sample space $E$ is a locally compact separable metric space.
    Denote by $\mathcal M_E^1$ the collection of all the finite measures on $E$ equipped with weak topology.
\item
    The spatial motion $\xi=\{(\xi_t)_{t\geq 0};(\Pi_x)_{x\in E}\}$ is an $E$-valued Hunt process with its lifetime denoted by $\zeta$.
\item
    %the branching mechanism $\psi: E\times[0,\infty) \to \mathbb R$ is given by
    The branching mechanism $\psi: E\times[0,\infty) \to \mathbb R$ is given by
\begin{equation}
\label{eq: branching mechanism}
    \psi(x,z)=
    - \beta(x) z + \alpha (x) z^2 + \int_{(0,\infty)} (e^{-zy} - 1 + zy) \pi(x,dy).
\end{equation}
    where $\beta \in \mathcal B_b(E)$, $\alpha \in \mathcal B_b(E, \mathbb R_+)$ and $\pi(x,dy)$ is a kernel from $E$ to $(0,\infty)$ such that $\sup_{x\in E} \int_{(0,\infty)} (y\wedge y^2) \pi(x,dy) < \infty$.
\item
    $X=\{(X_t)_{t\geq 0}; (\mathbf P_\mu)_{\mu \in \mathcal M^1_E}\}$ is an $\mathcal M^1_E$-valued Hunt process with is transition probability determined by
\begin{align}
    \mathbf P_\mu [e^{-X_t(f)}] = e^{-\mu(V_tf)},
    \quad t\geq 0, \mu \in \mathcal M_E^1, f\in \mathscr B^+_b(E),
\end{align}
    where for each $f\in b\mathscr B_E$, the function $(t,x)\mapsto V_tf(x)$ on $[0,\infty) \times E$ is the unique locally bounded positive solution to the equation
\begin{align}\label{eq:FKPP_in_definition}
    V_tf(x) + \Pi_x \Big[  \int_0^{t\wedge \zeta} \psi(\xi_s,V_{t-s}f)ds \Big]
    = \Pi_x [ f(\xi_t)\mathbf 1_{t<\zeta} ],
    \quad t \geq 0, x \in E.
\end{align}
\end{itemize}
    We refer our reader to \cite{Li2011Measure-valued} for more discussion about the definition and the existence of superprocesses.
    To avoid triviality, we assume that $\psi(x,z)\neq -\beta(x)z$ for some $x \in E$ and $z \geq 0$.

    Notice that, the branching mechanism $\psi$ can be extended as a map from $E \times \mathbb C_+$ to $\mathbb C$ using \eqref{eq: branching mechanism}.
    Define
\begin{equation}
    \psi'(x,z):= - \beta(x) + 2\alpha(x) z + \int_{(0,\infty)} (1-e^{-zy})y\pi(x,dy),
    \quad x\in E, z\in \mathbb C_+.
\end{equation}
    Then according to Lemma \ref{lem: extension lemma for branching mechanism}, for each $x \in E$, $z \mapsto \psi(x,z)$ is a holomorphic function on $\mathbb C_+^0$ with deriavetive $z \mapsto \psi'(x,z)$.
    Define $\psi_0(x,z) := \psi(x,z)+ \beta(x)z $ and $\psi'_0(x,z) := \psi'(x,z) + \beta(x)$.

    Denote by $\mathbb W$ the space of $\mathcal M_E^1$-valued c\`{a}dl\`{a}g paths with its conanical path denoted by $(W_t)_{t\geq 0}$.
    We say $X$ is \emph{non-persistent} if $\mathbf P_{\delta_x}(\|X_t\|= 0) > 0$ for all $x\in E$ and $t> 0$.
    Suppose that $(X_t)_{t\geq 0}$ is non-persistent, then according to \cite[Section 8.4]{Li2011Measure-valued},
    there is a unique family of measures $(\mathbb N_x)_{x\in E}$ on $\mathbb W$ such that
\begin{itemize}
\item
    $\mathbb N_x (\forall t \geq 0, \|W_t\|=0) =0$;
\item
    $\mathbb N_x(\|W_0 \|\neq 0) = 0$;
\item
    For any $\mu \in \mathcal M_E^1$, if $\mathcal N$ is a Poisson random measure defined on some probability space
    with intensity $\mathbb N_\mu(\cdot):= \int_E \mathbb N_x(\cdot )\mu(dx)$,
    then the superprocess $\{X;\mathbf P_\mu\}$ can be realized by $\widetilde X_0 := \mu$ and $\widetilde X_t(\cdot) := \mathcal N[W_t(\cdot)]$ for each $t>0$.
\end{itemize}
    We refer to $(\mathbb N_x)_{x\in E}$ the Kuznetsov measures of $X$.
\subsection{Semigroups related to superprocesses}
\label{sec: definition of vf}
    Let $X$ be the superprocess defined in Subsection \ref{sec: definition of superprocess}.
    Define the mean semigroup
\begin{equation}
    P_t^\beta f(x)
    := \Pi_{x}[e^{\int_0^t \beta(\xi_s)ds}f(\xi_t) \mathbf 1_{t< \zeta}],
    \quad t\geq 0, x\in E, f\in b\mathscr B^+_E.
\end{equation}
    Assume that the superprocess $X$ is non-presistent with its Kuznestov measure denoted by $(\mathbb N_x)_{x\in E}$.
    It is know from \cite[Proposition 2.27]{Li2011Measure-valued} and \cite[Theorem 2.7]{Kyprianou2014Fluctuations} that for any $t > 0$, $\mu \in \mathcal M_E^1$ and $f\in b\mathscr B^+_E$,
\begin{equation}
\label{eq: mean formula for superprocesses}
    \mathbb N_{\mu}[W_t(f)]
    =\mathbf P_{\mu}[X_t(f)]=\mu(P^\beta_t f).
\end{equation}
    Define
\begin{align}
    L_1(\xi)
    &:= \{f\in \mathscr B_E: \forall x\in E, t\geq 0, \quad \Pi_x[|f(\xi_t)|]< \infty\},
    \\L_2(\xi)
    &:= \{f \in \mathscr B_E: |f|^2 \in L_1(\xi)\}.
\end{align}
    Let $f\in L_1(\xi), t >0$ and $x\in E$.
    Evaluating \eqref{eq: mean formula for superprocesses} by monotonicity and linearity, we have
\begin{equation}
    \mathbb N_x[W_t(f)]
    =\mathbf P_{\delta_x}[X_t(f)]=P^\beta_t f(x) \in \mathbb R.
\end{equation}
    Notice that, from the branching property of the superprocess $X$, $\{X_t(f); \mathbf P_{\delta_x}\}$ is an infinitely divisible random variable with finite moment.
    Denote by $U_t(\theta f)(x) := \operatorname{Log} \mathbf P_{\delta_x}[e^{i \theta X_t(f)}]$  the charateristic exponent of random variable $\{X_t(f); \mathbf P_{\delta_x}\}$.
    According to Campbell's formula, see \cite[Theorem 2.7]{Kyprianou2014Fluctuations} for example, for each $\theta \in \mathbb R$,
$   \mathbf P_{\delta_x} [e^{i\theta X_t(f)}]
    = \exp(\mathbb N_x[ e^{i\theta W_t(f)} - 1]).
$
    Notice that $\theta \mapsto \mathbb N_x[e^{i\theta W_t(f)} - 1]$ is a continuous function.
    Also notice that $\mathbb N_x[e^{i\theta W_t(f)} - 1] = 0$ if $\theta = 0$.
    Therefore, according to \cite[Lemma 7.6]{Sato2013Levy}, we have
\begin{equation}
\label{eq: N and characteristic exponent}
    U_tf(x) = \mathbb N_x[e^{i W_t(f)} - 1]
    \quad t \geq 0.
\end{equation}

    We claim that if $f\in L^2(\xi)$ then the following expectation
\[
    \Pi_x\Big[\int_0^t \psi(\xi_s,- U_{t-s}f)ds\Big]
    \in \mathbb C
\]
    is well defined. Firstly, we only assume that $f\in L^1(\xi)$.
    Noticing that
\[
     e^{\operatorname{Re} U_tf(x)}
    = |e^{U_tf(x)}|
    = |\mathbf P_{\delta_x}[e^{i X_t(f)}]|
    \leq 1,
\]
    we have
\begin{equation}
\label{eq: -v has positive real part}
 \operatorname{Re} U_tf(x)
    \leq 0.
\end{equation}
%new added
    Therefore, it is legal to talk about $\psi(x,-U_tf)$since $z\mapsto \psi(x,z)$ is well defined on $\mathbb C_+$.
    According to Lemma \ref{lem: estimate of exponential remaining}, we have that
\begin{equation}
\label{eq: upper bound for vf}
    |U_tf(x)| \leq \mathbb N_x[|e^{i W_t(f)} - 1|]
    \leq \mathbb N_x[|i W_t(f)|]
    \leq (P^\beta_t |f|)(x).
\end{equation}
    Notice that, for any compact $K \subset \mathbb R$,
\begin{equation}
\label{eq: estimate of deriavetive of v(theta)}
    \mathbb N_x\Big[\sup_{\theta \in K} \Big|\frac{\partial}{\partial \theta} (e^{i\theta W_t(f)} - 1) \Big|\Big]
    \leq \mathbb N_x[|W_t(f)|] \leq (P^\beta_t |f|)(x) < \infty.
\end{equation}
    Therefore, according to \cite[Theorem A.5.2.]{Durrett2010Probability},
    $U_t(\theta f)(x)$ is differentiable in $\theta \in \mathbb R$ with
\[
    \frac{\partial}{\partial \theta} U_t(\theta f)(x)
    = i\mathbb N_x[W_t(f)e^{i\theta W_t(f)}],
    \quad \theta \in \mathbb R.
\]
    Moreover, from the above, it is clear that
\begin{equation}
\label{eq: upper bounded for derivative of v(theta)}
    \sup_{\theta \in \mathbb R}\Big| \frac{\partial}{\partial \theta}U_t(\theta f)(x)\Big|
    \leq ( P^\beta_t |f|)(x).
\end{equation}
    From the dominate convergence theorem, we can verify that $(\partial/\partial \theta)U_tf(x)$ is continuous in $\theta$.
    In another words, $\theta \mapsto -U_t(\theta f)(x)$ is a $C^1$ map from $\mathbb R$ to $\mathbb C_+$.
    According to this and \eqref{eq: path integration representation of h}, we can write
\begin{equation}
\label{eq: path integration representation of psi(v)}
    \psi(x,-U_tf) = -\int_0^1 \psi'\big(x,-U_t(\theta f)\big) \frac{\partial}{\partial \theta} U_t(\theta f)(x)~d\theta.
\end{equation}
    Notice that
\begin{equation}
\label{eq: upper bound of psi'(v)}
\begin{split}
    &|\psi'(x, -U_tf)|
    \\&= \Big| -\beta(x)- 2\alpha(x) U_tf(x)+ \int_{(0,\infty)} y (1- e^{y U_tf(x)} ) \pi(x,dy)\Big|
    \\&= \Big| - \beta(x)- 2\alpha(x)\mathbb N_x[e^{i W_{t}(f)} - 1]  + \int_{(0,\infty)} y \mathbf P_{y \delta_x}[1-e^{i X_{t}(f)}] \pi(x,dy) \Big|
\\ &\leq \|\beta\|_\infty + 2\alpha(x)\mathbb N_x[W_t(|f|)]+ \int_{(0,\infty)} y\mathbf P_{y\delta_x}[2\wedge X_t(|f|)] \pi(x,dy)
\\ &\leq \|\beta\|_\infty + 2\|\alpha\|_\infty  P^\beta_t |f|(x) + \Big(\sup_{x\in E}\int_{(0,1]}y^2 \pi(x,dy)\Big)~P^\beta_t |f|(x) + 2\sup_{x\in E}\int_{(1,\infty)} y \pi(x,dy)
\\ &=: C_1 + C_2(P^\beta_t |f|)(x).
\end{split}
\end{equation}
    where $C_1, C_2$ are constants not dependent on $f,x$ and $t$.
    Now, using \eqref{eq: path integration representation of psi(v)}, \eqref{eq: upper bounded for derivative of v(theta)} and \eqref{eq: upper bound of psi'(v)}, we have

\begin{align}
\label{eq: upper bound of psi(v)}
    \big|\psi\big(x,-U_tf\big)\big|
    \leq C_1 P^\beta_t |f|(x)+C_2 P^\beta_t |f| (x)^2.
\end{align}
    Now assume that $f \in L_2(\xi) \subset L_1(\xi)$.
    Then, using Jensen's inequality and the above, we have that
\begin{align}
\label{eq: domination of psi(v)}
    &\Pi_x\Big[\int_0^t \big|\psi \big(\xi_s,-U_{t-s}f\big)\big|ds\Big]
    \\&\leq \Pi_x\Big[\int_0^t \big(C_1 P_{t-s}^\beta|f|(\xi_s)+C_2 P_{t-s}^\beta|f|(\xi_s)^2\big)ds\Big]
    \\ &\leq \int_0^t \big(C_1 e^{t\|\beta\|}\Pi_x \big[ \Pi_{\xi_s}[|f(\xi_{t-s})|] \big]+C_2 e^{2t\|\beta\|}\Pi_x \big[ \Pi_{\xi_s}[|f (\xi_{t-s})|]^2 \big]\big)~ds
    \\ &\leq \int_0^t (C_1 e^{t\|\beta\|}\Pi_x [ |f(\xi_{t})|]+C_2e^{2t\|\beta\|}\Pi_x [ |f (\xi_{t})|^2 ])~ds < \infty.
\end{align}
	As a consequence, expectation
\[
     \Pi_x\Big[\int_0^t \psi(\xi_s,-U_{t-s}f)ds\Big]
    \in \mathbb C
\]
    is well defined for all $f\in L_2(\xi)$.

\subsection{Main result and proof}
	Let $X$ be the non-persistent superprocess discussed in Subsection \ref{sec: definition of vf}. 
	In this subsection, we will proof the following:
	\begin{prop}
\label{prop: complex FKPP-equation}
    Let $f\in L_2(\xi)$. Then for each $t\geq 0$ and $x\in E$,
\begin{equation}
\label{eq: complex FKPP-equation}
    U_tf(x) - \Pi_x \Big[\int_0^t \psi\big(\xi_s, - U_{t-s}f\big) ds \Big]
    \\= i \Pi_x [f(\xi_t)],
\end{equation}
and
\begin{equation}
\label{eq: complex FKPP-equation with FK-transform}
    U_tf(x) -  \int_0^t P_{t-s}^{\beta} \psi_0\big(\cdot,-U_sf\big) (x)~ds
    \\= iP_t^\beta f(x).
\end{equation}
\end{prop} 

    To do this, we will need the generalized spine decomposition theorem for the superprocesses \cite{RenSongSun2017Spine} which we now recall.
    Let $X$ be the non-persistent superprocess discussed in Subsection \ref{sec: definition of vf}.
    Let $f\in b\mathscr B_E^{++}$, $T >0$ and $x\in E$.
    The fact that $\mathbf P_{\delta_x}[X_T(f)] = \mathbb N_x[W_T(f)] = P^\beta_T f(x) \in (0,\infty)$ allows us to define the following probability transforms:
\begin{equation}
    d\mathbf P_{\delta_x}^{X_T(f)}
    := \frac{X_T(f)}{P_T^\beta f(x)} d\mathbf P_{\delta_x};
    \quad d\mathbb N_x^{W_T(f)}
    :=  \frac{W_T(f)}{P_T^\beta f(x)} d\mathbb N_x.
\end{equation}
    Following the definition in \cite{RenSongSun2017Spine}, we say that $\{\xi, \mathbf n;\mathbf Q_{x}^{(f,T)}\}$ is a spine representation of $\mathbb N_x^{W_T(f)}$ if
\begin{itemize}
\item
    The spine process $\{(\xi_t)_{0\leq t\leq T}; \mathbf Q^{(f,T)}_x\}$ is a copy of $\{(\xi_t)_{0\leq t\leq T}; \Pi^{(f,T)}_{x}\}$,
    where
\begin{equation}
    d\Pi_x^{(f,T)} := \frac{f(\xi_T)e^{\int_0^T \beta(\xi_s)ds}}{P^\beta_T f(x)} d \Pi_x;
\end{equation}
\item
    Given $\{(\xi_t)_{0\leq t\leq T}; \mathbf Q^{(f,T)}_x\}$, the immigration measure $\{\mathbf n(\xi,ds,dw); \mathbf Q^{(f,T)}_x[\cdot |(\xi_t)_{0\leq t\leq T}]\}$ is a Poisson random measure on $[0,T] \times \mathbb W$ with intensity
\begin{align}
\label{eq: conditional intensity}
    \mathbf m(\xi,ds,dw)
    %:= 2 \alpha(\xi_s) ds \cdot \mathbb N_{\xi_s}(dw) + ds \cdot \int_{(0,\infty)} y \mathbf P_{y\delta_{\xi_s}}(X\in dw) \pi(\xi_s,dy);
    := 2 \alpha(\xi_s) ds \cdot \mathbb N_{\xi_s}(dw) + ds \cdot \int_{y\in (0,\infty)} y \mathbf P_{y\delta_{\xi_s}}(X\in dw) \pi(\xi_s,dy);
\end{align}
\item
    $\{(Y_t)_{0\leq t\leq T}; \mathbf Q^{(f,T)}_x\}$ is an $\mathcal M^1_E$-valued process defined by
\begin{align}
    Y_t
    := \int_{(0,t] \times \mathbb W} w_{t-s} \mathbf n(\xi,ds,dw),
    \quad 0 \leq t\leq T.
\end{align}
\end{itemize}
    According to the spine decomposition theorem \cite{RenSongSun2017Spine}, we have that
\begin{align}
\label{eq: Spine decomposition 1}
    \{(X_s)_{s \geq 0};\mathbf P_{\delta_x}^{X_T(f)}\}
    \overset{f.d.d.}{=} \{(X_s + W_s)_{s \geq 0};\mathbf P_{\delta_x} \otimes \mathbb N_x^{W_T(f)} \}
\end{align}
    and
\begin{align}
\label{eq: Spine decomposition 2}
    \{(W_s)_{0\leq s\leq T};\mathbb N_x^{W_T(f)}\}
    \overset{f.d.d.}{=} \{(Y_s)_{s \geq 0};\mathbf Q_x^{(f,T)}\}.
\end{align}

\begin{proof}[Proof of Proposition \ref{prop: complex FKPP-equation}]
    Assume that $f\in b\mathscr B_E$.
    Set $t>0, r\in [0,t), x\in E$ and $g\in b\mathscr B_E^{++}$.
    Denoted by $\{\xi, \mathbf n; \mathbf Q_x^{(g,t)}\}$ the spine representation of $\mathbb N_x^{W_t(g)}$.
    %Given $\{\xi; \mathbf Q_x^{(g,t)}\}$, denote by $\mathbf m(\xi, ds,dw)$ the conditional intensity of $\mathbf n$.
    Conditioned on $\{\xi; \mathbf Q_x^{(g,t)}\}$, denote by $\mathbf m(\xi, ds,dw)$ the conditional intensity of $\mathbf n$ in \eqref{eq: conditional intensity}.
    Denote by $\Pi_{r,x}$ the probability of Hunt process $\{\xi; \Pi\}$ initiated at time $r$ and position $x$.
    From Lemma \ref{lem: estimate of exponential remaining}, we have $\mathbf Q^{(g,t)}_{x}$-a.s.ly
\begin{align}
&\int_{[0,t]\times \mathbb W}|e^{i w_{t-s}(f)} - 1| \mathbf m(\xi, ds,dw)
    \leq \int_{[0,t]\times \mathbb W}\big(| w_{t-s}(f)| \wedge 2\big) \mathbf m(\xi, ds,dw)
    \\&\leq \int_0^t \Big(2\alpha(\xi_s)\mathbb N_{\xi_s}\big( W_{t-s}(|f|)\big)  + \int_{(0,1]} y \mathbf P_{y \delta_{\xi_s}}[ X_{t-s}(|f|)] \pi(\xi_s,dy)
    \\&\qquad\qquad+ 2\int_{(1,\infty)}y\pi(\xi_s,dy)\Big) ds
     \\&\leq \int_0^t (P_{t-s}^\beta |f|)(\xi_s)\Big(2\alpha(\xi_s)  + \int_{(0,1]} y^2 \pi(\xi_s,dy)\Big) ds + 2t \sup_{x\in E}\int_{(1,\infty)}y\pi(x,dy)
    \\&\leq \Big(2\|\alpha\|_\infty +\sup_{x\in E}\int_{(0,1]} y^2 \pi(x,dy)\Big) t e^{t\|\beta\|_\infty}\|f\|_\infty + 2t \sup_{x\in E}\int_{(1,\infty)}y\pi(x,dy)
    < \infty.
\end{align}
    According to this, Fubini's theorem, \eqref{eq: N and characteristic exponent} and \eqref{eq: -v has positive real part} we have $\mathbf Q^{(g,t)}_{x}$-a.s.ly,
\begin{align}
    &\int_{[0,t]\times \mathbb N}(e^{i w_{t-s}(f)} - 1) \mathbf m(\xi, ds,dw)
    \\&=\int_0^t \Big(2\alpha(\xi_s)\mathbb N_{\xi_s}(e^{i W_{t-s}(f)} - 1)  + \int_{(0,\infty)} y \mathbf P_{y \delta_{\xi_s}}[e^{i X_{t-s}(f)} - 1] \pi(\xi_s,dy)\Big) ds
    \\&=\int_0^t \Big( 2\alpha(\xi_s) U_{t-s} f(\xi_s) + \int_{(0,\infty)} y (e^{y U_{t-s}f(\xi_s)} - 1) \pi(\xi_s,dy) \Big) ds
    \\&= -\int_0^t \psi'_0 \big(\xi_s, -U_{t-s}f\big)ds.
\end{align}
    Therefore, according to \eqref{eq: Spine decomposition 2}, Campbell's formula and above, we have that
\begin{align}
\label{eq: N to Pi}
    \mathbb N_x^{W_t(g)}[e^{i W_t(f)}]
    &=\mathbf Q_x^{(g,t)} \Big[\exp\Big\{\int_{[0,t]\times \mathbb N}(e^{i w_{t-s}(f)} - 1) \mathbf m(\xi, ds,dw)\Big\}\Big]
    \\&= \Pi_x^{(g,t)} [e^{-\int_0^t \psi'_0(\xi_s, -U_{t-s}f)ds}]
    \\&= \frac{1}{T_t^\beta g (x)} \Pi_x[ g(\xi_t) e^{-\int_0^t \psi'(\xi_s, -U_{t-s}f)ds} ].
\end{align}
    Let $\epsilon >0$.
    Define $f^+ = (f \vee 0) + \epsilon$ and $f^- = (-f) \vee 0 + \epsilon$, then $f^\pm \in b\mathscr B^{++}_E$ and $f = f^+ - f^-$.
    According to \eqref{eq: Spine decomposition 1}, we have that
\begin{equation}
    \frac{\mathbf P_{\delta_x}[X_t(f^{\pm})e^{i X_t(f)}]}{\mathbf P_{\delta_x}[X_t(f^{\pm})]}
    = \mathbf P_{\delta_x}[e^{i X_t(f)}] \mathbb N_x^{W_t(f^{\pm})}[e^{i X_t(f)}].
\end{equation}
    Using \eqref{eq: N to Pi} and above, we have
\begin{align}
    \frac{\mathbf P_{\delta_x}[X_t(f)e^{i X_t(f)}] }{\mathbf P_{\delta_x}[e^{i X_t(f)}]}
    &= \mathbf P_{\delta_x}[X_t(f^+)] \mathbb N_x^{W_t(f^+)} [e^{i X_t(f)}] - \mathbf P_{\delta_x}[X_t(f^-)]\mathbb N_x^{W_t(f^-)}[e^{i X_t(f)}]
    \\& = \Pi_x[ f(\xi_t) e^{- \int_0^t \psi'(\xi_s, -U_{t-s}f) ds}  ].
\end{align}
    Therefore, we have
\begin{align}
    \frac{\partial}{\partial \theta} {U_t(\theta f)(x)}
    = \frac{\mathbf P_{\delta_x}[iX_t(f)e^{i X_t(f)}] }{\mathbf P_{\delta_x}[e^{i X_t(f)}]}
    =  \Pi_x[ if(\xi_t) e^{ - \int_0^t \psi'(\xi_s, -U_{t-s}(\theta f)) ds} ].
\end{align}
    Since $\{(\xi_{r+t})_{t \geq 0}; \Pi_{r,x}\} \overset{d}{=} \{(\xi_{t})_{t\geq 0}; \Pi_{x}\} $, we have

\begin{align}
    &\frac{\partial}{\partial \theta} U_{t-r}(\theta f)( x)
    = \Pi_x[ i f(\xi_{t-r}) e^{-\int_0^{t-r} \psi'(\xi_s, -U_{t-r-s}(\theta f)) ds} ]
    \\&= \Pi_{r,x}[i f(\xi_t)e^{-\int_0^{t-r} \psi'(\xi_{r+s}, -U_{t-r-s}(\theta f)) ds} ]
    = \Pi_{r,x}[if(\xi_t)e^{-\int_r^t \psi'(\xi_{s}, -U_{t-s}(\theta f)) ds} ].
\end{align}

    From \eqref{eq: upper bound of psi'(v)}, we know that for each $\theta\in \mathbb R$, $(t,x) \mapsto |\psi'(x,-U_tf(x))|$ is locally bounded (i.e. bounded on $[0,T]\times E$ for each $T \geq 0$).
    Therefore, we can apply the argument in Subsection \ref{seq: complex Feynman-Kac transform} and get that

\[
    \frac{\partial}{\partial \theta} U_{t-r}(\theta f)(x) + \Pi_{r,x} \Big[\int_r^t \psi'\big(\xi_s,- U_{t-s}(\theta f)\big)\frac{\partial}{\partial \theta} U_{t-s}(\theta f)(\xi_s)~ds\Big]
    = \Pi_{r,x} [i f(\xi_t)],
\]
    and
\begin{align}
    \frac{\partial}{\partial \theta} U_{t-r}(\theta f)(x) + \Pi_{r,x} \Big[\int_r^t e^{\int_r^s \beta(\xi_u)du}\psi_0'\big(\xi_s,- U_{t-s}(\theta f)\big)\frac{\partial}{\partial \theta} U_{t-s}(\theta f)(\xi_s)~ds\Big]
    = \Pi_{r,x} [i e^{\int_r^t \beta(\xi_s)ds}f(\xi_t)].
\end{align}
    Integrating the aboves with respect to $\theta$  on [0,1], using \eqref{eq: path integration representation of psi(v)}, \eqref{eq: upper bound of psi'(v)}, \eqref{eq: upper bounded for derivative of v(theta)} and Fubini's theorem, we have

\begin{equation}
    U_{t-r}f(x) - \Pi_{r,x} \Big[\int_r^t \psi\big(\xi_s,-U_{t-s}f\big) ~ds\Big]
    = i\theta \Pi_{r,x} [f(\xi_t)],
\end{equation}
    and similarly,
\begin{equation}
    U_{t-r}f(x) - \Pi_{r,x} \Big[\int_r^t e^{\int_r^s \beta(\xi_u)du} \psi_0\big(\xi_s,- U_{t-s}f\big) ~ds\Big]
    = i\Pi_{r,x} [e^{\int_r^t\beta(\xi_u)du}f(\xi_t)].
\end{equation}
    Taking $r = 0$, we get that \eqref{eq: complex FKPP-equation} and \eqref{eq: complex FKPP-equation with FK-transform} is true if $f\in b\mathscr B_E$.

    %Now assume that $f\in L_2(\xi)$.
    The rest of the proof is to evaluate \eqref{eq: complex FKPP-equation} and \eqref{eq: complex FKPP-equation with FK-transform} for all $f\in L_2(\xi)$. We only do this for \eqref{eq: complex FKPP-equation} since the argument for \eqref{eq: complex FKPP-equation with FK-transform} is similar.
    Set $n \in \mathbb N$.
    Writing $f_n := (f^+ \wedge n) - (f^- \wedge n)$, then $f_n \xrightarrow[n\to \infty]{} f$ pointwisely.
    From what we have proved, we have
\begin{equation}
\label{eq: complex FKPP-equation for fn}
    U_tf_n(x) - \Pi_{x} \Big[\int_0^t \psi\big(\xi_s, - U_{t-s}f_n\big) ~ds\Big]
    = i \Pi_{x} [f_n(\xi_t)].
\end{equation}
    Notice the following:
\begin{itemize}
\item
    It is clear that $\Pi_{x}[f_n(\xi_t)] \xrightarrow[n\to \infty]{} \Pi_{x}[f(\xi_t)]$.
\item
     $U_tf_n(x) \xrightarrow[n\to \infty]{} U_tf(x)$ due to \eqref{eq: N and characteristic exponent}, dominated convergence theorem and the fact that
\[
    |e^{i W_t(f_n)} - 1| \leq W_t(|f|);
    \quad \mathbb N_x[W_t(|f|)] = (P_t^\beta |f|)(x) < \infty.
\]
\item
     $\Pi_{x} [\int_0^t \psi(\xi_s,- U_{t-s}f_n)ds] \xrightarrow[n\to \infty]{} \Pi_{x} [\int_0^t \psi(\xi_s,- U_{t-s}f)ds]$ due to donimated convergence theorem, \eqref{eq: domination of psi(v)} and the fact (see \eqref{eq: upper bound of psi(v)}) that
\begin{align}
    \big|\psi(\xi_s,- U_{t-s}f_n)\big|
    \leq C_1 P_{t-s}^\beta|f|(\xi_s)+C_2 P_{t-s}^\beta|f|(\xi_s)^2.
\end{align}
\end{itemize}
    Using the above arguements, letting $n \to \infty$ in \eqref{eq: complex FKPP-equation for fn}, we get the desired result.
\end{proof}

	Define $\mathcal P^+:= \mathcal P \cap \mathcal B(\mathbb R^d, \mathbb R_+)$ and $\mathcal P^*:= \{f\in \mathcal B(\mathbb R^d, \mathbb C): |f|\in \mathcal P\}$.
    We say $S$ is a $\gamma$-scalable operator for some $\gamma\in \mathbb R$ if $S: \mathcal P^+ \to \mathcal P^+$ and $S(\theta f) = \theta^\gamma Sf$ for each $\theta \geq 0$ and $f \in \mathcal P^+$.
    We say $R$ is a monotonic operator if $R:\mathcal P^+ \to \mathcal P^+$ and $Rf \leq Rg$ for each $f, g \in \mathcal P^+$ with $f\leq g$.
    We say $(R,S)$ is a $\gamma$-control-pair for some $\gamma \in \mathbb R$ if $R$ is a monotonic operator and $S$ is a $\gamma$-scalable operator and $Rf\leq Sf$ for each $f\in \mathcal P^+$.
    We say an operator $A$ is $\gamma$-controllable on $\mathcal D $ for some $\gamma \in \mathbb R$ if $A: \mathcal D \to \mathcal P^*$ and there  is a $\gamma$-control pair $(R,S)$ such that $|Af|\leq R|f|$ for each $f\in \mathcal D$.
    In this case we say $A$ is $\gamma$-controlled by the $\gamma$-control-pair $(R,S)$ on $\mathcal D$.
    We say a family of operator $(A_s)_{s\in \Lambda}$ is uniformly $\gamma$-controllable on $\mathcal D\subset \mathcal P^*$ for some $\gamma \in \mathbb R$ if there is a $\gamma$-control pair $(R,S)$ such that, for each $s\in \Lambda$, $A_s$ is $\gamma$-controlled by $(R, S)$ on $\mathcal D$.
    \begin{comment}

\deleted{
    The first reason for considering $\gamma$-controllable operators is the following:
\begin{lem}
    Suppose that operators $(A_\lambda)_{\lambda\in \Lambda}$ are uniformly $\gamma$-controllable on $\mathcal D$ for some $\gamma \in \mathbb R$ and $\mathcal D \subset \mathcal P^*$.
    Then \added{for each $\kappa\geq0$,} there exists a $\gamma$-scalable operator $S$ such that
\[
    |A_\lambda T_t^\alpha f|
    \leq e^{\gamma t (\alpha  - \added{\kappa} b)} S\added{Q_{\kappa}}f,
    \quad \lambda \in \Lambda, t\geq 0, f\in \added{ \mathcal{D}\cap\mathcal{P}_{\kappa}}.
\]
\end{lem}

\begin{proof}
    Let $(R,S)$ be the $\gamma$-control-pair for $(A_\lambda)_{\lambda\in \Lambda}$.
    Then
\[
    |A_\lambda T_t^\alpha f| \leq R|T_t^\alpha f|
    \leq R (e^{\alpha t  - \kappa bt}Q_{\kappa}f)
    \leq S (e^{\alpha t  - \kappa bt}Q_{\kappa}f)
    \leq e^{\gamma t (\alpha  - \kappa b)} SQ_{\kappa}f.
    \qedhere
\]
\end{proof}
}
\end{comment}
	For two operators $A: \mathcal D_A \to \mathcal P^*$ and $B: \mathcal D_B \to \mathcal P^*$, define $(A\times B)f (x):= Af(x) \times Bf(x)$ for each $f\in \mathcal D_A \cap \mathcal D_B$ and $x\in \mathbb R^d$.
    Let $a > 0$, define $A^{\times a}f(x):= (Af(x))^a$ for each $f\in \mathcal D_A$ and $x\in \mathbb R^d$.
    One of the reason for considering $\gamma$-controllable operators is that they have good algebra properties:
\begin{lem}
\label{lem: property of controllable operators}
    Let operators $(A_\lambda)_{\lambda\in \Lambda}$ be uniformly $\gamma$-controllable on $\mathcal D \subset \mathcal P^*$:
\begin{itemize}
\item[(1)]
    Suppose that $(\Lambda, \mathscr F)$ is a measurable space.
    Also suppose that $(\lambda,x)\mapsto A_\lambda f(x)$ is $\mathscr F \otimes \mathscr B(\mathbb R^d)$-measurable for each $f\in \mathcal D$.
    For each probability measure $\mu$ on $(\Lambda, \mathscr F)$ write
\[
    A_\mu f(x):= \int_{\Lambda} A_\lambda f (x)~\mu(d\lambda), \quad f\in \mathcal D, x\in \mathbb R^d.
\]
    Then operators $\{A_\mu: \mu \text{ is  a probability measure on } (\Lambda, \mathscr F)\}$ are uniformly $\gamma$-controllable on $\mathcal D$.
\item[(2)]
    Suppose that operators $(B_\delta)_{\delta\in \Delta}$ is uniformly $\beta$-controllable on $\mathcal D_0 \subset \mathcal P^*$ for some $\beta \in \mathbb R$.
    Also suppose that for each $\lambda$, $A_\lambda:\mathcal D \to \mathcal D_0$.
    Then operators $(B_\delta A_\lambda)_{\delta\in \Delta, \lambda \in \Lambda}$ is uniformly $(\gamma\beta)$-controllable on $\mathcal D$.
\item[(3)]
    Suppose that operators $(B_\delta)_{\delta \in \Delta}$ is uniformly $\beta$-controllable on $\mathcal D$ for some $\beta\in \mathbb R$.
    Then operators $(B_\delta\times A_\lambda)_{\delta \in \Delta, \lambda \in \Lambda}$ are uniformly $(\gamma+\beta)$-controllable.
\item[(4)]
    Let $a>0$. Suppose that, for each $\lambda \in \Lambda$, $A_\lambda : \mathcal D \to \mathcal P^+$.
    Then operators $(A^{\times a}_\lambda)_{\lambda \in \Lambda}$ are uniformly $(a\gamma)$-controllable.
\end{itemize}
\end{lem}
\begin{proof}
    Proof of (1): Let $(R,S)$ be the $\gamma$-control-pair of $(A_\lambda)_{\lambda\in\Lambda}$ on $\mathcal{D}$. For each $f \in \mathcal{D}$ and $\mu$ the probability measure on $(\Lambda, \mathscr F)$.
\[
   |A_{\mu}f(x)|\leq \int_{\Lambda}|A_{\lambda}f(x)|\mu(d\lambda) \leq \int_{\Lambda}R|f|(x)\mu(d\lambda) \leq R|f|(x).
\]

   	Proof of (2): Let $(R_A, S_A)$ be the $\gamma$-control-pair of $(A_\lambda)_{\lambda\in\Lambda}$ on $\mathcal{D}$ and $(R_B, S_B)$ be the $\beta$-control-pair of $(B_{\delta})_{\delta\in\Delta}$ on $\mathcal{D}_0$.
	Note that $(R_BR_A, S_BS_A)$ is a $\beta \gamma$-control-pair.
	In fact:
\begin{itemize}
\item
	For each $f,g \in \mathcal P^+$ with $f\leq g$, since $R_Af \leq R_A g$ we have $R_B(R_A f)\leq R_B(R_A g)$.
\item
	For each $f\in \mathcal{P}^+$ and $\theta \geq 0$, we have $S_BS_A(\theta f)=S_B(\theta^{\gamma}S_Af)=\theta^{\beta\gamma}S_BS_Af$.
\item
	For each $f,g \in \mathcal P^+$ with $f\leq g$, we have $R_B R_A f \leq R_B S_A f \leq S_BS_A f$.
\end{itemize}
	Finally, note that for each $\delta\in \Delta, \lambda\in\Lambda$ and $f\in \mathcal D$, $|B_{\delta}A_{\lambda}f|\leq R_B|A_{\lambda}f|\leq R_BR_A|f|$ which says that operators $(B_\delta\times A_\lambda)_{\delta \in \Delta, \lambda \in \Lambda}$ are uniformly $(\beta\gamma)$-controlled by $(R_BR_A,S_BS_A)$.

   Proof of (3) and (4): Similar to the Proof of (2).
\end{proof}
\subsection{}
    For each $f \in \mathcal{P}$, $x\in \mathbb{R}^d$ and $t\geq 0$, define
\begin{align}
\label{eq: def of Zf}
    \tilde U_t f(x)
    &:= i T^\alpha_t f(x) + \int_0^t T^\alpha_{t-s} \Psi_0(-i T_s^{\alpha}f)(x)ds
    \\Z_t f (x)
    &:= \int_0^t T^\alpha_{t-s} \Psi_0(-i T_s^{\alpha}f)(x)ds.
\end{align}

\begin{lem}
\label{lem: upper bound for usgx}
The following statements are true for the super-OU process considered in this chapter:
\begin{itemize}
\item[(1)]
    Operators $(-U_t)_{0\leq t\leq 1}$ are uniformly $1$-controllable from $\mathcal P$ to $\mathcal P^*\cap \mathcal B(\mathbb R^d, \mathbb C_+)$.
\item[(2)]
    Operators $(T^\alpha_t)_{0\leq t\leq 1}$ are uniformly $1$-controllable on $\mathcal P$.
\item[(3)]
    Operator $\Psi_0$ is $(1+\beta)$-controllable on $\mathcal P^* \cap \mathcal B(\mathbb R^d, \mathbb C_+)$.
\item[(4)]
    Operators $(U_t- iT_t^{\alpha})_{0\leq t\leq 1}$ are uniformly $(1+\beta)$-controllable on $\mathcal P$.
\item[(5)]
    Operators $\{\Psi_0(-U_t) - \Psi_0(-iT_t^\alpha): 0\leq t\leq 1\}$ are uniformly $(1+2\beta)$-controllable on $\mathcal P$.
\item[(6)]
    Operator $(U_t-\tilde U_t)_{0\leq t\leq 1}$ are uniformly $(1+2\beta)$-controllable on $\mathcal P$.
\item[(7)]
    Operators $(Z_t)_{0\leq t\leq 1}$ are uniformly $(1+\beta)$-controallable on $\mathcal P$.
\end{itemize}
\end{lem}

\begin{proof}
    Proof of (1): From \eqref{eq: upper bound for vf} in the appendix, we have for each $g\in \mathcal P$, $0\leq t\leq 1$ and $x\in \mathbb R^d$,
\[
    |U_t g(x)|
    \leq \sup_{0\leq u\leq 1}T_u^\alpha |g| (x).
\]
    Note that $f\mapsto\sup_{0\leq u\leq 1}T^{\alpha}_u|f|$ is monotonic and $1$-scalable.

    Proof of (2): Similar to the Proof of (1).

    Proof of (3): According to Lemma \ref{lem: Lip of power function}, for each $f\in \mathcal P^* \cap \mathcal B(\mathbb R^d, \mathbb C_+)$,
\[
    |\Psi_0 f(x)| = |f(x)^{1+\beta}| = |f(x)|^{1+\beta}.
\]
    Also note that $f\mapsto |f|^{1+\beta}$ is monotonic and $(1+\beta)$-scalable.

    Proof of (4): From (1), (2), (3) and Lemma \ref{lem: property of controllable operators}(2) we know that operators
\[
    f
    \mapsto T^{\alpha}_{t-s}\Psi_0(-U_sf)(x),
    \quad 0\leq s\leq t\leq 1
\]
    are uniformly $(1+\beta)$-controllable.
    From \eqref{eq:chareq2}, Lemma \ref{lem: property of controllable operators}(1) and this we get the desired result.

    Proof of (5): Notice that from Lemma \ref{lem: Lip of power function},
\[
    |\Psi_0(-U_t f) - \Psi_0(-iT_t^\alpha f) |
    \leq  (1+\beta) |U_u f-iT_u^{\alpha}f|(|U_u f|^{\beta}+|i T_u^{\alpha}f|^{\beta}).
\]
    From (1), (2), (4) and Lemma \ref{lem: property of controllable operators} (3,4) we get the desired result.

    Proof of (6): Note that
\[
    U_sf - \tilde U_sf
    = \int_0^s T_{s-u}^{\alpha}\big(\Psi_0(-U_t f)-\Psi_0(-i T_t^{\alpha}f)\big)~du.
\]
    Then from (2), (5) and Lemma \ref{lem: property of controllable operators}(1,2) we get the desired result.

    Proof of (7): Similar to the proof of (4).
\end{proof}
\begin{comment}
\textbf{ Zhenyao: I think the following Corollary is incorrect. In its proof, we seems used Jensen's inequality assuming $\mu$ is a probability measure (which is not the case). At first, I think I may find a way to fix this. But then I realize we don't really need this Corrollary to calculate $\|X_t(g)\|_{1+\gamma}$. So I didn't touch it. Yet, I'm not sure if this Corrollary is needed somewhere else in section 3.}

\begin{cor}
\label{cor: corollary1}
    There exists an $R^g_3 \in \mathcal{P}$ such that for any $\mu\in \mathcal M(\mathbb R^d)$ with compact support, $k \in \mathbb{N}$,$\theta \in \mathbb{R}$ and $0\leq s\leq 1$, we have
\[
    \mathbb P_{\mu}[e^{i\theta(\langle g_k, X_s\rangle-\langle T_s^{\alpha}g_k,\mu \rangle)}]
    =1+\langle Z_{g_k}(s,\cdot,\theta),\mu\rangle+ err(k,\mu,\theta),
\]
    where $|err(k,\mu,\theta)| \leq (C_g|\theta e^{(\alpha-\kappa b)k}|^{2+2\beta} + C_g|\theta e^{(\alpha-\kappa b)k}|^{1+2\beta} )\langle R^g_3,\mu\rangle$.
\end{cor}
\begin{proof}
    Let
\begin{align}
\label{eq: definition of varphi-mu-k-theta}
    \varphi_{k,\mu,s}(\theta)
    =\mathbb{P}_{\mu}[e^{i\theta(\langle g_k, X_s\rangle-\langle T_s^{\alpha}g_k,\mu \rangle)}]
    =e^{(\langle v_{g_k}(s,\cdot,\theta),\mu \rangle-i\theta \langle T_s^{\alpha} g_k, \mu \rangle)},
    \quad \theta \in \mathbb R.
\end{align}
    Then we have
\begin{align*}
    &|\varphi_{k,\mu,s}(\theta)-1-\langle Z_{g_k}(s,\cdot, \theta),\mu\rangle|\\
    &\leq|\varphi_{\mu,k,s}(\theta)-1-\left( \langle v_{g_k}(s,\cdot,\theta), \mu \rangle-i\theta \langle T_s^{\alpha}g_k,\mu\rangle\right)| + |\langle v_{g_k}(s,\cdot,\theta)-\tilde{v}_{g_k}(s,\cdot,\theta),\mu \rangle|.
\end{align*}
    Let $R^g_1\in \mathcal P$ be the control function in Lemma \ref{lemma2}.
    Let $\tilde R^g$ be the control function in \eqref{eq: Zineq}.
    Notice that
$
    \operatorname{Re} \big(v_{g_k}(s,x,\theta) - i\theta T_s^\alpha g_k(x)\big)
    = \operatorname{Re} v_{g_k}(s,x,\theta)
    \leq 0,
$
    (see \eqref{eq: -v has positive real part} in the appendix); and if $ \operatorname{Re} z\leq 0$, we have $|e^z-1-z|\leq |z|^2$, (see Lemma \ref{lem: estimate of exponential remaining} in the appendix).
    Therefore, for any $\mu\in \mathcal M(\mathbb R^d)$ with compact support, $k \in \mathbb{N}$, $\theta \in \mathbb{R}$ and $0\leq s\leq1$, using Jensen's inequality,
\begin{align*}
    &|\varphi_{k,\mu,s}(\theta)-1- \langle Z_{g_k}(s,\cdot, \theta),\mu\rangle|\\
    &\leq |\langle v_{g_k}(s,\cdot,\theta)-i\theta T_s^{\alpha}g_k, \mu \rangle|^2 + \langle |v_{g_k}(s,\cdot,\theta)-\tilde{v}_{g_k}(s,\cdot,\theta)|, \mu \rangle
    \\&\leq \langle |v_{g_k}(s,\cdot,\theta)-i\theta T_s^{\alpha}g_k|^2, \mu \rangle + \langle |v_{g_k}(s,\cdot,\theta)-\tilde{v}_{g_k}(s,\cdot,\theta)|, \mu \rangle
    \\&\leq e^{2\alpha}|C_g\theta e^{(\alpha-\kappa b)k+\alpha}|^{2+2\beta}\langle (\tilde R^g)^2,\mu\rangle + |C_g\theta e^{(\alpha-\kappa b)k}|^{1+2\beta}\langle R^g_1,\mu\rangle.
    \qedhere
\end{align*}
\end{proof}
\end{comment}

\subsection{Application: the tail probability estimation}

 In this subsection, for any $\gamma \in (0,\beta)$, we want to bound the $(1+\gamma)$-th moment of $\langle g ,X_t \rangle$. Denote $\mathcal{M}_c(\mathbb{R}^d)$ by the space of all finite measures on $\mathbb{R}^d$ with compact support.
	For each $0 \leq a \leq b <\infty$ and  random variable $Y \in$ with finite mean, define random variable
$
   	\mathcal I_a^b Y
    := \mathbb P[Y|\mathscr F_b] - \mathbb P[Y|\mathscr F_a].
$
\begin{lem}
\label{lem: control pair for P(M>lambda)}
    There is an $(1+\beta)$-control-pair $(R,S)$ such that for each $0\leq t\leq 1$, $g\in \mathcal P$, $\lambda >0$ and $\mu\in \mathcal M_c(\mathbb R^d)$ we have
\[
    \mathbb P_\mu ( |\mathcal{I}_0^t\langle g,X_t\rangle| > \lambda)
    \leq \frac{\lambda}{2}\int_{-2/\lambda}^{2/\lambda}\langle R|\theta g|,\mu\rangle d\theta.
\]
\end{lem}

\begin{proof}
    Denote by $(R,S)$ the $(1+\beta)$-control pair for Lemma \ref{lem: upper bound for usgx}(4).
    Using Lemma \ref{lem: estimate of exponential remaining} and the argument in \cite[Proof of Theorem 3.3.6]{Durrett2010Probability}, we have
\begin{align}
    &\big|\mathbb P_\mu (|\mathcal{I}_0^t\langle g,X_t\rangle| > \lambda)\big|
    \leq \Big|\frac{\lambda}{2}\int_{-2/\lambda}^{2/\lambda}(1 - \mathbb P_\mu[e^{i\theta \mathcal{I}_0^t\langle g,X_t\rangle]})d\theta\Big|
    \\&\leq \frac{\lambda}{2}\int_{-2/\lambda}^{2/\lambda}|1-e^{\langle U_t(\theta g)-iT_t^{\alpha}(\theta g),\mu \rangle}|d\theta
    \leq \frac{\lambda}{2}\int_{-2/\lambda}^{2/\lambda}\langle |U_t(\theta g) - iT_t^\alpha(\theta g)|,\mu\rangle d\theta
    \\&\leq \frac{\lambda}{2}\int_{-2/\lambda}^{2/\lambda}\langle R|\theta g|,\mu\rangle d\theta.
      \qedhere
\end{align}
\end{proof}

% vim:ts=4:sw=4
