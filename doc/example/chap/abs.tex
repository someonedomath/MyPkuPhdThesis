% Copyright (c) 2014,2016 Casper Ti. Vector
% Public domain.

\begin{cabstract}
	中文摘要。
\end{cabstract}

\begin{eabstract}
	This thesis focusses on the asymptotic behaviors of some Markovian branching processes in the critical regime.
	Properties and relationships between the multi-spine theory and the limiting theory of some Markovian branching processes are considered.
	In particular, we study systematically the spine decompositions of critical Galton-Watson trees and critical superprocesses, and their Kolmogrov type, Yaglom type and Slack type results. 

	We begin by proposing a two-spine decomposition of the critical Galton-Watson tree and useing that decomposition to give a new probabilistic proof of Yaglom’s theorem.
	
	Next, we establish a spine decomposition theorem and a 2-spine decomposition theorem for some critical superprocesses. These two kinds of decompositions are unified as a decomposition theorem for size-biased Poisson random measures. We use these decompositions to give probabilistic proofs of the asymptotic behavior of the survival probability and Yaglom’s exponential limit law for some critical superprocesses with second moments.

	Then, we proof that the characteristic functions of superprocesses satisfy a mild complex-valued integral equation using the spine decomposition theory developed here. Using this equation we estimate the tail probability of superprocess with stable branching.

	Finally, we consider a critical superprocess $\{X;\mathbf P_\mu\}$ with general spatial motion and spatially dependent stable branching mechanism with lowest stable index $\gamma_0 > 1$. We show that, under some conditions, $\mathbf P_{\mu}(\|X_t\|\neq 0)$ converges to $0$ as $t\to \infty$ and is regularly varying with index $(\gamma_0-1)^{-1}$. Then we proof the Slack type result that for a large class of non-negative testing functions $f$, the distribution of $\{X_t(f);\mathbf P_\mu(\cdot|\|X_t\|\neq 0)\}$, after appropriate rescaling, converges weakly to a positive random variable $\mathbf z^{(\gamma_0-1)}$ with Laplace transform $E[e^{-u\mathbf z^{(\gamma_0-1)}}]=1-(1+u^{-(\gamma_0-1)})^{-1/(\gamma_0-1)}.$
\end{eabstract}

% vim:ts=4:sw=4
